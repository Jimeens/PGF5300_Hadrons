\RequirePackage{pdfmanagement-testphase} % tranparent package doesn't work without this
\DeclareDocumentMetadata{} % and this
\documentclass[a4paper, 11pt, oneside]{impression}

\hwnumber{2} % Homework number in the cover page
\title{Lista de Exercícios} % Title
\subtitle{Introdução a Física de Hádrons} % Subtitle
\author{Lucas R. Ximenes} % Author
\extra{11917239} % Extra information

\begin{document}

\maketitle

\chapter{Funcional gerador de um campo escalar livre}\label{quest: one}

\begin{exercise}{}
    O funcional gerador de um campo escalar livre é escrito como
        \begin{equation*}
            Z_{0}[J] = N \int \exp\qty{
                i \int \qty[
                    \dfrac{1}{2}\qty(
                        \xi\vec{K}_{x}\xi +
                        \xi J +
                        \bar{\phi}\vec{K}_{x}\xi + 
                        \bar{\phi}J
                    ) + 
                    J\bar{\phi} + 
                    J\xi
                ]\dd[4]{x}
            }[D\xi]
        \end{equation*}
    após uma expansão do campo original $\phi$ em torno de sua configuração clássica $\bar{\phi}$ e sua correspondente flutuação $\xi$, $\phi = \bar{\phi} + \xi$. O operador $\vec{K}_{x}$ é o de Klein-Gordon, $\vec{K}_{x} = (\partial^{2} + m^{2})_{x}$. Obtenha
        \begin{equation*}
            Z_{0}[J] = N'\exp\qty[
                -\dfrac{i}{2}\int J(x)\Delta(x-y)J(y)\dd[4]{x}\dd[4]{y}
            ]
        \end{equation*}
\end{exercise}

Na expressão original do problema, podemos identificar que o termo $\dfrac{1}{2}\xi\vec{K}_{x}\xi$ é um termo quadrático em $\xi$, a soma dos termos $\dfrac{1}{2}\xi J + \dfrac{1}{2}\bar{\phi}\vec{K}_{x}\xi + J\xi$ é um termo linear em $\xi$ e os termos restantes independem dessa variável, o que sugere uma integração gaussiana em $\xi$.

\begin{itemize}
    \item Assumir que $\vec{K}_{x}$ é uma matriz diagonal
    \item Discretizar os campos pra usar a expressão da aula 7/8 (eu acho)
    \item Fazer uma mudança de variável conveniente
    \item Integrar em $[D\xi]$
\end{itemize}

\chapter{Integração gaussiana com números de Grassman}\label{quest: two}

\begin{exercise}{}
    A integral de trajetória para férmions envolve o cálculo, no limite do contínuo, da integral discretizada
        \begin{equation*}
            I(\vb{A}) = \int \exp\qty(\sum_{j,k=1}^{N}\bar{\theta}_{j}A_{jk}\theta_{k})\prod_{i=1}^{N}\dd{\theta_{i}}\dd{\bar{\theta}_{i}} = \text{det}(\vb{A})
        \end{equation*}
    Mostre explicitamente o resultado acima para $N = 3$, e depois, generalize para um $N$ arbitrário.
\end{exercise}

Para $N=3$, temos diretaamente que
    \begin{align*}
        I(\vb{A}) &\eq \int \exp\qty(\sum_{j,k=1}^{3}\bar{\theta}_{j}A_{jk}\theta_{k})\prod_{i=1}^{3}\dd{\theta_{i}} \dd{\bar{\theta}_{i}} \\
    \end{align*}

Abrindo a exponencial em uma série de Taylor, temos
    \begin{align*}
        I(\vb{A}) &\eq \int \qty[
            1 + \sum_{j,k=1}^{3}\bar{\theta}_{j}A_{jk}\theta_{k} + 
            \dfrac{1}{2!}\qty(\sum_{j,k=1}^{3}\bar{\theta}_{j}A_{jk}\theta_{k})^{2} + 
            \dfrac{1}{3!}\qty(\sum_{j,k=1}^{3}\bar{\theta}_{j}A_{jk}\theta_{k})^{3} + \cdots
        ]\prod_{i=1}^{3}\dd{\theta_{i}}\dd{\bar{\theta}_{i}} \\
        &\eq \int \prod_{i=1}^{3}\dd{\theta_{i}}\dd{\bar{\theta}_{i}} + 
        \int \sum_{j,k=1}^{3}\bar{\theta}_{j}A_{jk}\theta_{k}\prod_{i=1}^{3}\dd{\theta_{i}}\dd{\bar{\theta}_{i}} + \dfrac{1}{2!}\int \qty(\sum_{j,k=1}^{3}\bar{\theta}_{j}A_{jk}\theta_{k})^{2}\prod_{i=1}^{3}\dd{\theta_{i}}\dd{\bar{\theta}_{i}} + \\
        &\noeq + \dfrac{1}{3!}\int \qty(\sum_{j,k=1}^{3}\bar{\theta}_{j}A_{jk}\theta_{k})^{3}\prod_{i=1}^{3}\dd{\theta_{i}}\dd{\bar{\theta}_{i}} + \cdots
    \end{align*}

O primeiro termo é claramente nulo, pois não há variáveis de Grassman para integrar. O segundo termo também é nulo, pois cada termo da soma possui apenas um $\theta$ e um $\bar{\theta}$, e portanto, ao integrar sobre as outras variáveis, o resultado será zero. O terceiro termo também é nulo, pois cada termo da soma ao quadrado terá no máximo dois $\theta$ e dois $\bar{\theta}$, e portanto, ao integrar sobre as outras variáveis, o resultado será zero. Restando apenas o quarto termo, onde também podemos levar em conta que qualquer termo subsequente da expansão de Taylor será nulo, pois terá mais de três $\theta$ ou $\bar{\theta}$. 

O quarto termo pode ser escrito como
    \begin{align*}
        I(\vb{A}) &\eq \dfrac{1}{3!}\int \sum_{j,k,\ell,m,n,p=1}^{3}\bar{\theta}_{j}A_{jk}\theta_{k}\bar{\theta}_{\ell}A_{\ell m}\theta_{m}\bar{\theta}_{n}A_{np}\theta_{p}\prod_{i=1}^{3}\dd{\theta_{i}}\dd{\bar{\theta}_{i}}
    \end{align*}

Note que, para que a integral não seja nula, é necessário que $j \neq \ell \neq n$, assim como $k \neq m \neq p$. Um ponto a se notar é que temos essencialmente $9^{3}$ termos dentro dessa integral, mas muitos deles são idênticos, pois a ordem dos fatores não importa. Por exemplo, o termo com $j=1$, $\ell=2$, $n=3$, $k=1$, $m=2$ e $p=3$
    \begin{equation*}
        \bar{\theta}_{1}A_{11}\theta_{1}\bar{\theta}_{2}A_{22}\theta_{2}\bar{\theta}_{3}A_{33}\theta_{3}
    \end{equation*}
é idêntico ao termo com $j=2$, $\ell=1$, $n=3$, $k=2$, $m=1$ e $p=3$
    \begin{equation*}
        \bar{\theta}_{2}A_{22}\theta_{2}\bar{\theta}_{1}A_{11}\theta_{1}\bar{\theta}_{3}A_{33}\theta_{3}
    \end{equation*}
pois levando em conta a anticomutatividade dos números de Grassmann, faremos 4 trocas de posição para chegar de um termo ao outro, o que é equivalente a multiplicar por $(-1)^{4} = 1$. Isto faz com que muitos termos sejam idênticos, e portanto, possamos considerar apenas um representante de cada conjunto de termos idênticos. Note que, para cada conjunto de termos idênticos, há exatamente $3! = 6$ termos, pois há $3!$ maneiras de ordenar os índices $j$, $\ell$ e $n$, e outras $3!$ maneiras de ordenar os índices $k$, $m$ e $p$. Portanto, podemos eliminar o fator $1/3!$ que está na frente da integral. Os representantes não-nulos formam então o seguinte resultado
    \begin{align*}
        I(\vb{A}) &\eq \int (
            \bar{\theta}_{1}A_{11}\theta_{1}\bar{\theta}_{2}A_{22}\theta_{2}\bar{\theta}_{3}A_{33}\theta_{3} + 
            \bar{\theta}_{1}A_{11}\theta_{1}\bar{\theta}_{2}A_{23}\theta_{3}\bar{\theta}_{3}A_{32}\theta_{2} + 
            \bar{\theta}_{1}A_{12}\theta_{2}\bar{\theta}_{2}A_{21}\theta_{1}\bar{\theta}_{3}A_{33}\theta_{3} + \\
            &\noeq + 
            \bar{\theta}_{1}A_{12}\theta_{2}\bar{\theta}_{2}A_{23}\theta_{3}\bar{\theta}_{3}A_{31}\theta_{1} + 
            \bar{\theta}_{1}A_{13}\theta_{3}\bar{\theta}_{2}A_{21}\theta_{1}\bar{\theta}_{3}A_{32}\theta_{2} + 
            \bar{\theta}_{1}A_{13}\theta_{3}\bar{\theta}_{2}A_{22}\theta_{2}\bar{\theta}_{3}A_{31}\theta_{1}
        ) \times \\
        &\noeq \times 
            \dd{\theta_{1}} \dd{\bar{\theta}_{1}} \dd{\theta_{2}} \dd{\bar{\theta}_{2}} \dd{\theta_{3}} \dd{\bar{\theta}_{3}} \\
        &\eq 
        \int 
            \bar{\theta}_{1}\theta_{1}\bar{\theta}_{2}\theta_{2}\bar{\theta}_{3}\theta_{3}(A_{11}A_{22}A_{33}) 
            \dd{\theta_{1}} \dd{\bar{\theta}_{1}} \dd{\theta_{2}} \dd{\bar{\theta}_{2}} \dd{\theta_{3}} \dd{\bar{\theta}_{3}} + 
        \int 
            \bar{\theta}_{1}\theta_{1}\bar{\theta}_{2}\theta_{3}\bar{\theta}_{3}\theta_{2}(A_{11}A_{23}A_{32})\times \\
        &\noeq \times 
            \dd{\theta_{1}} \dd{\bar{\theta}_{1}} \dd{\theta_{2}} \dd{\bar{\theta}_{2}} \dd{\theta_{3}} \dd{\bar{\theta}_{3}} + 
        \int 
            \bar{\theta}_{1}\theta_{2}\bar{\theta}_{2}\theta_{1}\bar{\theta}_{3}\theta_{3}(A_{12}A_{21}A_{33}) 
            \dd{\theta_{1}} \dd{\bar{\theta}_{1}} \dd{\theta_{2}} \dd{\bar{\theta}_{2}} \dd{\theta_{3}} \dd{\bar{\theta}_{3}} + \\
        &\noeq + 
        \int 
            \bar{\theta}_{1}\theta_{2}\bar{\theta}_{2}\theta_{3}\bar{\theta}_{3}\theta_{1}(A_{12}A_{23}A_{31}) 
            \dd{\theta_{1}} \dd{\bar{\theta}_{1}} \dd{\theta_{2}} \dd{\bar{\theta}_{2}} \dd{\theta_{3}} \dd{\bar{\theta}_{3}} + 
        \int 
            \bar{\theta}_{1}\theta_{3}\bar{\theta}_{2}\theta_{1}\bar{\theta}_{3}\theta_{2}(A_{13}A_{21}A_{32}) \times \\
        &\noeq \times 
            \dd{\theta_{1}} \dd{\bar{\theta}_{1}} \dd{\theta}_{2} \dd{\bar{\theta}_{2}} \dd{\theta}_{3} \dd{\bar{\theta}_{3}} + 
        \int 
            \bar{\theta}_{1}\theta_{3}\bar{\theta}_{2}\theta_{2}\bar{\theta}_{3}\theta_{1}(A_{13}A_{22}A_{31}) 
            \dd{\theta_{1}} \dd{\bar{\theta}_{1}} \dd{\theta_{2}} \dd{\bar{\theta}_{2}} \dd{\theta}_{3} \dd{\bar{\theta}_{3}}
    \end{align*}

Aqui precisamos levar em conta que a ordem da integração importa, portanto é necessário que a ordem dos fatores dentro da integral seja a mesma que a ordem de integração, que é
    \begin{equation*}
        \dd{\theta_{1}} \to \dd{\bar{\theta}_{1}} \to \dd{\theta_{2}} \to \dd{\bar{\theta}_{2}} \to \dd{\theta_{3}} \to \dd{\bar{\theta}_{3}}
    \end{equation*}

Sendo assim, os termos ficam:
    \begin{align*}
        \bar{\theta}_{1}\theta_{1}\bar{\theta}_{2}\theta_{2}\bar{\theta}_{3}\theta_{3} &= (-1)^{12} \bar{\theta}_{3}\theta_{3}\bar{\theta}_{2}\theta_{2}\bar{\theta}_{1}\theta_{1} = \bar{\theta}_{3}\theta_{3}\bar{\theta}_{2}\theta_{2}\bar{\theta}_{1}\theta_{1} \\
        \bar{\theta}_{1}\theta_{1}\bar{\theta}_{2}\theta_{3}\bar{\theta}_{3}\theta_{2} &= (-1)^{11}\bar{\theta}_{3}\theta_{3}\bar{\theta}_{2}\theta_{2}\bar{\theta}_{1}\theta_{1} = -\bar{\theta}_{3}\theta_{3}\bar{\theta}_{2}\theta_{2}\bar{\theta}_{1}\theta_{1}\\
        \bar{\theta}_{1}\theta_{2}\bar{\theta}_{2}\theta_{1}\bar{\theta}_{3}\theta_{3} &= (-1)^{11}\bar{\theta}_{3}\theta_{3}\bar{\theta}_{2}\theta_{2}\bar{\theta}_{1}\theta_{1} = -\bar{\theta}_{3}\theta_{3}\bar{\theta}_{2}\theta_{2}\bar{\theta}_{1}\theta_{1}\\
        \bar{\theta}_{1}\theta_{2}\bar{\theta}_{2}\theta_{3}\bar{\theta}_{3}\theta_{1} &= (-1)^{10}\bar{\theta}_{3}\theta_{3}\bar{\theta}_{2}\theta_{2}\bar{\theta}_{1}\theta_{1} = \bar{\theta}_{3}\theta_{3}\bar{\theta}_{2}\theta_{2}\bar{\theta}_{1}\theta_{1}\\
        \bar{\theta}_{1}\theta_{3}\bar{\theta}_{2}\theta_{1}\bar{\theta}_{3}\theta_{2} &= (-1)^{8}\bar{\theta}_{3}\theta_{3}\bar{\theta}_{2}\theta_{2}\bar{\theta}_{1}\theta_{1} = \bar{\theta}_{3}\theta_{3}\bar{\theta}_{2}\theta_{2}\bar{\theta}_{1}\theta_{1}\\
        \bar{\theta}_{1}\theta_{3}\bar{\theta}_{2}\theta_{2}\bar{\theta}_{3}\theta_{1} &= (-1)^{7}\bar{\theta}_{3}\theta_{3}\bar{\theta}_{2}\theta_{2}\bar{\theta}_{1}\theta_{1} = -\bar{\theta}_{3}\theta_{3}\bar{\theta}_{2}\theta_{2}\bar{\theta}_{1}\theta_{1}
    \end{align*}

Essa ordenação é interessante, pois ao integrarmos, todos terão o mesmo valor a menos de um sinal. Este valor é
    \begin{equation*}
        \int \bar{\theta}_{3}\theta_{3}\bar{\theta}_{2}\theta_{2}\bar{\theta}_{1}\theta_{1} \dd{\theta_{1}} \dd{\bar{\theta}_{1}} \dd{\theta_{2}} \dd{\bar{\theta}_{2}} \dd{\theta}_{3} \dd{\bar{\theta}_{3}} = 1
    \end{equation*}

Portanto, $I(\vb{A})$ pode ser escrito como
    \begin{align*}
        I(\vb{A}) &\eq A_{11}A_{22}A_{33} - A_{11}A_{23}A_{32} - A_{12}A_{21}A_{33} + A_{12}A_{23}A_{31} + A_{13}A_{21}A_{32} - A_{13}A_{22}A_{31}
    \end{align*}
que é exatamente a definição de $\text{det}(\vb{A})$ para uma matriz $\vb{A}$ de $\text{dim}(\vb{A}) = 3$. Mostrando então que
    \begin{answer}\label{eq: answer 2 pt 1}
        I(\vb{A}) = \int \exp\qty(\sum_{j,k=1}^{3}\bar{\theta}_{j}A_{jk}\theta_{k})\prod_{i=1}^{3}\dd{\theta_{i}}\dd{\bar{\theta}_{i}} = \text{det}(\vb{A})
    \end{answer}

\chapter{Relações de campos de gauge não-abelianos}\label{quest: three}

\begin{exercise}{}
    No caso de um campo de gauge não-abeliano, mostre explicitamente que
    \begin{enumerate}[(a)]
        \item $[D_{\mu}, D_{\nu}]\Phi = -igF_{\mu\nu}\Phi$.
        \item $F_{\mu\nu}' \Phi' = UF_{\mu\nu}\Phi$.
    \end{enumerate}
    em que $\Phi$ é um vetor coluna de campos de bósons escalares que satisfaz a simetria de gauge do grupo, e $U = \exp[T_{a}\theta_{a}(x)]$, sendo $T_{a}$ os geradores do grupo.
\end{exercise}

\begin{itemize}
    \item Utilizar o gauge $D_{\mu}\Phi = \qty(\partial_{\mu} - ig\dfrac{\sigma_{a}{2}}A_{\mu}^{a})\Phi$ e lembrar que $F_{\mu\nu} = \partial_{\mu}A_{\nu} - \partial_{\nu}A_{\mu}$
    \item No item (b), preciso confirmar se $F_{\mu\nu}' = UF_{\mu\nu}$ e $\Phi' = U\Phi$ pra tentar abrir as contas
\end{itemize}

\chapter{Teoria \texorpdfstring{$\lambda\phi^{4}$}{}}\label{quest: four}

\begin{exercise}{}
    \begin{multicols}{2}
        Na teoria $\lambda\phi^{4}$ calcule, em regularização dimensional, com todos os detalhes necessários (incluindo a expansão em torno de $\epsilon\to0$, com $D = 4-2\epsilon$), a integral do diagrama de Feynman da figura ao lado.
        
        \begin{center}
            \begin{tikzpicture}
                \begin{feynman}
                    \vertex (i1);
                    \vertex [below right=1.5cm and 2cm of i1, dot] (i2) {};
                    \vertex [above right=1.5cm and 2cm of i2] (i3);
                    
                    \vertex [below=6cm of i1] (i4);
                    \vertex [above right=1.5cm and 2cm of i4, dot] (i5) {};
                    \vertex [below right=1.5cm and 2cm of i5] (i6);

                    \diagram* {
                        (i1) -- [plain, momentum={[arrow shorten=0.7]$p_{3}$}] (i2) -- [plain, rmomentum={[arrow shorten=0.7]$p_{4}$}] (i3),
                        (i2) -- [plain, half left, looseness=1.5] (i5),
                        (i5) -- [plain, half left, looseness=1.5] (i2),
                        (i4) -- [plain, rmomentum={[arrow shorten=0.7]$p_{1}$}] (i5) -- [plain, momentum={[arrow shorten=0.7]$p_{2}$}] (i6),
                    };
                \end{feynman}
            \end{tikzpicture}
        \end{center}
        
        
    \end{multicols}
\end{exercise}

\begin{itemize}
    \item Aula 6 + Aula 11, o diagrama é do canal-t, bem determinado então
    \item O diagrama depende dos momentos externos apenas através da variável t
\end{itemize}

\chapter{Regras de Feynman da QCD}\label{quest: five}

\begin{exercise}{}
    Utilizando as regras de Feynman da QCD, mostre que o diagrama de 1 loop (auto-energia) do propagador do quark difere do correspondente ao elétron na QED por um fator multiplicativo $C_{F} = \displaystyle\sum_{a}T_{a}^{2}$. Obs: não é necessário calcular a integração de loop.
\end{exercise}

\begin{itemize}
    \item Os diagramas de 1 loop do propagador do quark (esquerda) de do elétron (direita) são
        \begin{center}
            \begin{tikzpicture}
                \begin{feynman}
                    \vertex (i1);
                    \vertex [right=1cm of i1, dot] (i2) {};
                    \vertex [right=2cm of i2, dot] (i3) {};
                    \vertex [right=1cm of i3] (i4);

                    \diagram* {
                        (i1) -- (i2) -- [fermion] (i3) -- (i4),
                        (i2) -- [gluon, half left, looseness=1.5] (i3),
                    };
                \end{feynman}
            \end{tikzpicture}
            \hspace{2cm}
            \begin{tikzpicture}
                \begin{feynman}
                    \vertex (i1);
                    \vertex [right=1cm of i1, dot] (i2) {};
                    \vertex [right=2cm of i2, dot] (i3) {};
                    \vertex [right=1cm of i3] (i4);

                    \diagram* {
                        (i1) -- (i2) -- (i3) -- (i4),
                        (i2) -- [boson, half left, looseness=1.5] (i3),
                    };
                \end{feynman}
            \end{tikzpicture}
        \end{center}
    \item O fato de não precisar calcular a integração de loop indica que existe algum truque pra obter a resposta.
\end{itemize}

\chapter{Espalhamento elástico elétron-múon}\label{quest: six}

\begin{exercise}{}
    Calcule explicitamente a contração $L^{\alpha\beta}W_{\alpha\beta}$ no espalhamento elástico elétron-múon. Obs: despreze a massa do elétron comparada a outras escalas de energia, assumidas muito maiores.
\end{exercise}



Com as definições fornecidas, podemos escrever o campo $\psi(x)$ sob a forma
    \begin{equation*}
        \psi(x) = \int \dfrac{1}{(2\pi)^3}\sum_{s'}\qty[
            b_{s'}(\vb{k}') u(\vb{k}',s') \dfrac{e^{-ik'x}}{\sqrt{2E_{\vb{k}'}}} +
            d^{\dagger}_{s'}(\vb{k}') v(\vb{k}',s') \dfrac{e^{ik'x}}{\sqrt{2E_{\vb{k}'}}}
        ]\dd[3]{k'}
    \end{equation*}
e portanto 
    \begin{equation*}
        \psi^{\dagger}(x) = \int \dfrac{1}{(2\pi)^3}\sum_{s'} \qty[
            b_{s'}^{\dagger}(\vb{k}') u^{\dagger}(\vb{k}',s') \dfrac{e^{ik'x}}{\sqrt{2E_{\vb{k}'}}} +
            d_{s'}(\vb{k}') v^{\dagger}(\vb{k}',s') \dfrac{e^{-ik'x}}{\sqrt{2E_{\vb{k}'}}}
        ]\dd[3]{k'}
    \end{equation*}

Tendo $\psi(x)$ e $\psi^{\dagger}(x)$, vamos aplicar $u^{\dagger}(\vb{k},s)$ pela esquerda do campo $\psi(x)$ (fazemos isso, pois $\psi(x)$ é um vetor devidos à presença dos espinores, então ao aplicar $u^{\dagger}(\vb{k},s)$, teremos um ``número'', o que faz com que as contas possam ser feitas de forma padrão) de modo que
    \begin{equation*}
        u^{\dagger}(\vb{k},s)\psi(x) = \int \dfrac{1}{(2\pi)^3} \sum_{s'} \qty[
            b_{s'}(\vb{k}') u^{\dagger}(\vb{k},s) u(\vb{k}',s') \dfrac{e^{-ik'x}}{\sqrt{2E_{\vb{k}'}}} +
            d_{s'}^{\dagger}(\vb{k}) u^{\dagger}(\vb{k},s) v(\vb{k}',s') \dfrac{e^{ik'x}}{\sqrt{2E_{\vb{k}'}}}
        ] \dd[3]{k'}
    \end{equation*}

Calculando uma transformada de Fourier inversa no espaço desta quantidade, temos
    \begin{align*}
        \int u^{\dagger}(\vb{k},s) \psi(x) e^{-i\vb{k}\cdot\vb{x}} \dd[3]{x} 
        &\eq \iint \dfrac{1}{(2\pi)^3} \sum_{s'}\Bigg[
            b_{s'}(\vb{k}') u^{\dagger}(\vb{k},s) u(\vb{k}',s') \dfrac{e^{-iE_{\vb{k}'}t + i(\vb{k}' - \vb{k})\cdot\vb{x}}}{\sqrt{2E_{\vb{k}'}}} + \\
        &\noeq + 
            d_{s'}^{\dagger}(\vb{k}') u^{\dagger}(\vb{k},s) v(\vb{k}',s') \dfrac{e^{iE_{\vb{k}'}t - i(\vb{k}'+\vb{k})\cdot\vb{x}}}{\sqrt{2E_{\vb{k}'}}}
        \Bigg] \dd[3]{k'} \dd[3]{x} \\
        &\eq \iint \dfrac{e^{-iE_{\vb{k}'}t}}{(2\pi)^3\sqrt{2E_{\vb{k}'}}} \sum_{s'} \Big[
            b_{s'}(\vb{k}') u^{\dagger}(\vb{k},s) u(\vb{k}',s') e^{i(\vb{k}' - \vb{k})\cdot\vb{x}} + \\
        &\noeq +  
            d_{s'}^{\dagger}(\vb{k}') u^{\dagger}(\vb{k},s) v(\vb{k}',s') e^{2iE_{\vb{k}'}t} e^{-i(\vb{k}' + \vb{k})\cdot\vb{x}}
        \Big] \dd[3]{k'} \dd[3]{x}
    \end{align*}

As integrações em $\dd[3]{x}$ vão nos dar distribuições delta de Dirac, de modo que 
    \begin{align*}
        \int u^{\dagger}(\vb{k},s) \psi(x) e^{-i\vb{k}\cdot\vb{x}} \dd[3]{x} 
        =&\; \int \dfrac{e^{-iE_{\vb{k}'}t}}{\sqrt{2E_{\vb{k}'}}} \sum_{s'} \Big[
            b_{s'}(\vb{k}') u^{\dagger}(\vb{k},x) u(\vb{k}',s') \delta^3(\vb{k}' - \vb{k}) + \\
        &+
            d_{s'}^{\dagger}(\vb{k}') u^{\dagger}(\vb{k},s) v(\vb{k}',s') e^{2iE_{\vb{k}'}t} \delta^3(\vb{k}' + \vb{k})
        \Big]\dd[3]{k'}
    \end{align*}

Como $E_{\vb{k}} = E_{-\vb{k}}$ por construção, temos pela integração em $\dd[3]{k'}$:
    \begin{align*}
        \int u^{\dagger}(\vb{k},s) \psi(x) e^{-i\vb{k}\cdot\vb{x}} \dd[3]{x} 
        = \dfrac{e^{-iE_{\vb{k}}t}}{\sqrt{2E_{\vb{k}}}} \sum_{s'} \Big[ 
            b_{s'}(\vb{k}) u^{\dagger}(\vb{k},s) u(\vb{k},s') + 
            d_{s'}^{\dagger}(-\vb{k}) u^{\dagger}(\vb{k},s) v(-\vb{k},s') e^{2iE_{\vb{k}}t}
        \Big]
    \end{align*}

Considerando a forma explicita dos espinores, o produto entre $u^{\dagger}(\vb{k},s)$ e $u(\vb{k},s')$ vai resultar em
    \begin{align*}
        u^{\dagger}(\vb{k},s) u(\vb{k},s') &\eq \dfrac{1}{E_{\vb{k}}+m} \qty[
            (E_{\vb{k}} + m)^2 \chi^{\dagger}_{s}\chi_{s'} + \chi^{\dagger}_{s}(\boldsymbol{\sigma}\cdot\vb{p})^{\dagger}(\boldsymbol{\sigma}\cdot\vb{p})\chi_{s'}
        ] \\
        &\eq \dfrac{1}{E_{\vb{k}}+m}\qty[
            (E_{\vb{k}} + m)^2 \delta_{ss'} + 
            (E_{\vb{k}} + m)(E_{\vb{k}} - m) \delta_{ss'}
        ] \\
        &\eq (E_{\vb{k}} + m)\delta_{ss'} + (E_{\vb{k}} - m)\delta_{ss'} \\
        &\eq 2E_{\vb{k}}\delta_{ss'}
    \end{align*}

Usando agora a forma explicita para $v(-\vb{k},s')$, teremos
    \begin{align*}
        u^{\dagger}(\vb{k},s) v(-\vb{k},s') &\eq \dfrac{1}{E_{\vb{k}} + m}\qty{
            (E_{\vb{k}}+m)\chi^{\dagger}_{s}[\boldsymbol{\sigma}\cdot(-\vb{k})]\chi_{s'} +
            (E_{-\vb{k}}+m)(\boldsymbol{\sigma}\cdot\vb{k})\chi^{\dagger}_{s}\chi_{s'}
        } \\
        &\eq -(\boldsymbol{\sigma}\cdot\vb{k})\chi^{\dagger}_{s}\chi_{s'} + 
        (\boldsymbol{\sigma}\cdot\vb{k})\chi^{\dagger}_{s}\chi_{s'} \\
        &\eq 0
    \end{align*}

Portanto, a integral que estávamos calculando resulta em
    \begin{equation*}
        \int u^{\dagger}(\vb{k},s) \psi(x) e^{-i\vb{k}\cdot\vb{x}} \dd[3]{x} = \dfrac{e^{-iE_{\vb{k}}t}}{\sqrt{2E_{\vb{k}}}} \sum_{s'} b_{s'}(\vb{k}) 2E_{\vb{k}} \delta_{ss'} = \sqrt{2E_{\vb{k}}} e^{-iE_{\vb{k}}t} b_{s}(\vb{k})
    \end{equation*}

Isolando $b_{s}(\vb{k})$, obtemos
    \begin{align*}
        b_{s}(\vb{k}) &\eq \dfrac{1}{\sqrt{2E_{\vb{k}}}} e^{iE_{\vb{k}}t} \int u^{\dagger}(\vb{k},s) \psi(x) e^{-i\vb{k}\cdot\vb{x}} \dd[3]{x} \\
        &\eq \int u^{\dagger}(\vb{k},s) \dfrac{e^{ikx}}{\sqrt{2E_{\vb{k}}}} \psi(x) \dd[3]{x}
    \end{align*}

Podemos ainda manipular o argumento da integral adicionando uma matriz identidade entre o espinor $u^{\dagger}(\vb{k},s)$ e o campo $\psi(x)$, onde essa matriz identidade pode ser escrita por $I = (\gamma^{0})^2$, de modo que 
    \begin{equation*}
        u^{\dagger}(\vb{k},s) \psi(x) = u^{\dagger}(\vb{k},s)\gamma^{0}\gamma^{0} \psi(x) = \bar{u}(\vb{k},s) \gamma^{0} \psi(x)
    \end{equation*}

Ou seja, o argumento da integral fica, com base na definição de $U_{k}^{s}(x)$ dada no enunciado:
    \begin{equation*}
        \bar{u}(\vb{k},s) \dfrac{e^{ikx}}{\sqrt{2E_{\vb{k}}}} \gamma^{0} \psi(x) = \bar{U}_{k}^{s}(x) \gamma^{0} \psi(x)
    \end{equation*}

Concluindo que
    \begin{answer}\label{eq: annihilation operator for particle fermion}
        b_{s}(\vb{k}) = \int \bar{U}_{k}^{s}(x) \gamma^{0} \psi(x) \dd[3]{x}
    \end{answer}

Aplicando agora $v(\vb{k},s)$ pela direita do campo $\psi^{\dagger}(x)$, pelo mesmo motivo do caso anterior, temos
    \begin{equation*}
        \psi^{\dagger}(x) v(\vb{k},s) = \int \dfrac{1}{(2\pi)^3}\sum_{s'} \qty[
            b_{s'}^{\dagger}(\vb{k}') u^{\dagger}(\vb{k}',s') v(\vb{k},s) \dfrac{e^{ik'x}}{\sqrt{2E_{\vb{k}'}}} +
            d_{s'}(\vb{k}') v^{\dagger}(\vb{k}',s') v(\vb{k},s) \dfrac{e^{-ik'x}}{\sqrt{2E_{\vb{k}'}}}
        ]\dd[3]{k'}
    \end{equation*}

Fazendo também uma transformada de Fourier inversa no espaço, temos
    \begin{align*}
        \int \psi^{\dagger}(x) v(\vb{k},s) e^{-i\vb{k}\cdot\vb{x}} \dd[3]{x} &\eq 
        \iint \dfrac{1}{(2\pi)^3} \sum_{s'} \Bigg[
            b_{s'}^{\dagger}(\vb{k}') u^{\dagger}(\vb{k}',s') v(\vb{k},s) \dfrac{e^{iE_{\vb{k}}t - i(\vb{k}'+\vb{k})\cdot\vb{x}}}{\sqrt{2E_{\vb{k}'}}} + \\
        &\noeq+ d_{s'}(\vb{k}') v^{\dagger}(\vb{k}',s') v(\vb{k},s) \dfrac{e^{-iE_{\vb{k}'}t + i(\vb{k}'-\vb{k})\cdot\vb{x}}}{\sqrt{2E_{\vb{k}'}}}
        \Bigg] \dd[3]{k'} \dd[3]{x} \\
        &\eq \iint \dfrac{e^{-iE_{\vb{k}}'}t}{(2\pi)^3\sqrt{2E_{\vb{k}'}}}  \sum_{s'} \Big[
            b_{s'}^{\dagger}(\vb{k}') u^{\dagger}(\vb{k}',s') v(\vb{k},s) e^{2iE_{\vb{k}'}t} e^{-i(\vb{k}' + \vb{k})\cdot \vb{x}} + \\
        &\noeq+
            d_{s'}(\vb{k}') v^{\dagger}(\vb{k}',s') v(\vb{k},s) e^{i(\vb{k}' - \vb{k})\cdot\vb{x}}
        \Big] \dd[3]{k'} \dd[3]{x}
    \end{align*}

Integrando em $\dd[3]{x}$ teremos o surgimento das deltas de Dirac tal que
    \begin{align*}
        \int \psi^{\dagger}(x) v(\vb{k},s) e^{-i\vb{k}\cdot\vb{x}} \dd[3]{x} &\eq \int \dfrac{e^{-iE_{\vb{k}}'}t}{(2\pi)^3\sqrt{2E_{\vb{k}'}}}  \sum_{s'} \Big[
            b_{s'}^{\dagger}(\vb{k}') u^{\dagger}(\vb{k}',s') v(\vb{k},s) e^{2iE_{\vb{k}'}t} \delta^3(\vb{k}' + \vb{k}) \\
        &\noeq+
            d_{s'}(\vb{k}') v^{\dagger}(\vb{k}',s') v(\vb{k},s) \delta^3(\vb{k}' - \vb{k})
        \Big] \dd[3]{k'} \dd[3]{x}
    \end{align*}

E levando novamente em consideração que $E_{\vb{k}} = E_{-\vb{k}}$, ao realizarmos a integração em $\dd[3]{k'}$, temos
    \begin{equation*}
        \int \psi^{\dagger}(x) v(\vb{k},s) e^{-i\vb{k}\cdot\vb{x}} \dd[3]{x} = \dfrac{e^{-iE_{\vb{k}}t}}{\sqrt{2E_{\vb{k}}}} \sum_{s'} \qty[
            b_{s'}^{\dagger}(-\vb{k}) u^{\dagger}(-\vb{k},s') v(\vb{k},s) e^{2iE_{\vb{k}}t} +
            d_{s'}(\vb{k}) v^{\dagger}(\vb{k},s') v(\vb{k},s)
        ]
    \end{equation*}

Dado que encontramos $u^{\dagger}(\vb{k},s)v(-\vb{k},s')$, podemos simplesmente trocar $\vb{k} \leftrightarrow -\vb{k}$ e $s \leftrightarrow s'$, de modo que o resultado se mantém, ou seja
    \begin{equation*}
        u^{\dagger}(-\vb{k},s') v(\vb{k},s) = 0
    \end{equation*}

Já para o segundo produto de espinores, basta considerar a forma explícita, tal que
    \begin{align*}
        v^{\dagger}(\vb{k},s') v(\vb{k},s) &\eq \dfrac{1}{E_{\vb{k}}+m}\qty[
            (\boldsymbol{\sigma}\cdot\vb{k})^{\dagger}\chi^{\dagger}_{s'}\chi_{s} (\boldsymbol{\sigma}\cdot\vb{k}) + 
            (E_{\vb{k}} + m)^2 \chi^{\dagger}_{s'}\chi_{s}
        ] \\
        &\eq \dfrac{1}{E_{\vb{k}} + m} \qty[
            (E_{\vb{k}} + m)(E_{\vb{k}} - m) \delta_{s's} + (E_{\vb{k}} + m)^2 \delta_{s's}
        ] \\
        &\eq (E_{\vb{k}} - m)\delta_{s's} + (E_{\vb{k}} + m)\delta_{s's} \\
        &\eq 2E_{\vb{k}}\delta_{s's}
    \end{align*}

Portanto
    \begin{equation*}
        \int \psi^{\dagger}(x) v(\vb{k},s) e^{-i\vb{k}\cdot\vb{x}} \dd[3]{x} = \dfrac{e^{-iE_{\vb{k}}t}}{\sqrt{2E_{\vb{k}}}} \sum_{s'} 
            d_{s'}(\vb{k}) 2 E_{\vb{k}}\delta_{s's} = \sqrt{2E_{\vb{k}}} e^{-iE_{\vb{k}}t} d_{s}(\vb{k})
    \end{equation*}

Isolando $d_{s}(\vb{k})$, temos
    \begin{align*}
        d_{s}(\vb{k}) &\eq \dfrac{e^{iE_{\vb{k}}t}}{\sqrt{2E_{\vb{k}}}} \int \psi^{\dagger}(x) v(\vb{k},s) e^{-i\vb{k}\cdot\vb{x}} \dd[3]{x} \\
        &\eq \int \psi^{\dagger}(x) v(\vb{k},s) \dfrac{e^{ikx}}{\sqrt{2E_{\vb{k}}}} \dd[3]{x}
    \end{align*}

Inserindo a identidade $I = (\gamma^{0})^2$ entre $\psi^{\dagger}(x)$ e $v(\vb{k},s)$, podemos reescrever o integrando por
    \begin{equation*}
        \psi^{\dagger}(x) v(\vb{k},s) \dfrac{e^{ikx}}{\sqrt{2E_{\vb{k}}}} = 
        \psi^{\dagger}(x) \gamma^{0} \gamma^{0} v(\vb{k},s) \dfrac{e^{ikx}}{\sqrt{2E_{\vb{k}}}} = \bar{\psi}(x) \gamma^{0} V_{k}^{s}(x)
    \end{equation*}

Concluindo que
    \begin{answer}\label{eq: annihilation operator for antiparticle fermion}
        d_{s}(\vb{k}) = \int \bar{\psi}(x) \gamma^{0} V_{k}^{s}(x) \dd[3]{x}
    \end{answer}

\chapter{Amplitude DVCS}\label{quest: seven}

\begin{exercise}{}
    Vimos que a amplitude DVCS (deeply-virtual Compton scattering) é dada por
        \begin{equation*}
            T_{\mu\nu} = i\int e^{iqz}\bra{p}T\{J_{\mu}(z)J_{\nu}(0)\}\ket{p}\dd[4]{z}
        \end{equation*}
    \begin{enumerate}[(a)]
        \item Partindo da expressão acima, mostre que $\text{Im}[T_{\alpha\beta}] = \pi W_{\alpha\beta}$, ou seja, que a parte imaginária da amplitude DVCS é proporcional ao tensor hadrônico do espalhamento inélástico profundo.
        \item Usando um conjunto completo de estados intermediários entre as correntes eletromagnéticas, argumente por quê
            \begin{equation*}
                \int e^{iqz} \bra{p}J_{\mu}(0)J_{\nu}(z)\ket{p}
            \end{equation*}
        é igual a zero. \textbf{Dica:} obtenha uma função delta de Dirac e conservação do quadri-momento total.
    \end{enumerate}
\end{exercise}



\noindent \textbf{(a)} Podemos representar as componentes de $\boldsymbol{\pi}$ por
    \begin{equation*}
        \pi_{1} = \dfrac{\pi^{+} + \pi^{-}}{\sqrt{2}} \qquad \& \qquad 
        \pi_{2} = \dfrac{i(\pi^{-} - \pi^{+})}{\sqrt{2}} \qquad \& \qquad 
        \pi_{3} = \pi_{0}
    \end{equation*}

Dessa forma, o campo $\Phi = \boldsymbol{\tau}\cdot\boldsymbol{\sigma}$ pode ser representado matricialmente
    \begin{align*}
        \Phi &\eq \tau_{1}\pi_{1} + \tau_{2}\pi_{2} + \tau_{3}\pi_{3} \\
        &\eq 
        \tau_{1} \qty(\dfrac{\pi^{+} + \pi^{-}}{\sqrt{2}}) + 
        \tau_{2}\qty[\dfrac{i(\pi^{-} - \pi^{+})}{\sqrt{2}}] + 
        \tau_{3} \pi_{0} \\
        &\eq 
        \begin{bmatrix}
            0 & 1 \\
            1 & 0
        \end{bmatrix} \qty(\dfrac{\pi^{+} + \pi^{-}}{\sqrt{2}}) +
        \begin{bmatrix}
            0 & -i \\
            i & 0
        \end{bmatrix} \qty[\dfrac{i(\pi^{-} - \pi^{+})}{\sqrt{2}}] + 
        \begin{bmatrix}
            1 & 0 \\
            0 & -1
        \end{bmatrix} \pi_{0} \\
        &\eq 
            \dfrac{1}{\sqrt{2}}
            \begin{bmatrix}
                0 & \pi^{+} + \pi^{-} \\
                \pi^{+} + \pi^{-} & 0
            \end{bmatrix} +
            \dfrac{1}{\sqrt{2}}
            \begin{bmatrix}
                0 & \pi^{-} - \pi^{+} \\
                -\pi^{-} + \pi^{+} & 0
            \end{bmatrix} +
            \begin{bmatrix}
                \pi_{0} & 0 \\
                0 & -\pi_{0}
            \end{bmatrix} \\
        &\eq 
        \begin{bmatrix}
            \pi_{0} & \sqrt{2} \pi^{-} \\
            \sqrt{2} \pi^{+} & -\pi_{0}
        \end{bmatrix} = 
        \begin{bmatrix}
        		\pi_{3} & \pi_{1} + i\pi_{2} \\
        		\pi_{1} - i\pi_{2} & \pi_{3}
        \end{bmatrix}
    \end{align*}

Portanto
    \begin{equation*}
        \Phi^{\dagger} = 
        \begin{bmatrix}
            \pi_{0}^{\ast} & \sqrt{2}(\pi^{+})^{\ast} \\
            \sqrt{2}(\pi^{-})^{\ast} & -\pi_{0}^{\ast}
        \end{bmatrix}
    \end{equation*}

Pela forma definida de $\pi^{\pm}$, é fácil ver que $(\pi^{-})^{\ast} = \pi^{+}$ e $(\pi^{+})^{\ast} = \pi^{-}$, logo
    \begin{equation*}
        \Phi^{\dagger} = 
        \begin{bmatrix}
            \pi_{0}^{\ast} & \sqrt{2}\pi^{-} \\
            \sqrt{2}\pi^{+} & -\pi_{0}^{\ast}
        \end{bmatrix}
    \end{equation*}

Segue que
    \begin{align*}
        \partial_{\mu}\Phi^{\dagger} \partial^{\mu}\Phi &\eq 
        \begin{bmatrix}
            \partial_{\mu}\pi_{0}^{\ast} & \sqrt{2}\partial_{\mu}\pi^{-} \\
            \sqrt{2}\partial_{\mu}\pi^{+} & -\partial_{\mu}\pi_{0}^{\ast}
        \end{bmatrix}
        \begin{bmatrix}
            \partial^{\mu}\pi_{0} & \sqrt{2}\partial^{\mu}\pi^{-} \\
            \sqrt{2}\partial^{\mu}\pi^{+} & -\partial^{\mu}\pi_{0}
        \end{bmatrix} \\
        &\eq
        \begin{bmatrix}
            \partial_{\mu}\pi_{0}^{\ast}\partial^{\mu}\pi_{0} + 2\partial_{\mu}\pi^{-}\partial^{\mu}\pi^{+} & \sqrt{2}\partial_{\mu}\pi_{0}^{\ast}\partial^{\mu}\pi^{-} - \sqrt{2}\partial_{\mu}\pi^{-}\partial^{\mu}\pi_{0} \\
            \sqrt{2}\partial_{\mu}\pi^{+}\partial^{\mu}\pi_{0} - \sqrt{2}\partial_{\mu}\pi_{0}^{\ast}\partial^{\mu}\pi^{2} & 2\partial_{\mu}\pi^{+}\partial^{\mu}\pi^{-} + \partial_{\mu}\pi_{0}^{\ast}\partial^{\mu}\pi_{0}
        \end{bmatrix}
    \end{align*}

Calculando o traço desta matriz:
    \begin{align*}
        \expval{\partial_{\mu}\Phi^{\dagger}\partial^{\mu}\Phi} &\eq \partial_{\mu}\pi_{0}^{\ast}\partial^{\mu}\pi_{0} + 
        2\partial_{\mu}\pi^{-}\partial^{\mu}\pi^{+} + 
        2\partial_{\mu}\pi^{+}\partial^{\mu}\pi^{-} + 
        \partial_{\mu}\pi_{0}^{\ast}\partial^{\mu}\pi_{0} \\
        &\eq 2\partial_{\mu}\pi_{0}^{\ast} \partial^{\mu}\pi_{0} + 
        2\partial_{\mu}\pi^{-}\partial^{\mu}\pi^{+} + 
        2\partial_{\mu}\pi^{+}\partial^{\mu}\pi^{-}
    \end{align*}

Fazendo o mesmo procedimento para $\Phi^{\dagger}\Phi$, temos
    \begin{align*}
        \Phi^{\dagger}\Phi &\eq 
        \begin{bmatrix}
            \pi_{0}^{\ast} & \sqrt{2}\pi^{-} \\
            \sqrt{2}\pi^{+} & -\pi_{0}^{\ast}
        \end{bmatrix}
        \begin{bmatrix}
            \pi_{0} & \sqrt{2} \pi^{-} \\
            \sqrt{2} \pi^{+} & -\pi_{0}
        \end{bmatrix} \\
        &\eq 
        \begin{bmatrix}
            \pi_{0}^{\ast} \pi_{0} + 2 \pi^{-} \pi^{+} & 
            \sqrt{2} \pi_{0}^{\ast} \pi^{-} - \sqrt{2} \pi^{-}\pi_{0} \\
            \sqrt{2} \pi^{+} \pi_{0} - \sqrt{2}\pi_{0}^{\ast} \pi^{+} & 
            2 \pi^{+} \pi^{-} + \pi_{0}^{\ast} \pi_{0}
        \end{bmatrix}
    \end{align*}

Então
    \begin{align*}
        \expval{\Phi^{\dagger}\Phi} &\eq \pi_{0}^{\ast}\pi_{0} + 2\pi^{-}\pi^{+} + 2\pi^{+}\pi^{-} + \pi_{0}^{\ast}\pi_{0} \\
        &\eq 2\pi_{0}^{\ast} \pi_{0} + 2\pi^{-}\pi^{+} + 2\pi^{+}\pi^{-}
    \end{align*}

A lagrangiana fica então
    \begin{align*}
        \mathcal{L} &\eq \dfrac{1}{4}\qty(
            2\partial_{\mu}\pi_{0}^{\ast} \partial^{\mu}\pi_{0} + 
            2\partial_{\mu}\pi^{-}\partial^{\mu}\pi^{+} + 
            2\partial_{\mu}\pi^{+}\partial^{\mu}\pi^{-}
        ) - \dfrac{m^2}{4}\qty(
            2\pi_{0}^{\ast} \pi_{0} + 2\pi^{-}\pi^{+} + 2\pi^{+}\pi^{-}
        ) \\
        &\eq \qty[
            \dfrac{1}{2}\partial_{\mu}\pi_{0}^{\ast} \partial^{\mu}\pi_{0} - 
            \dfrac{m^2}{2} \pi_{0}^{\ast} \pi_{0}
        ] + \qty[
            \qty(
                \dfrac{1}{2}\partial_{\mu}\pi^{-} \partial^{\mu}\pi^{+} +
                \dfrac{1}{2}\partial_{\mu}\pi^{+} \partial^{\mu}\pi^{-}
            ) -
            \dfrac{m^2}{2} 
            \qty( 
                \pi^{-}\pi^{+} +
                \pi^{+}\pi^{-}
            )
        ]
    \end{align*}

Sabendo que $\pi_{0}(x)$ é um campo real, temos que $\pi_{0}^{\ast} = \pi_{0}$, o que nos permite escrever o primeiro $[\cdots]$ da lagrangiana sob a forma de uma lagrangiana livre de um campo escalar neutro para $\pi_{0}$:
    \begin{answer}\label{eq: free lagrangian of a neutral scalar field}
        \mathcal{L}_{\pi_{0}} = \dfrac{1}{2}\partial_{\mu}\pi_{0}\partial^{\mu}\pi_{0} - \dfrac{1}{2}m^2\pi_{0}^2
    \end{answer}

No segundo $[\cdots]$, identificamos que $\partial_{\mu}\pi^{-}\partial^{\mu}\pi^{+} = \partial_{\mu}\pi^{+}\partial^{\mu}\pi^{-}$ e $\pi^{-}\pi^{+} = \pi^{+}\pi^{-}$, nos permitindo escrever uma lagrangiana de campo escalar carregado
    \begin{answer}\label{eq: free lagrangian of a charged scalar field}
        \mathcal{L}_{\pi^{\pm}} = \partial_{\mu}\pi^{+}\partial^{\mu}\pi^{-} - m^2 \pi^{+}\pi^{-}
    \end{answer}
	

\noindent \textbf{(b)} A lagrangiana original pode ser reescrita após calcularmos os traços. O primeiro traço da lagrangiana fica, lembrando que $\Phi^{\dagger} = \Phi = \boldsymbol{\tau}\cdot\boldsymbol{\pi}$:
    \begin{align*}
        \expval{\partial_{\mu}\Phi^{\dagger} \partial^{\mu}\Phi} &\eq
        \expval{\partial_{\mu}(\boldsymbol{\tau}\cdot\boldsymbol{\pi})\partial^{\mu}(\boldsymbol{\tau}\cdot\boldsymbol{\pi})} = 
        \expval{\sum_{a,b}\partial_{\mu}(\tau_{a}\pi_{a})\partial^{\mu}(\tau_{b}\pi_{b})} \\
        &\eq 
        \expval{\sum_{a,b}\tau_{a}\tau_{b}(\partial_{\mu}\pi_{a})(\partial^{\mu}\pi_{b})} =
        \sum_{a,b} \expval{\tau_{a}\tau_{b}} (\partial_{\mu}\pi_{a})(\partial^{\mu}\pi_{b})
    \end{align*}
As matrizes de Pauli satisfazem a propriedade $\expval{\tau_{a}\tau_{b}} = 2\delta_{ab}$, portanto
    \begin{equation*}
        \expval{\partial_{\mu}\Phi^{\dagger}\partial^{\mu}\Phi} = 
        2\sum_{a,b} \delta_{ab} (\partial_{\mu}\pi_{a})(\partial^{\mu}\pi_{b}) =
        2(\partial_{\mu}\pi_{a})(\partial^{\mu}\pi_{a}) = 2(\partial_{\mu}\boldsymbol{\pi})\cdot(\partial^{\mu}\boldsymbol{\pi})
    \end{equation*}


O segundo traço da lagrangiana fica
    \begin{align*}
        \expval{\Phi^{\dagger}\Phi} &\eq 
        \expval{(\boldsymbol{\tau}\cdot\boldsymbol{\pi})(\boldsymbol{\tau}\cdot\boldsymbol{\pi})} = 
        \expval{\sum_{a,b}\tau_{a}\pi_{a}\tau_{b}\pi_{b}} =
        \sum_{a,b}\pi_{a}\pi_{b}\expval{\tau_{a}\tau_{b}} \\
        &\eq
        2\sum_{a,b}\pi_{a}\pi_{b}\delta_{ab} = 
        2\pi_{a}\pi_{a} =
        2(\boldsymbol{\pi}\cdot\boldsymbol{\pi})
    \end{align*}

A lagrangiana na representação cartesiana pode ser escrita então por
    \begin{equation*}
        \mathcal{L} = \dfrac{1}{2}(\partial_{\mu}\boldsymbol{\pi})\cdot(\partial^{\mu}\boldsymbol{\pi}) - 
        \dfrac{m^2}{2}(\boldsymbol{\pi}\cdot\boldsymbol{\pi})
    \end{equation*}

Considerando uma transformação $U(\boldsymbol{\theta}) = \exp(-i\boldsymbol{\tau}\cdot\boldsymbol{\theta})$, com $\boldsymbol{\theta}$ um vetor de componentes infinitesimais no espaço de isospin, podemos aproximar
    \begin{equation*}
        U(\boldsymbol{\theta}) \approx \boldone_{2\times2} - i\boldsymbol{\tau}\cdot\boldsymbol{\theta}
    \end{equation*}

Sendo $\Phi$ uma matriz $2\times2$ hermitiana e de traço nulo, ele é um elemento da álgebra de Lie $\mathfrak{su}(2)$, cujos elementos se transformam como $A' = UAU^{-1}$, para uma transformação unitária $U$ em $\text{SU}(2)$, então como $U^{\dagger}(\boldsymbol{\theta}) = U^{-1}(\boldsymbol{\theta})$ e é um elemento de $\text{SU}(2)$, uma transformação equivalente à do enunciado que mantém o campo invariante é $\Phi' = U(\boldsymbol{\theta}) \Phi U^{\dagger}(\boldsymbol{\theta})$. Usar esta transformação ao invés de apenas $\Phi' = U(\boldsymbol{\theta})\Phi$ é possível, pois ambas vão gerar a mesma variação dos campos $\pi_{j}$ (basta utilizar a forma $\Phi = \boldsymbol{\tau}\cdot\boldsymbol{\pi}$ em $\delta\Phi$ e identificar a parte que representa a variação nos campos $\pi_{j}$) , a única diferença seria o fato de $U(\boldsymbol{\theta})\Phi$ não manter a hermiticidade e o traço nulo ao determinar $\delta\Phi$, logo, utilizo a transformação padrão de elementos de uma álgebra de Lie $\mathfrak{su}(2)$ apenas por buscar manter essas duas propriedades em $\delta\Phi$. Portanto
    \begin{align*}
        \Phi' = U(\boldsymbol{\theta})\Phi U^{\dagger}(\boldsymbol{\theta}) &\eq  
        \qty[\Phi - i(\boldsymbol{\tau}\cdot\boldsymbol{\theta})\Phi]
        \qty[\boldone_{2\times2} + i(\boldsymbol{\tau}\cdot\boldsymbol{\theta})] \\
        &\eq
        \Phi + i\Phi(\boldsymbol{\tau}\cdot\boldsymbol{\theta}) - 
        i(\boldsymbol{\tau}\cdot\boldsymbol{\theta})\Phi + 
        \mathcal{O}(\boldsymbol{\theta}^2) \\
        &\eq \Phi + 
        i[\Phi,(\boldsymbol{\tau}\cdot\boldsymbol{\theta})] + 
        \mathcal{O}(\boldsymbol{\theta}^2)
    \end{align*}

Ignorando os termos quadráticos em $\boldsymbol{\theta}$ por ele ter componentes infinitesimais, temos que a variação do campo nos dá
    \begin{equation*}
        \delta\Phi = i\sum_{a}[\Phi, \tau_{a}]\theta_{a} = i\sum_{a,b} [\tau_{b},\tau_{a}]\theta_{a}\pi_{b}
    \end{equation*}

As matrizes de Pauli por serem geradores de uma álgebra de Lie satisfazem a relação de comutação
    \begin{equation*}
        [\tau_{a},\tau_{b}] = \sum_{c}2i\epsilon_{abc}\tau_{c}
    \end{equation*}
ou seja
    \begin{equation*}
        \delta\Phi = -2\sum_{a,b,c} \epsilon_{bac}\tau_{c}\theta_{a}\pi_{b} = 2\sum_{a,b,c}\epsilon_{abc}\tau_{c}\theta_{a}\pi_{b}
    \end{equation*}

Podemos considerar que 
    \begin{equation*}
        \delta\Phi = \delta\qty(\sum_{c}\tau_{c}\pi_{c}) = \sum_{c}\tau_{c} \delta\pi_{c} = 2\sum_{a,b,c}\epsilon_{abc}\tau_{c}\theta_{a}\pi_{b}
    \end{equation*}

Então a variação do campo fica
    \begin{equation*}
        \delta \pi_{c} = 2\sum_{a,b} \epsilon_{abc}\theta_{a}\pi_{b}
    \end{equation*}

Para obter a corrente de Noether, consideramos a simetria global em $\theta_{k}$, de tal forma que para $k=1,2,3$, temos
    \begin{equation*}
        J_{k}^{\mu} = \sum_{c}\fdv{\mathcal{L}}{(\partial_{\mu}\pi_{c})} \fdv{\pi_{c}}{\theta_{k}}
    \end{equation*}

O segundo termo é facilmente obtido, tal que
    \begin{equation*}
        \fdv{\pi_{c}}{\theta_{k}} = 2\sum_{a,b} \epsilon_{abc}\delta_{ak}\pi_{b} = 2\sum_{b} \epsilon_{kbc} \pi_{b}
    \end{equation*}

Já o primeiro
    \begin{align*}
        \fdv{\mathcal{L}}{(\partial_{\mu}\pi_{c})} &\eq \dfrac{1}{2}\fdv{}{(\partial_{\mu}\pi_{c})}\qty[
            \sum_{d}\partial_{\nu}\pi_{d}\partial^{\nu}\pi_{d}
        ] = \dfrac{1}{2}\sum_{d}\qty[
            \fdv{(\partial_{\nu}\pi_{d})}{(\partial_{\mu}\pi_{c})} \partial^{\nu}\pi_{d} +
            \partial_{\nu}\pi_{d} g^{\nu\beta} \fdv{(\partial_{\beta}\pi_{d})}{(\partial_{\mu}\pi_{c})}
        ] \\
        &\eq \dfrac{1}{2}\sum_{d}\qty[
            \delta_{\nu}^{\mu}\delta_{cd} \partial^{\nu}\pi_{d} + 
            \partial_{\nu}\pi_{d} g^{\nu\beta} \partial_{\beta}^{\mu}\delta_{cd}
        ] = 
        \dfrac{1}{2}\sum_{d}\qty[
            \delta_{cd}\partial^{\mu}\pi_{d} + 
            \partial^{\mu}\pi_{d}\delta_{cd}
        ] \\
        &\eq \dfrac{1}{2}\qty[\partial^{\mu}\pi_{c} + \partial^{\mu}\pi_{c}] = \partial^{\mu}\pi_{c}
    \end{align*}

Logo
    \begin{equation*}
        J_{k}^{\mu} = 2\sum_{c}\partial^{\mu}\pi_{c}\sum_{b}\epsilon_{kbc}\pi_{b} = 2\sum_{b,c} \epsilon_{kbc}\pi_{b}\partial^{\mu}\pi_{c}
    \end{equation*}

Juntando todas as componentes $k$, concluímos que a corrente de Noether é
    \begin{answer}\label{eq: Noether current for a simple sigma model}
        J^{\mu} = 2\boldsymbol{\pi}\times\partial^{\mu}\boldsymbol{\pi}
    \end{answer}

\chapter{Cálculo de diagramas de Feynman}\label{quest: eight}

\begin{exercise}{}
    Utilizando as regras de Feynman da QCD (aula 13), calcule os diagramas de Feynman abaixo, que contribuem para o processo $e^{+}(k_{1})e^{-}(k_{2}) \to \bar{q}(p_{1})q(p_{2})$. No segundo diagrama, o glúon emitido possui momento muito baixo (\textit{soft glúon}). Mostre que os dois diagramas são infinitos, explique a origem desses infinitos, e como lidar com eles para obter um resultado físico consistente.

    \noindent\begin{tikzpicture}
        \begin{feynman}
            \vertex (i1) {$e^{+}$};
            \vertex [below right=1.5cm and 2cm of i1] (i2);
            \vertex [below left=1.25cm and 1.75cm of i2] (i3) {$e^{-}$};

            \vertex [right=6cm of i1] (i4) {$\bar{q}$};
            \vertex [below left=1.5cm and 2cm of i4] (i5);
            \vertex [below right=1.25cm and 1.75cm of i5] (i6) {$q$};

            \vertex [below left=0.375cm and 0.5cm of i4] (g1);
            \vertex [below=2.25cm of g1] (g2);

            \diagram*{
                (i3) -- [fermion] (i2) -- [fermion] (i1),
                (i2) -- [boson] (i5),
                (i4) -- [fermion] (i5) -- [fermion] (i6),
                (g1) -- [gluon, half left, looseness=1] (g2)
            };

        \end{feynman}
    \end{tikzpicture}
    \hfill
    \begin{tikzpicture}
        \begin{feynman}
            \vertex (i1) {$e^{+}$};
            \vertex [below right=1.5cm and 2cm of i1] (i2);
            \vertex [below left=1.25cm and 1.75cm of i2] (i3) {$e^{-}$};

            \vertex [right=6cm of i1] (i4) {$\bar{q}$};
            \vertex [below left=1.5cm and 2cm of i4] (i5);
            \vertex [below right=1.25cm and 1.75cm of i5] (i6) {$q$};

            \vertex [below left=0.375cm and 0.5cm of i4] (g1);
            \vertex [below right=0.5cm and 0.5cm of g1] (g2);

            \diagram*{
                (i3) -- [fermion] (i2) -- [fermion] (i1),
                (i2) -- [boson] (i5),
                (i4) -- [fermion] (i5) -- [fermion] (i6),
                (g1) -- [gluon] (g2)
            };

        \end{feynman}
    \end{tikzpicture}
\end{exercise}

% \makefinal

\end{document}
