\RequirePackage{pdfmanagement-testphase} % tranparent package doesn't work without this
\DeclareDocumentMetadata{} % and this
\documentclass[a4paper, 11pt, oneside]{impression}

\hwnumber{2} % Homework number in the cover page
\title{Lista de Exercícios} % Title
\subtitle{Introdução a Física de Hádrons} % Subtitle
\author{Lucas R. Ximenes} % Author
\extra{11917239} % Extra information

\begin{document}

\maketitle

\chapter{Funcional gerador de um campo escalar livre}\label{quest: one}

\begin{exercise}{}
    O funcional gerador de um campo escalar livre é escrito como
        \begin{equation*}
            Z_{0}[J] = N \int \exp\qty{
                i \int \qty[
                    \dfrac{1}{2}\qty(
                        \xi\vec{K}_{x}\xi +
                        \xi J +
                        \bar{\phi}\vec{K}_{x}\xi + 
                        \bar{\phi}J
                    ) + 
                    J\bar{\phi} + 
                    J\xi
                ]\dd[4]{x}
            }[D\xi]
        \end{equation*}
    após uma expansão do campo original $\phi$ em torno de sua configuração clássica $\bar{\phi}$ e sua correspondente flutuação $\xi$, $\phi = \bar{\phi} + \xi$. O operador $\vec{K}_{x}$ é o de Klein-Gordon, $\vec{K}_{x} = (\partial^{2} + m^{2})_{x}$. Obtenha
        \begin{equation*}
            Z_{0}[J] = N'\exp\qty[
                -\dfrac{i}{2}\int J(x)\Delta(x-y)J(y)\dd[4]{x}\dd[4]{y}
            ]
        \end{equation*}
\end{exercise}



Para determinar a ação, integramos a densidade de lagrangiana em $\dd[4]{x}$, de modo que
    \begin{equation*}
        S = \int \mathcal{L}_{0} \dd[4]{x}
    \end{equation*}

Fazendo uma variação na ação $\delta S$, como a densidade de lagrangiana $\mathcal{L}_{0} \equiv \mathcal{L}_{0}(\phi_{0}, \partial_{\mu}\phi_{0})$, a variação fica
    \begin{equation*}
        \delta S = \int \qty[
            \fdv{\mathcal{L}_{0}}{\phi_{0}} \delta\phi_{0} +
            \fdv{\mathcal{L}_{0}}{(\partial_{\mu}\phi_{0})} \delta(\partial_{\mu}\phi_{0})
        ] \dd[4]{x}
    \end{equation*}

Note que
    \begin{align*}
        \partial_{\mu}\qty[
            \fdv{\mathcal{L}_{0}}{(\partial_{\mu}\phi_{0})} \delta\phi_{0}
        ] = \partial_{\mu}\qty[
            \fdv{\mathcal{L}_{0}}{(\partial_{\mu}\phi_{0})}
        ]\delta\phi_{0} + 
        \fdv{\mathcal{L}_{0}}{(\partial_{\mu}\phi_{0})}\partial_{\mu}(\delta\phi_{0})
    \end{align*}

Como uma variação $\delta$ pode permutar com uma derivada parcial $\partial_{\mu}$, temos
    \begin{align*}
        \partial_{\mu}\qty[
            \fdv{\mathcal{L}_{0}}{(\partial_{\mu}\phi_{0})} \delta\phi_{0}
        ] &= \partial_{\mu}\qty[
            \fdv{\mathcal{L}_{0}}{(\partial_{\mu}\phi_{0})}
        ]\delta\phi_{0} + 
        \fdv{\mathcal{L}_{0}}{(\partial_{\mu}\phi_{0})}\delta(\partial_{\mu}\phi_{0})
    \end{align*}

Portanto podemos substituir a variação da ação por
    \begin{align*}
        \delta S &\eq \int \qty{
            \fdv{\mathcal{L}_{0}}{\phi_{0}} \delta\phi_{0} - \partial_{\mu}\qty[
                \fdv{\mathcal{L}_{0}}{(\partial_{\mu}\phi_{0})}
            ] \delta\phi_{0} +
            \partial_{\mu}\qty[
                \fdv{\mathcal{L}_{0}}{(\partial_{\mu}\phi_{0})}\delta\phi_{0}
            ]
        }\dd[4]{x} \\
        &\eq \int \qty{
            \fdv{\mathcal{L}_{0}}{\phi_{0}} - 
            \partial_{\mu}\qty[
                \fdv{\mathcal{L}_{0}}{(\partial_{\mu}\phi_{0})}
            ]
        }\delta\phi_{0} \dd[4]{x} +
        \int \partial_{\mu}\qty[
            \fdv{\mathcal{L}_{0}}{(\partial_{\mu}\phi_{0})}\delta\phi_{0}
        ]\dd[4]{x}
    \end{align*}

O último termo desta expressão é uma derivada total em todo o espaço-tempo, de modo que ao impormos que nos limites assintóticos o campo desaparece, concluímos que a integração vai dar zero, restando apenas
    \begin{equation*}
        \delta S = \int \qty{
            \fdv{\mathcal{L}_{0}}{\phi_{0}} - \partial_{\mu}\qty[\fdv{\mathcal{L}_{0}}{(\partial_{\mu}\phi_{0})}]
        }\delta\phi_{0} \dd[4]{x}
    \end{equation*}

O princípio de mínima ação $\delta S = 0$ fornece
    \begin{equation*}
        \delta S = \int \qty{
            \fdv{\mathcal{L}_{0}}{\phi_{0}} - \partial_{\mu}\qty[\fdv{\mathcal{L}_{0}}{(\partial_{\mu}\phi_{0})}]
        }\delta\phi_{0} \dd[4]{x} = 0
    \end{equation*}
que deve ser satisfeito para qualquer variação $\delta\phi_{0}$ do campo, logo impomos que o argumento dentro das chaves $\{\cdots\}$ é igual a zero, gerando a equação de Euler-Lagrange:
    \begin{equation*}
        \fdv{\mathcal{L}_{0}}{\phi_{0}} - \partial_{\mu}\qty[\fdv{\mathcal{L}_{0}}{(\partial_{\mu}\phi_{0})}] = 0
    \end{equation*}

Calculando o primeiro termo com base na densidade de lagrangiana, temos
    \begin{equation*}
        \fdv{\mathcal{L}_{0}}{\phi_{0}} = \fdv{\phi_{0}}(-\dfrac{1}{2}m^2\phi_{0}^2) = -\dfrac{1}{2}m^2 (2\phi_{0}) = -m^2\phi_{0}
    \end{equation*}
em que apenas o segundo termo da lagrangiana vai ser relevante, pois o primeiro depende apenas das derivadas do campo. Para o segundo termo da equação de Euler-Lagrange, apenas o primeiro termo da lagrangiana vai ser relevante, de modo que
    \begin{align*}
        \partial_{\mu}\qty[\fdv{\mathcal{L}_{0}}{(\partial_{\mu}\phi_{0})}] 
        &\eq \partial_{\mu}\qty[
            \fdv{}{(\partial_{\mu}\phi_{0})}\qty(\dfrac{1}{2}\partial_{\mu}\phi_{0}\partial^{\mu}\phi_{0})
        ] 
        = \partial_{\mu}\qty[
            \dfrac{1}{2}\fdv{}{(\partial_{\mu}\phi_{0})}\qty(g^{\alpha\beta}\partial_{\alpha}\phi_{0}\partial_{\beta}\phi_{0})
        ] \\
        &\eq \partial_{\mu}\qty{
            \dfrac{1}{2}g^{\alpha\beta}\qty[
                \fdv{(\partial_{\alpha}\phi_{0})}{(\partial_{\mu}\phi_{0})} \partial_{\beta}\phi_{0} +
                \partial_{\alpha}\phi_{0} \fdv{(\partial_{\beta}\phi_{0})}{(\partial_{\mu}\phi_{0})}
            ]
        }
    \end{align*}

Sabendo então que
    \begin{equation*}
        \fdv{(\partial_{\alpha}\phi_{0})}{(\partial_{\mu}\phi_{0})} = \delta_{\alpha}^{\mu} 
        \qquad \& \qquad 
        \fdv{(\partial_{\beta}\phi_{0})}{(\partial_{\mu}\phi_{0})} = \delta_{\beta}^{\mu} 
    \end{equation*}
obtemos
    \begin{align*}
        \partial_{\mu}\qty[\fdv{\mathcal{L}_{0}}{(\partial_{\mu}\phi_{0})}] &\eq \partial_{\mu}\qty[
            \dfrac{1}{2}g^{\alpha\beta} \qty(
                \delta_{\alpha}^{\mu}\partial_{\beta}\phi_{0} + 
                \delta_{\beta}^{\mu}\partial_{\alpha}\phi_{0}
            )
        ] = \partial_{\mu}\qty[
            \dfrac{1}{2}\qty(
                g^{\mu\beta}\partial_{\beta}\phi_{0} + 
                g^{\alpha\mu}\partial_{\alpha}\phi_{0}
            )
        ] \\
        &\eq \partial_{\mu}\qty[
            \dfrac{1}{2}\qty(
                \partial^{\mu}\phi_{0} + 
                \partial^{\mu}\phi_{0}
            )
        ] = \partial_{\mu}\partial^{\mu}\phi_{0}
    \end{align*}

Juntando então os resultados na equação de Euler-Lagrange:
    \begin{equation*}
        -m^2\phi_{0} - \partial_{\mu}\partial^{\mu}\phi_{0} = 0
    \end{equation*}

Concluindo que o campo livre $\phi_{0}$ satisfaz a equação de Klein-Gordon
    \begin{answer}\label{eq: Klein-Gordon}
        (\partial_{\mu}\partial^{\mu} + m^2)\phi_{0} = 0
    \end{answer}


% Sabemos que a \textbf{equação de Klein-Gordon} se escreve na forma $(\partial_{\mu} \partial^{\mu} + m^2)\phi(x) = 0$ e que ela descreve as equações de movimento de um \textbf{campo escalar livre}. Comecemos escrevendo $\phi(x)$ no espaço de Fourier (momento):
%     \begin{equation*}
%         \phi(x) = \int \dfrac{1}{(2\pi)^4} \tilde{\phi}(k) e^{-ikx} \dd[4]{k}
%     \end{equation*}

% Utilizando então a equação de Klein-Gordon, temos
%     \begin{equation*}
%         (\partial_{\mu}\partial^{\mu} + m^2)\int \dfrac{1}{(2\pi)^4} \tilde{\phi}(k) e^{-ikx} \dd[4]{k} = 0
%     \end{equation*}

% Como o d'Alambertiano está no espaço de posição e a integração é feita no espaço de momento, podemos inserir as derivações dentro da integral:
%     \begin{equation*}
%         \int \dfrac{1}{(2\pi)^4} \tilde{\phi}(k) \qty[\partial_{\mu}\partial^{\mu}\qty(e^{-ikx}) + m^2e^{-ikx}]\dd[4]{k} = 0
%     \end{equation*}

% Expandindo o d'Alambertiano sob a forma $\partial_{\mu}\partial^{\mu} = \pdv[2]{}{t} - \nabla^2$:
%     \begin{equation*}
%         \int \dfrac{1}{(2\pi)^4}\tilde{\phi}(k) \qty[\pdv[2]{}{t} \qty(e^{-ikx}) - \nabla^2 \qty(e^{-ikx}) + m^2e^{-ikx}]\dd[4]{k} = 0
%     \end{equation*}

% As derivações ficam
%     \begin{align*}
%         \pdv[2]{}{t}\qty(e^{-ikx}) &= \pdv[2]{}{t}\qty(e^{-ik_{0}t + i\vb{k}\cdot\vb{x}}) 
%         = \pdv[2]{}{t}\qty(e^{-ik_{0}t})e^{i\vb{k}\cdot\vb{x}} 
%         = (-ik_{0})^2 e^{-ik_{0}t}e^{i\vb{k}\cdot\vb{x}}
%         = -k_{0}^2 e^{-ikx} \\
%         \nabla^2\qty(e^{-ikx}) &= \nabla^2\qty(e^{-ik_{0}t+i\vb{k}\cdot\vb{x}}) 
%         = e^{-ik_{0}t} \nabla^2 \qty(e^{i\vb{k}\cdot\vb{x}}) 
%         = e^{-ik_{0}t} \abs{\vb{k}}^2 e^{i\vb{k}\cdot\vb{x}} 
%         = \abs{\vb{k}}^2 e^{-ikx}
%     \end{align*}
% e portanto
%     \begin{equation*}
%         \int \dfrac{1}{(2\pi)^4} \tilde{\phi}(k) \qty(-k_{0}^2 + \abs{\vb{k}}^2 + m^2)e^{-ikx}\dd[4]{k} = 0
%     \end{equation*}

% Podemos multiplicar ambos os lados da equação por $-1$ afim de visualizar que $k_{0}^2 - \abs{\vb{k}}^2 = k^2$, restando então
%     \begin{equation*}
%         \int \dfrac{1}{(2\pi)^4} \tilde{\phi}(k) \qty(k^2 - m^2)e^{-ikx}\dd[4]{k} = 0
%     \end{equation*}

% Olhando para esta equação, vemos que o $\tilde{\phi}(k)$ que satisfaz a igualdade consiste de modos onde $k^2 = m^2$, ou seja, \textit{on-shell}. Isto nos permite assumir que
%     \begin{equation*}
%         \tilde{\phi}(k) = 2\pi \delta(k^2-m^2)C(k)
%     \end{equation*}
% onde $C(k)$ é uma função arbitrária de $k$ a ser determinada. Portanto
%     \begin{align*}
%         \phi(x) &= \int \dfrac{1}{(2\pi)^4} 2\pi \delta(k^2 - m^2) C(k) e^{-ikx}\dd[4]{k} \\
%         &= \int \dfrac{1}{(2\pi)^3} \delta(k_{0}^2 - \abs{\vb{k}}^2 - m^2) C(k) e^{-ikx}\dd[4]{k}
%     \end{align*}

% Vemos então que o argumento da distribuição delta de Dirac possui zeros em $k_{0} = \pm \sqrt{\abs{\vb{k}}^2 + m^2}$. Podemos definir esta quantidade como sendo
%     \begin{equation*}
%         \omega_{\vb{k}} \coloneqq \sqrt{\vb{\abs{k}}^2 + m^2}
%     \end{equation*}
% como um análogo de uma frequência de um oscilador harmônico simples, pois ao passarmos a equação de Klein-Gordon para o espaço de Fourier, temos
%     \begin{equation*}
%         \qty[\pdv[2]{}{t} + \qty(\abs{\vb{k}}^2 + m^2)]\tilde{\phi}(k) = 0
%     \end{equation*}
% que é uma equação de movimento de um oscilador harmônico simples com a frequência de oscilação $\omega_{\vb{k}}$. Podemos então utilizar a seguinte propriedade da distribuição delta de Dirac:
%     \begin{equation*}
%         \delta[f(x)] = \sum_{f(x_{i})=0} \dfrac{1}{\abs{f'(x_{i})}} \delta(x-x_{i})
%     \end{equation*}

% No caso, $f(x)$ vai ser uma função de $k_{0}$, ou seja $f(k_{0}) = k_{0}^2 - \abs{\vb{k}}^2 - m^2$, cuja derivada é $f'(k_{0}) = 2k_{0}$, sendo assim, temos
%     \begin{equation*}
%         \delta(k^2 - m^2) = 
%         \underbrace{\dfrac{\delta(k_{0} - \omega_{\vb{k}})}{2\omega_{\vb{k}}}}_{k_{0}>0} + 
%         \underbrace{\dfrac{\delta(k_{0}+\omega_{\vb{k}})}{\abs{-2\omega_{\vb{k}}}}}_{k_{0}<0}
%     \end{equation*}

% % EU ACHO Q A FUNÇÃO DE HEAVISIDE ENTRA AQUI NESSE PONTO
% O fato do 4-momento ser sempre time-like, o sinal de $k_{0}$ é invariante de Lorentz, implicando que a integração de $k_{0}$ precisa ser feita apenas em $k_{0} > 0$, portanto, inserimos na integral uma função de Heaviside definida por
%     \begin{equation*}
%         \Theta(k_{0}) = \begin{cases}
%             1&, k_{0}\geqslant 0 \\
%             0&, k_{0}< 0
%         \end{cases}
%     \end{equation*} 

% Isto nos permite escrever
%     \begin{equation*}
%         \phi(x) = \int \dfrac{1}{(2\pi)^3}\Theta(k_{0})\qty[\dfrac{\delta(k_{0}-\omega_{\vb{k}})}{2\omega_{\vb{k}}} + \dfrac{\delta(k_{0} + \omega_{\vb{k}})}{\abs{2\omega_{\vb{k}}}}]C(k_{0},\vb{k})e^{-ik_{0}t + i\vb{k}\cdot\vb{x}}\dd[4]{k}
%     \end{equation*}

% Performando então uma integração em $k_{0}$, temos
%     \begin{align*}
%         \phi(x) &= \int \dfrac{1}{(2\pi)^3}\dfrac{1}{2\omega_{\vb{k}}}\qty[
%             C(\omega_{\vb{k}},\vb{k})e^{-i\omega_{\vb{k}}t + i\vb{k}\cdot\vb{x}} +
%             C(-\omega_{\vb{k}},\vb{k})e^{+i\omega_{\vb{k}}t + i\vb{k}\cdot\vb{x}}
%         ]\dd[3]{k}
%     \end{align*}

% Fazendo uma mudança no momento tridimensional $\vb{k} \mapsto -\vb{k}$ ({\color{orange}PORQUE?}) do segundo termo:
%     \begin{equation*}
%         \phi(x) = \int \dfrac{1}{(2\pi)^3}\dfrac{1}{2\omega_{\vb{k}}}\qty[
%             C(\omega_{\vb{k}},\vb{k})e^{-i\omega_{\vb{k}}t + i\vb{k}\cdot\vb{x}} +
%             C(-\omega_{\vb{k}}, -\vb{k})e^{i\omega_{\vb{k}}t - i\vb{k}\cdot\vb{x}}
%         ]\dd[3]{k}
%     \end{equation*}

% Afim de quantizar este campo, fazemos com que a função $C(k)$ seja proporcional aos operadores de criação ({\color{MyOrange}$a^{\dagger}(k)$}) e aniquilação ({\color{MyOrange}$a(k)$})
%     \begin{equation*}
%         C(\omega_{\vb{k}},\vb{k}) = \sqrt{2\omega_{\vb{k}}} a(k) \qquad \& \qquad 
%         C(-\omega_{\vb{k}},-\vb{k}) = \sqrt{2\omega_{\vb{k}}} a^{\dagger}(k)
%     \end{equation*}

% Podemos também chamar de $e^{-i\omega_{\vb{k}}t+i\vb{k}\cdot\vb{x}} = \sqrt{2\omega_{\vb{k}}}f_{k}(x)$, concluindo que
%     \begin{answer} \label{eq: free scalar field}
%         \phi(x) = \int \dfrac{1}{(2\pi)^3}\qty[
%             a(k)f_{k}(x) + 
%             a^{\dagger}(k)f_{k}^{\ast}(x)
%         ]\dd[3]{k}
%     \end{answer}

% Como o momento conjugado $\pi(x) = \dot{\phi}(x)$, temos
%     \begin{equation*}
%         \pi(x) = \int \dfrac{1}{(2\pi)^3} \qty[
%             a(k) \pdv{f_{k}(x)}{t} +
%             a^{\dagger}(k) \pdv{f_{k}^{\ast}(x)}{t}
%         ]\dd[3]{k}
%     \end{equation*}
% onde
%     \begin{align*}
%         \pdv{f_{k}(x)}{t} &= \sqrt{2\omega_{\vb{k}}}\pdv{}{t}\qty(e^{-i\omega_{\vb{k}}t})e^{i\vb{k}\cdot\vb{x}} = -i\omega_{\vb{k}} \sqrt{2\omega_{\vb{k}}}e^{-i\omega_{\vb{k}}t+i\vb{k}\cdot\vb{x}} = -i\omega_{\vb{k}}f_{k}(x) \\
%         \pdv{f_{k}^{\ast}(x)}{t} &= \sqrt{2\omega_{\vb{k}}}\pdv{}{t}\qty(e^{+i\omega_{\vb{k}}t})e^{-i\vb{k}\cdot\vb{x}} = +i\omega_{\vb{k}} \sqrt{2\omega_{\vb{k}}}e^{i\omega_{\vb{k}}t-i\vb{k}\cdot\vb{x}} = +i\omega_{\vb{k}}f_{k}^{\ast}(x)
%     \end{align*}

% Concluindo que
%     \begin{answer}\label{eq: conjugate momenta of free scalar field}
%         \pi(x) = \int \dfrac{i\omega_{\vb{k}}}{(2\pi)^3} \qty[
%             -a(k)f_{k}(x) +
%             a^{\dagger}(k)f_{k}^{\ast}(x)
%         ]\dd[3]{k}
%     \end{answer}

% % Toda esta construção impõe automaticamente a quantização dos campos $\phi(x)$ e $\pi(x)$ a partir da relação de comutação
% %     \begin{equation}\label{eq: commutation rule between free scalar fields}
% %         [\phi(\vb{x},t), \pi(\vb{y},t)] = i\delta^3(\vb{x} - \vb{y})
% %     \end{equation}

% Para computar os operadores de criação e aniquilação em termos do campo $\phi(x)$, aplicamos uma nova transformação de Fourier nos campos $\phi(x)$ e $\pi(x)$. Fazendo então uma transformada de Fourier em \eqref{eq: free scalar field}, temos
%     \begin{equation*}
%         \int \phi(x)e^{ikx}\dd[3]{x} = \int \qty{\int \dfrac{1}{(2\pi)^3}\qty[
%             a(k')f_{k'}(x) + 
%             a^{\dagger}(k')f_{k'}^{\ast}(x)
%         ]\dd[3]{k'}}e^{ikx}\dd[3]{x}
%     \end{equation*}

% Por simplicidade, retomarei $e^{-i\omega_{\vb{k}}t+i\vb{k}\cdot\vb{x}} = \sqrt{2\omega_{\vb{k}}} f_{k}(x)$ sob a forma $e^{-ikx} = \sqrt{2\omega_{\vb{k}}}f_{k}(x)$, assumindo que $k = (\omega_{\vb{k}},\vb{k})$. A expressão fica então
%     \begin{equation*}
%         \int \phi(x) e^{ikx}\dd[3]{x} = \iint \dfrac{1}{(2\pi)^3}\dfrac{1}{\sqrt{2\omega_{\vb{k}'}}}\qty[
%             a(k')e^{-i(k'-k)x} + 
%             a^{\dagger}(k')e^{i(k'+k)x}
%         ]\dd[3]{k'}\dd[3]{x}
%     \end{equation*}

% Como do lado direito da expressão, apenas as exponenciais dependem das coordenadas espaciais, temos como escrever
%     \begin{equation*}
%         \int \phi(x)e^{ikx}\dd[3]{x} = \int \dfrac{1}{(2\pi)^3}\dfrac{1}{\sqrt{2\omega_{\vb{k}'}}}\qty[
%             a(k')\int e^{-i(k'-k)x}\dd[3]{x} + 
%             a^{\dagger}(x) \int e^{i(k'+k)x}\dd[3]{x}
%         ]\dd[3]{k'}
%     \end{equation*}

% Note então que as integrais em $\dd[3]{x}$ são partes de uma das principais definições da distribuição delta de Dirac, de modo que
%     \begin{align*}
%         \int e^{-i(k'-k)x}\dd[3]{x} &= e^{-i(\omega_{\vb{k}'}-\omega_{\vb{k}})t} \int e^{i(\vb{k}'-\vb{k})\cdot\vb{x}} \dd[3]{x} \\
%         &= e^{-i(\omega_{\vb{k}'} - \omega_{\vb{k}})t} (2\pi)^3\delta^3(\vb{k}' - \vb{k}) \\
%         \int e^{i(k'+k)x}\dd[3]{x} &= e^{i(\omega_{\vb{k}'}+\omega_{\vb{k}})t} \int e^{-i(\vb{k}' + \vb{k})\cdot\vb{x}} \dd[3]{x} \\
%         &= e^{i(\omega_{\vb{k}'} + \omega_{\vb{k}})t} (2\pi)^3 \delta^3(\vb{k}' + \vb{k})
%     \end{align*}

% Sendo assim, obtemos
%     \begin{equation*}
%         \int \phi(x)e^{ikx}\dd[3]{x} = \int \dfrac{1}{\sqrt{2\omega_{\vb{k}'}}}\qty[
%             a(k')e^{-i(\omega_{\vb{k}'}-\omega_{\vb{k}})t}\delta^3(\vb{k}'-\vb{k}) + 
%             a^{\dagger}(k')e^{i(\omega_{\vb{k}'}+\omega_{\vb{k}})t}\delta^3(\vb{k}'+\vb{k})
%         ]\dd[3]{k'}
%     \end{equation*}

% Podemos notar também que $\omega_{-\vb{k}'} = \sqrt{\abs{-\vb{k}'}^2 + m^2} = \omega_{\vb{k}'}$, logo ao performar a integral obtemos
%     \begin{align*}
%         \int \phi(x) e^{ikx}\dd[3]{x} &= \dfrac{1}{\sqrt{2\omega_{\vb{k}}}}\qty[
%             a(k) e^{-i(\omega_{\vb{k}}-\omega_{\vb{k}})t} +
%             a^{\dagger}(k) e^{i(\omega_{-\vb{k}}+\omega_{\vb{k}})t}
%         ]\\ 
%         &= \dfrac{1}{\sqrt{2\omega_{\vb{k}}}}\qty[
%             a(k) + 
%             a^{\dagger}(k) e^{2i\omega_{\vb{k}}t}
%         ]
%     \end{align*}

% No caso do momento conjugado, temos
%     \begin{align*}
%         \int \pi(x) e^{ikx}\dd[3]{x} &= \int \qty{
%             \int \dfrac{i\omega_{\vb{k}'}}{(2\pi)^3} \qty[
%                 -a(k')f_{k}(x) +
%                 a^{\dagger}(k')f_{k'}^{\ast}(x)
%             ]\dd[3]{k'}
%         }e^{ikx}\dd[3]{x} \\
%         &= \iint \dfrac{1}{(2\pi)^3}\dfrac{i\omega_{\vb{k}'}}{\sqrt{2\omega_{\vb{k}'}}}\qty[
%             -a(k')e^{-i(k'-k)x} +
%             a^{\dagger}(k')e^{i(k'+k)x}
%         ]\dd[3]{k'}\dd[3]{x} \\
%         &= \int \dfrac{-i}{(2\pi)^3}\sqrt{\dfrac{\omega_{\vb{k}'}}{2}}\qty[
%             a(k')\int e^{-i(k'-k)x}\dd[3]{x} - 
%             a^{\dagger}(k')\int e^{i(k'+k)x}\dd[3]{x}
%         ]\dd[3]{k'} \\
%         &= -i\int \sqrt{\dfrac{\omega_{\vb{k}'}}{2}}\qty[
%             a(k')e^{-i(\omega_{\vb{k}'} - \omega_{\vb{k}})t} \delta^3(\vb{k}' - \vb{k}) -
%             a^{\dagger}(k') e^{i(\omega_{\vb{k}'} + \omega_{\vb{k}})t} \delta^3(\vb{k}' + \vb{k})
%         ] \dd[3]{k'} \\
%         &= -i\sqrt{\dfrac{\omega_{\vb{k}}}{2}}\qty[
%             a(k) e^{-i(\omega_{\vb{k}} - \omega_{\vb{k}})t} -
%             a^{\dagger}(k) e^{i(\omega_{-\vb{k}} + \omega_{\vb{k}})t}
%         ] \\
%         &= -i\sqrt{\dfrac{\omega_{\vb{k}}}{2}}\qty[
%             a(k) - a^{\dagger}(k)e^{2i\omega_{\vb{k}}t}
%         ]
%     \end{align*}

% Com estes resultados, podemos obter as formas
%     \begin{equation*}
%         \sqrt{2\omega_{\vb{k}}} \int \phi(x) e^{ikx}\dd[3]{x} = a(k) + a^{\dagger}(k) e^{2i\omega_{\vb{k}}t}
%     \end{equation*}
%     \begin{equation*}
%         i\sqrt{\dfrac{2}{\omega_{\vb{k}}}} \int \pi(x) e^{ikx}\dd[3]{x} = a(k) - a^{\dagger}(k) e^{2i\omega_{\vb{k}}t}
%     \end{equation*}

% Somando as duas equações:
%     \begin{align*}
%         a(k) &= \int \qty[
%             \dfrac{i}{\sqrt{2\omega_{\vb{k}}}}\pi(x) + 
%             \sqrt{\dfrac{\omega_{\vb{k}}}{2}}\phi(x) 
%         ] e^{ikx}\dd[3]{x} \\
%         &= \int \qty[
%             \dfrac{i}{\sqrt{2\omega_{\vb{k}}}}\dot{\phi}(x) +
%             \sqrt{\dfrac{\omega_{\vb{k}}}{2}}\phi(x) 
%         ]\sqrt{2\omega_{\vb{k}}}f_{k}^{\ast}(x) \dd[3]{x} \\
%         &= \int \qty[
%             i\dot{\phi}(x) + 
%             \omega_{\vb{k}} \phi(x)
%         ]f_{k}^{\ast}(x) \dd[3]{x} \\
%         &= \int \qty[
%             i\dot{\phi}(x)f_{k}^{\ast}(x) + 
%             \omega_{\vb{k}} \phi(x) f_{k}^{\ast}(x)
%         ]\dd[3]{x} \\
%         &= i\int \qty[
%             f_{k}^{\ast}(x)\dot{\phi}(x) -
%             i\omega_{\vb{k}}\phi(x)f_{k}^{\ast}(x)
%         ]\dd[3]{x} \\
%         &= i\int \qty[
%             f_{k}^{\ast}(x) \pdv{\phi(x)}{t} - 
%             \phi(x) \pdv{f_{k}^{\ast}(x)}{t}
%         ]\dd[3]{x}
%     \end{align*}

% Pela notação
%     \begin{equation*}
%         A \overset{\leftrightarrow}{\partial_0} B = 
%         A \pdv{B}{t} - \pdv{A}{t} B
%     \end{equation*}

% Podemos concluir que o operador de aniquilação $a(k)$ admite ser escrito sob a forma
%     \begin{answer}\label{eq: annihilation operator}
%         a(k) = i\int f_{k}^{\ast}(x) \overset{\leftrightarrow}{\partial_{0}}\phi(x) \dd[3]{x}
%     \end{answer}

% Como estamos considerando um campo $\phi(x)$ real, temos $\phi^{\dagger}(x) = \phi(x)$, portanto ao calcular $a^{\dagger}(k)$, trocaremos $i \mapsto -i$ e $f_{k}^{\ast}(x) \mapsto f_{k}(x)$, concluindo que
%     \begin{answer}\label{eq: creation operator}
%         a^{\dagger}(k) = -i\int f_{k}(x) \overset{\leftrightarrow}{\partial_{0}}\phi(x)\dd[3]{x}
%     \end{answer}

% Para determinar a relação de comutação $[\phi(\vb{x},t), \pi(\vb{y},t)]$, podemos expandir separadamente os termos do comutador:
%     \begin{align*}
%         \phi(\vb{x},t)\pi(\vb{y},t) 
%         &= \int \dfrac{1}{(2\pi)^3}\qty[
%             a(k) f_{k}(x) + a^{\dagger}(k) f_{k}^{\ast}(x)
%         ] \dd[3]{k} \int \dfrac{-i\omega_{\vb{p}}}{(2\pi)^3}\qty[
%             a(p) f_{p}(y) - a^{\dagger}(p) f_{p}^{\ast}(y)
%         ] \dd[3]{p} \\
%         &= \iint \dfrac{-i\omega_{\vb{p}}}{(2\pi)^6}\qty[
%             a(k) f_{k}(x) + a^{\dagger}(k) f_{k}^{\ast}(x)
%         ]\qty[
%             a(p) f_{p}(y) - a^{\dagger}(p) f_{p}^{\ast}(y)
%         ] \dd[3]{k} \dd[3]{p} \\
%         &= \iint \dfrac{-i\omega_{\vb{p}}}{(2\pi)^6}\Big[
%             a(k)a(p) f_{k}(x)f_{p}(y) -
%             a(k)a^{\dagger}(p) f_{k}(x)f_{p}^{\ast}(y) + \\
%         &+  a^{\dagger}(k)a(p) f_{k}^{\ast}(x) f_{p}(y) -
%             a^{\dagger}(k)a^{\dagger}(p) f_{k}^{\ast}(x) f_{p}^{\ast}(y)
%         \Big] \dd[3]{k} \dd[3]{p}
%     \end{align*}
%     \begin{align*}
%         \pi(\vb{y},t)\phi(\vb{x},t) 
%         &= \int \dfrac{-i\omega_{\vb{p}}}{(2\pi)^3}\qty[
%             a(p) f_{p}(y) - a^{\dagger}(p) f_{p}^{\ast}(y)
%         ] \dd[3]{p} \int \dfrac{1}{(2\pi)^3}\qty[
%             a(k) f_{k}(x) + a^{\dagger}(k) f_{k}^{\ast}(x)
%         ] \dd[3]{k} \\
%         &= \iint \dfrac{-i\omega_{\vb{p}}}{(2\pi)^6}\qty[
%             a(p) f_{p}(y) - a^{\dagger}(p) f_{p}^{\ast}(y)
%         ]\qty[
%             a(k) f_{k}(x) + a^{\dagger}(k) f_{k}^{\ast}(x)
%         ] \dd[3]{k} \dd[3]{p} \\
%         &= \iint \dfrac{-i\omega_{\vb{p}}}{(2\pi)^6}\Big[
%             a(p) a(k) f_{p}(y) f_{k}(x) + 
%             a(p) a^{\dagger}(k) f_{p}(y) f_{k}^{\ast}(x) - \\
%         &-  a^{\dagger}(p) a(k) f_{p}^{\ast}(y) f_{k}(x) -
%             a^{\dagger}(p) a^{\dagger}(k) f_{p}^{\ast}(y) f_{k}^{\ast}(x)
%         \Big] \dd[3]{k} \dd[3]{p}
%     \end{align*}

% Juntando tudo no comutador $[\phi(\vb{x},t),\pi(\vb{y},t)]$:
%     \begin{align*}
%         [\phi(\vb{x},t),\pi(\vb{y},t)] 
%         &= \phi(\vb{x},t)\pi(\vb{y},t) - \pi(\vb{y},t)\phi(\vb{x},t) \\
%         &= \iint \dfrac{-i\omega_{\vb{p}}}{(2\pi)^6} \Big[
%             a(k)a(p) f_{k}(x)f_{p}(y) -
%             {\color{MyOrange} a(k)a^{\dagger}(p) f_{k}(x)f_{p}^{\ast}(y)} + \\
%         &+  {\color{Blue} a^{\dagger}(k)a(p) f_{k}^{\ast}(x) f_{p}(y)} -
%             a^{\dagger}(k)a^{\dagger}(p) f_{k}^{\ast}(x) f_{p}^{\ast}(y) - \\
%         &-  a(p) a(k) f_{p}(y) f_{k}(x) - 
%             {\color{Blue} a(p) a^{\dagger}(k) f_{p}(y) f_{k}^{\ast}(x)} - \\
%         &+  {\color{MyOrange} a^{\dagger}(p) a(k) f_{p}^{\ast}(y) f_{k}(x)} +
%             a^{\dagger}(p) a^{\dagger}(k) f_{p}^{\ast}(y) f_{k}^{\ast}(x)
%         \Big] \dd[3]{k} \dd[3]{p} \\
%         &= \iint \dfrac{-i\omega_{\vb{p}}}{(2\pi)^6} \Big[
%             [a(k), a(p)] f_{k}(x) f_{p}(y) + 
%             {\color{MyOrange}[a^{\dagger}(p),a(k)] f_{k}(x) f_{p}^{\ast}(y)} - \\
%         &-  {\color{Blue}[a(p), a^{\dagger}(k)] f_{k}^{\ast}(x) f_{p}(y)} +
%             [a^{\dagger}(p), a^{\dagger}(k)] f_{p}^{\ast}(y)  f_{k}^{\ast}(x)
%         \Big] \dd[3]{k} \dd[3]{p}
%     \end{align*}

% Considerando então as relações de comutação
%     \begin{equation}\label{eq: canonical comutation rules}
%         \begin{matrix}
%             [a(k), a^{\dagger}(k')] = (2\pi)^3 \delta^3(\vb{k} - \vb{k}') \\ \\
%             [a(k), a(k')] = 0 \qquad \& \qquad 
%             [a^{\dagger}(k), a^{\dagger}(k')] = 0
%         \end{matrix}
%     \end{equation}
% temos
%     \begin{equation*}
%         [\phi(\vb{x},t), \pi(\vb{y},t)] = \iint \dfrac{-i\omega_{\vb{p}}}{(2\pi)^3} \Big[
%             -\delta^3(\vb{k} - \vb{p}) f_{k}(x) f_{p}^{\ast}(y) -
%             \delta^3(\vb{p} - \vb{k}) f_{k}^{\ast}(x) f_{p}(y)
%         \Big]\dd[3]{k} \dd[3]{p}
%     \end{equation*}

% Expandindo $f_{k}(x)f_{p}^{\ast}(y)$ e $f_{k}^{\ast}(x)f_{p}(y)$:
%     \begin{align*}
%         f_{k}(x)f_{p}^{\ast}(y) &= \dfrac{1}{\sqrt{2\omega_{\vb{k}}}}e^{-ikx} \dfrac{1}{\sqrt{2\omega_{\vb{p}}}}e^{ipy} &
%         f_{k}^{\ast}(x)f_{p}(y) &= \dfrac{1}{\sqrt{2\omega_{\vb{k}}}}e^{ikx} \dfrac{1}{\sqrt{2\omega_{\vb{p}}}}e^{-ipy} \\
%         &= \dfrac{1}{\sqrt{2\omega_{\vb{k}}}\sqrt{2\omega_{\vb{p}}}}e^{-i\omega_{\vb{k}}t + i\vb{k}\cdot\vb{x}}e^{i\omega_{\vb{p}}t - \vb{p}\cdot\vb{y}} & &= \dfrac{1}{\sqrt{2\omega_{\vb{k}}}\sqrt{2\omega_{\vb{p}}}}e^{i\omega_{\vb{k}}t - i\vb{k}\cdot\vb{x}} e^{-i\omega_{\vb{p}}t + i\vb{p}\cdot\vb{y}} \\
%         &= \dfrac{e^{-i(\omega_{\vb{k}} - \omega_{\vb{p}})}}{\sqrt{2\omega_{\vb{k}}}\sqrt{2\omega_{\vb{p}}}}e^{i(\vb{k}\cdot\vb{x} - \vb{p}\cdot\vb{y})} & &= \dfrac{e^{i(\omega_{\vb{k}} - \omega_{\vb{p}})}}{\sqrt{2\omega_{\vb{k}}}\sqrt{2\omega_{\vb{p}}}}e^{-i(\vb{k}\cdot\vb{x} - \vb{p}\cdot\vb{y})}
%     \end{align*}

% Como $\delta(\vb{x}) = \delta(-\vb{x})$, temos a forma expandida
%     \begin{align*}
%         [\phi(\vb{x},t), \pi(\vb{y},t)] &= \iint \dfrac{i\omega_{\vb{p}}}{(2\pi)^3}\dfrac{1}{\sqrt{2\omega_{\vb{k}}}\sqrt{2\omega_{\vb{p}}}}\Big[
%             e^{-i(\omega_{\vb{k}} - \omega_{\vb{p}})} e^{i(\vb{k}\cdot\vb{x} - \vb{p}\cdot\vb{y})} + \\
%         &+  e^{i(\omega_{\vb{k}} - \omega_{\vb{p}})} e^{-i(\vb{k}\cdot\vb{x} - \vb{p}\cdot\vb{y})} \Big]\delta^3(\vb{p} - \vb{k}) \dd[3]{k} \dd[3]{p}
%     \end{align*}

% Realizando a integração em $\dd[3]{p}$, temos pela presença da distribuição delta de Dirac que $\vb{p} \mapsto \vb{k}$, resultando em
%     \begin{align*}
%         [\phi(\vb{x},t), \pi(\vb{y},t)] &= \int \dfrac{i\omega_{\vb{k}}}{(2\pi)^3}\dfrac{1}{\sqrt{2\omega_{\vb{k}}}\sqrt{2\omega_{\vb{k}}}}\qty[
%             e^{-i(\omega_{\vb{k}} - \omega_{\vb{k}})} e^{i(\vb{k}\cdot\vb{x} - \vb{k}\cdot\vb{y})} + 
%             e^{i(\omega_{\vb{k}} - \omega_{\vb{k}})} e^{-i(\vb{k}\cdot\vb{x} - \vb{k}\cdot\vb{y})}
%         ] \dd[3]{k} \\
%         &= \dfrac{i}{2}\int \dfrac{1}{(2\pi)^3}\qty[
%             e^{i\vb{k}\cdot(\vb{x} - \vb{y})} +
%             e^{-i\vb{k}\cdot(\vb{x} - \vb{y})}
%         ] \dd[3]{k} \\
%         &= \dfrac{i}{2}\qty[
%             \int \dfrac{1}{(2\pi)^3}e^{i\vb{k}\cdot(\vb{x}-\vb{y})}\dd[3]{k} +
%             \int \dfrac{1}{(2\pi)^3}e^{-i\vb{k}\cdot(\vb{x}-\vb{y})}\dd[3]{k}
%         ] \\
%         &= \dfrac{i}{2}\qty[
%             \delta^3(\vb{x} - \vb{y}) + \delta^3(\vb{y} - \vb{x})
%         ]
%     \end{align*}

% Concluindo que a partir das relações de comutação \eqref{eq: canonical comutation rules}, chegamos em
%     \begin{answer}\label{eq: canonical commutation rule between free scalar fields}
%         [\phi(\vb{x},t), \pi(\vb{y},t)] = i\delta^3(\vb{x} - \vb{y})
%     \end{answer}

% Sabendo da forma da densidade de lagrangiana livre $\mathcal{L}(\phi,\partial_{\mu}\phi)$ e dos campos $\phi(\vb{x},t)$ e $\pi(\vb{x},t)$, podemos determinar a densidade de hamiltoniana $\mathcal{H}(\phi,\pi,\partial_{\mu}\phi,\partial_{\mu}\pi)$ por
%     \begin{equation*}
%         \mathcal{H}(\phi,\pi,\partial_{\mu}\phi,\partial_{\mu}\pi) = \dot{\phi}(x)\pi(x) - \mathcal{L}(\phi,\partial_{\mu}\phi)
%     \end{equation*}

% Como $\pi(x) = \partial_{0}\phi(x)$, é fácil ver que
%     \begin{align*}
%         \mathcal{H} &= \pi^2 - \dfrac{1}{2}\partial_{\mu}\phi\partial^{\mu}\phi + \dfrac{1}{2}m^2\phi^2 \\
%         &= \pi^2 - \dfrac{1}{2}\dot{\phi}^2 + \dfrac{1}{2}\nabla\phi\cdot\nabla\phi + \dfrac{1}{2} m^2 \phi^2 \\
%         &= \dfrac{1}{2}\pi^2 + \dfrac{1}{2}(\nabla\phi)^2 + \dfrac{1}{2}m^2\phi^2
%     \end{align*}

% Concluindo que 
%     \begin{answer}\label{eq: hamiltonian density}
%         \mathcal{H} = \dfrac{1}{2}\qty[
%             \pi^2 + (\nabla\phi)^2 + m^2\phi^2
%         ]
%     \end{answer}

% Tendo a densidade de hamiltoniana, somos então capazes de mostrar a hamiltoniana de fato, então ``basta'' integrar nas coordenadas espaciais:
%     \begin{align*}
%         H &= \int \mathcal{H} \dd[3]{x} = \dfrac{1}{2}\int [\pi^2 + (\nabla\phi)^2 + m^2\phi^2]\dd[3]{x} \\
%         &= \dfrac{1}{2}\int \pi^2\dd[3]{x} + \dfrac{1}{2}\int \nabla\phi\cdot\nabla\phi \dd[3]{x} + \dfrac{m^2}{2}\int \phi^2\dd[3]{x}
%     \end{align*}

% O primeiro termo vai ficar
%     \begin{align*}
%         \dfrac{1}{2}\int\pi^2\dd[3]{x} 
%         =&\; \dfrac{1}{2}\iiint \dfrac{\omega_{\vb{k}}\omega_{\vb{k}'}}{(2\pi)^3}
%             \qty[
%                 a(k)f_{k}(x) - a^{\dagger}(k)f_{k}^{\ast}(x)
%             ]\qty[
%                 a(k')f_{k'}(x) - a^{\dagger}(k')f_{k'}^{\ast}(x)
%             ]\dd[3]{k}\dd[3]{k'}
%         \dd[3]{x} \\
%         =&\; \dfrac{1}{2}\iiint \dfrac{\omega_{\vb{k}}\omega_{\vb{k}'}}{(2\pi)^3}
%         \Big[
%             a(k)a(k')f_{k}(x)f_{k'}(x) - a(k)a^{\dagger}(k')f_{k}(x)f_{k'}^{\ast}(x) - \\
%         &- a^{\dagger}(k)a(k')f_{k}^{\ast}(x)f_{k'}(x) + a^{\dagger}(k)a^{\dagger}(k')f_{k}^{\ast}(x)f_{k'}^{\ast}(x)
%         \Big]\dd[3]{k}\dd[3]{k'}\dd[3]{x}
%     \end{align*}

% Lembrando da forma de $f_{k}(x)$, temos que os produtos entre estas funções são
%     \begin{align*}
%         f_{k}(x)f_{k'}(x) 
%         &= \dfrac{1}{\sqrt{2\omega_{\vb{k}}}\sqrt{2\omega_{\vb{k}'}}} 
%             e^{-i\omega_{\vb{k}}t + i\vb{k}\cdot\vb{x}} 
%             e^{-i\omega_{\vb{k}'}t + i\vb{k}'\cdot\vb{x}} 
%         = \dfrac{1}{\sqrt{2\omega_{\vb{k}}}\sqrt{2\omega_{\vb{k}'}}} 
%             e^{-i(\omega_{\vb{k}} + \omega_{\vb{k}'})t} 
%             e^{i(\vb{k}+\vb{k}')\cdot\vb{x}}
%     \end{align*}
%     \begin{align*}
%         f_{k}(x)f_{k'}^{\ast}(x) 
%         &= \dfrac{1}{\sqrt{2\omega_{\vb{k}}}\sqrt{2\omega_{\vb{k}'}}} 
%             e^{-i\omega_{\vb{k}}t + i\vb{k}\cdot\vb{x}} 
%             e^{i\omega_{\vb{k'}}t - i\vb{k}'\cdot\vb{x}} 
%         = \dfrac{1}{\sqrt{2\omega_{\vb{k}}}\sqrt{2\omega_{\vb{k}'}}} 
%             e^{-i(\omega_{\vb{k}} - \omega_{\vb{k}'})t} 
%             e^{i(\vb{k} - \vb{k}')\cdot\vb{x}}
%     \end{align*}
%     \begin{align*}
%         f_{k}^{\ast}(x)f_{k'}(x) 
%         &= \dfrac{1}{\sqrt{2\omega_{\vb{k}}}\sqrt{2\omega_{\vb{k}'}}} 
%             e^{i\omega_{\vb{k}}t - i\vb{k}\cdot\vb{x}} 
%             e^{-i\omega_{\vb{k}'}t + i\vb{k}'\cdot\vb{x}} 
%         = \dfrac{1}{\sqrt{2\omega_{\vb{k}}}\sqrt{2\omega_{\vb{k}'}}} 
%             e^{i(\omega_{\vb{k}} - \omega_{\vb{k}'})t} 
%             e^{-i(\vb{k} - \vb{k}')\cdot\vb{x}}
%     \end{align*}
%     \begin{align*}
%         f_{k}^{\ast}(x)f_{k'}^{\ast}(x) 
%         &= \dfrac{1}{\sqrt{2\omega_{\vb{k}}}\sqrt{2\omega_{\vb{k}'}}} 
%             e^{i\omega_{\vb{k}}t - i\vb{k}\cdot\vb{x}} 
%             e^{i\omega_{\vb{k}'}t - i\vb{k}'\cdot\vb{x}} 
%         = \dfrac{1}{\sqrt{2\omega_{\vb{k}}}\sqrt{2\omega_{\vb{k}'}}} 
%             e^{i(\omega_{\vb{k}} + \omega_{\vb{k}'})t} 
%             e^{-i(\vb{k} + \vb{k}')\cdot\vb{x}}
%     \end{align*}

% Note então que as integrais em $\dd[3]{x}$ podem ser feitas apenas nos produtos de $f_{k}(x)$, pois os operadores de criação e aniquilação independem de $\vb{x}$, logo, junto com um fator $1/(2\pi)^3$, temos
%     \begin{align*}
%         \dfrac{1}{(2\pi)^3}\int f_{k}(x)f_{k'}(x)\dd[3]{x} 
%         &= \dfrac{e^{-i(\omega_{\vb{k}} + \omega_{\vb{k}'})t}}{\sqrt{2\omega_{\vb{k}}}\sqrt{2\omega_{\vb{k}'}}} 
%         \int \dfrac{1}{(2\pi)^3} e^{i(\vb{k} + \vb{k}')\cdot\vb{x}}\dd[3]{x} 
%         = \dfrac{e^{-i(\omega_{\vb{k}} + \omega_{\vb{k}'})t}}{\sqrt{2\omega_{\vb{k}}}\sqrt{2\omega_{\vb{k}'}}} \delta^3(\vb{k} + \vb{k}')
%     \end{align*}

\chapter{Integração gaussiana com números de Grassman}\label{quest: two}

\begin{exercise}{}
    A integral de trajetória para férmions envolve o cálculo, no limite do contínuo, da integral discretizada
        \begin{equation*}
            I(A) = \int \exp\qty(\sum_{j,k=1}^{N}\bar{\theta}_{j}A_{jk}\theta_{k})\prod_{i=1}^{N}\dd{\theta_{i}}\dd{\bar{\theta}_{i}} = \text{det}(A)
        \end{equation*}
    Mostre explicitamente o resultado acima para $N = 3$, e depois, generalize para um $N$ arbitrário.
\end{exercise}

Para $N=3$, temos diretaamente que
    \begin{align*}
        I(\vb{A}) &\eq \int \exp\qty(\sum_{j,k=1}^{3}\bar{\theta}_{j}A_{jk}\theta_{k})\prod_{i=1}^{3}\dd{\theta_{i}} \dd{\bar{\theta}_{i}} \\
    \end{align*}

Abrindo a exponencial em uma série de Taylor, temos
    \begin{align*}
        I(\vb{A}) &\eq \int \qty[
            1 + \sum_{j,k=1}^{3}\bar{\theta}_{j}A_{jk}\theta_{k} + 
            \dfrac{1}{2!}\qty(\sum_{j,k=1}^{3}\bar{\theta}_{j}A_{jk}\theta_{k})^{2} + 
            \dfrac{1}{3!}\qty(\sum_{j,k=1}^{3}\bar{\theta}_{j}A_{jk}\theta_{k})^{3} + \cdots
        ]\prod_{i=1}^{3}\dd{\theta_{i}}\dd{\bar{\theta}_{i}} \\
        &\eq \int \prod_{i=1}^{3}\dd{\theta_{i}}\dd{\bar{\theta}_{i}} + 
        \int \sum_{j,k=1}^{3}\bar{\theta}_{j}A_{jk}\theta_{k}\prod_{i=1}^{3}\dd{\theta_{i}}\dd{\bar{\theta}_{i}} + \dfrac{1}{2!}\int \qty(\sum_{j,k=1}^{3}\bar{\theta}_{j}A_{jk}\theta_{k})^{2}\prod_{i=1}^{3}\dd{\theta_{i}}\dd{\bar{\theta}_{i}} + \\
        &\noeq + \dfrac{1}{3!}\int \qty(\sum_{j,k=1}^{3}\bar{\theta}_{j}A_{jk}\theta_{k})^{3}\prod_{i=1}^{3}\dd{\theta_{i}}\dd{\bar{\theta}_{i}} + \cdots
    \end{align*}

O primeiro termo é claramente nulo, pois não há variáveis de Grassman para integrar. O segundo termo também é nulo, pois cada termo da soma possui apenas um $\theta$ e um $\bar{\theta}$, e portanto, ao integrar sobre as outras variáveis, o resultado será zero. O terceiro termo também é nulo, pois cada termo da soma ao quadrado terá no máximo dois $\theta$ e dois $\bar{\theta}$, e portanto, ao integrar sobre as outras variáveis, o resultado será zero. Restando apenas o quarto termo, onde também podemos levar em conta que qualquer termo subsequente da expansão de Taylor será nulo, pois terá mais de três $\theta$ ou $\bar{\theta}$. 

O quarto termo pode ser escrito como
    \begin{align*}
        I(\vb{A}) &\eq \dfrac{1}{3!}\int \sum_{j,k,\ell,m,n,p=1}^{3}\bar{\theta}_{j}A_{jk}\theta_{k}\bar{\theta}_{\ell}A_{\ell m}\theta_{m}\bar{\theta}_{n}A_{np}\theta_{p}\prod_{i=1}^{3}\dd{\theta_{i}}\dd{\bar{\theta}_{i}}
    \end{align*}

Note que, para que a integral não seja nula, é necessário que $j \neq \ell \neq n$, assim como $k \neq m \neq p$. Um ponto a se notar é que temos essencialmente $9^{3}$ termos dentro dessa integral, mas muitos deles são idênticos, pois a ordem dos fatores não importa. Por exemplo, o termo com $j=1$, $\ell=2$, $n=3$, $k=1$, $m=2$ e $p=3$
    \begin{equation*}
        \bar{\theta}_{1}A_{11}\theta_{1}\bar{\theta}_{2}A_{22}\theta_{2}\bar{\theta}_{3}A_{33}\theta_{3}
    \end{equation*}
é idêntico ao termo com $j=2$, $\ell=1$, $n=3$, $k=2$, $m=1$ e $p=3$
    \begin{equation*}
        \bar{\theta}_{2}A_{22}\theta_{2}\bar{\theta}_{1}A_{11}\theta_{1}\bar{\theta}_{3}A_{33}\theta_{3}
    \end{equation*}
pois levando em conta a anticomutatividade dos números de Grassmann, faremos 4 trocas de posição para chegar de um termo ao outro, o que é equivalente a multiplicar por $(-1)^{4} = 1$. Isto faz com que muitos termos sejam idênticos, e portanto, possamos considerar apenas um representante de cada conjunto de termos idênticos. Note que, para cada conjunto de termos idênticos, há exatamente $3! = 6$ termos, pois há $3!$ maneiras de ordenar os índices $j$, $\ell$ e $n$, e outras $3!$ maneiras de ordenar os índices $k$, $m$ e $p$. Portanto, podemos eliminar o fator $1/3!$ que está na frente da integral. Os representantes não-nulos formam então o seguinte resultado
    \begin{align*}
        I(\vb{A}) &\eq \int (
            \bar{\theta}_{1}A_{11}\theta_{1}\bar{\theta}_{2}A_{22}\theta_{2}\bar{\theta}_{3}A_{33}\theta_{3} + 
            \bar{\theta}_{1}A_{11}\theta_{1}\bar{\theta}_{2}A_{23}\theta_{3}\bar{\theta}_{3}A_{32}\theta_{2} + 
            \bar{\theta}_{1}A_{12}\theta_{2}\bar{\theta}_{2}A_{21}\theta_{1}\bar{\theta}_{3}A_{33}\theta_{3} + \\
            &\noeq + 
            \bar{\theta}_{1}A_{12}\theta_{2}\bar{\theta}_{2}A_{23}\theta_{3}\bar{\theta}_{3}A_{31}\theta_{1} + 
            \bar{\theta}_{1}A_{13}\theta_{3}\bar{\theta}_{2}A_{21}\theta_{1}\bar{\theta}_{3}A_{32}\theta_{2} + 
            \bar{\theta}_{1}A_{13}\theta_{3}\bar{\theta}_{2}A_{22}\theta_{2}\bar{\theta}_{3}A_{31}\theta_{1}
        ) \times \\
        &\noeq \times 
            \dd{\theta_{1}} \dd{\bar{\theta}_{1}} \dd{\theta_{2}} \dd{\bar{\theta}_{2}} \dd{\theta_{3}} \dd{\bar{\theta}_{3}} \\
        &\eq 
        \int 
            \bar{\theta}_{1}\theta_{1}\bar{\theta}_{2}\theta_{2}\bar{\theta}_{3}\theta_{3}(A_{11}A_{22}A_{33}) 
            \dd{\theta_{1}} \dd{\bar{\theta}_{1}} \dd{\theta_{2}} \dd{\bar{\theta}_{2}} \dd{\theta_{3}} \dd{\bar{\theta}_{3}} + 
        \int 
            \bar{\theta}_{1}\theta_{1}\bar{\theta}_{2}\theta_{3}\bar{\theta}_{3}\theta_{2}(A_{11}A_{23}A_{32})\times \\
        &\noeq \times 
            \dd{\theta_{1}} \dd{\bar{\theta}_{1}} \dd{\theta_{2}} \dd{\bar{\theta}_{2}} \dd{\theta_{3}} \dd{\bar{\theta}_{3}} + 
        \int 
            \bar{\theta}_{1}\theta_{2}\bar{\theta}_{2}\theta_{1}\bar{\theta}_{3}\theta_{3}(A_{12}A_{21}A_{33}) 
            \dd{\theta_{1}} \dd{\bar{\theta}_{1}} \dd{\theta_{2}} \dd{\bar{\theta}_{2}} \dd{\theta_{3}} \dd{\bar{\theta}_{3}} + \\
        &\noeq + 
        \int 
            \bar{\theta}_{1}\theta_{2}\bar{\theta}_{2}\theta_{3}\bar{\theta}_{3}\theta_{1}(A_{12}A_{23}A_{31}) 
            \dd{\theta_{1}} \dd{\bar{\theta}_{1}} \dd{\theta_{2}} \dd{\bar{\theta}_{2}} \dd{\theta_{3}} \dd{\bar{\theta}_{3}} + 
        \int 
            \bar{\theta}_{1}\theta_{3}\bar{\theta}_{2}\theta_{1}\bar{\theta}_{3}\theta_{2}(A_{13}A_{21}A_{32}) \times \\
        &\noeq \times 
            \dd{\theta_{1}} \dd{\bar{\theta}_{1}} \dd{\theta}_{2} \dd{\bar{\theta}_{2}} \dd{\theta}_{3} \dd{\bar{\theta}_{3}} + 
        \int 
            \bar{\theta}_{1}\theta_{3}\bar{\theta}_{2}\theta_{2}\bar{\theta}_{3}\theta_{1}(A_{13}A_{22}A_{31}) 
            \dd{\theta_{1}} \dd{\bar{\theta}_{1}} \dd{\theta_{2}} \dd{\bar{\theta}_{2}} \dd{\theta}_{3} \dd{\bar{\theta}_{3}}
    \end{align*}

Aqui precisamos levar em conta que a ordem da integração importa, portanto é necessário que a ordem dos fatores dentro da integral seja a mesma que a ordem de integração, que é
    \begin{equation*}
        \dd{\theta_{1}} \to \dd{\bar{\theta}_{1}} \to \dd{\theta_{2}} \to \dd{\bar{\theta}_{2}} \to \dd{\theta_{3}} \to \dd{\bar{\theta}_{3}}
    \end{equation*}

Sendo assim, os termos ficam:
    \begin{align*}
        \bar{\theta}_{1}\theta_{1}\bar{\theta}_{2}\theta_{2}\bar{\theta}_{3}\theta_{3} &= (-1)^{12} \bar{\theta}_{3}\theta_{3}\bar{\theta}_{2}\theta_{2}\bar{\theta}_{1}\theta_{1} = \bar{\theta}_{3}\theta_{3}\bar{\theta}_{2}\theta_{2}\bar{\theta}_{1}\theta_{1} \\
        \bar{\theta}_{1}\theta_{1}\bar{\theta}_{2}\theta_{3}\bar{\theta}_{3}\theta_{2} &= (-1)^{11}\bar{\theta}_{3}\theta_{3}\bar{\theta}_{2}\theta_{2}\bar{\theta}_{1}\theta_{1} = -\bar{\theta}_{3}\theta_{3}\bar{\theta}_{2}\theta_{2}\bar{\theta}_{1}\theta_{1}\\
        \bar{\theta}_{1}\theta_{2}\bar{\theta}_{2}\theta_{1}\bar{\theta}_{3}\theta_{3} &= (-1)^{11}\bar{\theta}_{3}\theta_{3}\bar{\theta}_{2}\theta_{2}\bar{\theta}_{1}\theta_{1} = -\bar{\theta}_{3}\theta_{3}\bar{\theta}_{2}\theta_{2}\bar{\theta}_{1}\theta_{1}\\
        \bar{\theta}_{1}\theta_{2}\bar{\theta}_{2}\theta_{3}\bar{\theta}_{3}\theta_{1} &= (-1)^{10}\bar{\theta}_{3}\theta_{3}\bar{\theta}_{2}\theta_{2}\bar{\theta}_{1}\theta_{1} = \bar{\theta}_{3}\theta_{3}\bar{\theta}_{2}\theta_{2}\bar{\theta}_{1}\theta_{1}\\
        \bar{\theta}_{1}\theta_{3}\bar{\theta}_{2}\theta_{1}\bar{\theta}_{3}\theta_{2} &= (-1)^{8}\bar{\theta}_{3}\theta_{3}\bar{\theta}_{2}\theta_{2}\bar{\theta}_{1}\theta_{1} = \bar{\theta}_{3}\theta_{3}\bar{\theta}_{2}\theta_{2}\bar{\theta}_{1}\theta_{1}\\
        \bar{\theta}_{1}\theta_{3}\bar{\theta}_{2}\theta_{2}\bar{\theta}_{3}\theta_{1} &= (-1)^{7}\bar{\theta}_{3}\theta_{3}\bar{\theta}_{2}\theta_{2}\bar{\theta}_{1}\theta_{1} = -\bar{\theta}_{3}\theta_{3}\bar{\theta}_{2}\theta_{2}\bar{\theta}_{1}\theta_{1}
    \end{align*}

Essa ordenação é interessante, pois ao integrarmos, todos terão o mesmo valor a menos de um sinal. Este valor é
    \begin{equation*}
        \int \bar{\theta}_{3}\theta_{3}\bar{\theta}_{2}\theta_{2}\bar{\theta}_{1}\theta_{1} \dd{\theta_{1}} \dd{\bar{\theta}_{1}} \dd{\theta_{2}} \dd{\bar{\theta}_{2}} \dd{\theta}_{3} \dd{\bar{\theta}_{3}} = 1
    \end{equation*}

Portanto, $I(\vb{A})$ pode ser escrito como
    \begin{align*}
        I(\vb{A}) &\eq A_{11}A_{22}A_{33} - A_{11}A_{23}A_{32} - A_{12}A_{21}A_{33} + A_{12}A_{23}A_{31} + A_{13}A_{21}A_{32} - A_{13}A_{22}A_{31}
    \end{align*}
que é exatamente a definição de $\text{det}(\vb{A})$ para uma matriz $\vb{A}$ de $\text{dim}(\vb{A}) = 3$. Mostrando então que
    \begin{answer}\label{eq: answer 2 pt 1}
        I(\vb{A}) = \int \exp\qty(\sum_{j,k=1}^{3}\bar{\theta}_{j}A_{jk}\theta_{k})\prod_{i=1}^{3}\dd{\theta_{i}}\dd{\bar{\theta}_{i}} = \text{det}(\vb{A})
    \end{answer}

Para generalizar para um $N$ arbitrário, podemos reconsiderar a oredem de integração, isto é, podemos mudar 
    \begin{equation*}
        \prod_{i=1}^{N}\dd{\theta_{i}}\dd{\bar{\theta}_{i}} = (-1)^{\frac{N(N-1)}{2}}\prod_{i=1}^{N}\dd{\theta_{i}}\prod_{\ell=1}^{N}\dd{\bar{\theta}_{\ell}}
    \end{equation*}
onde o fator $(-1)^{\frac{N(N-1)}{2}}$ surge do número de trocas necessárias para colocar todas as variáveis $\theta$ juntas e todas as variáveis $\bar{\theta}$ juntas, pela propriedade anticomutativa dos números de Grassmann. O expoente pode ser interpretado como sendo feita  $N-1$ trocas em $N$ termos, porém isto consideraria uma troca tanto dos termos $\dd{\theta}$ quanto dos termos $\dd{\bar{\theta}}$, por isso dividimos por 2 para considerar apenas a troca de ordem de um dos conjuntos de termos. Com isso e considerando que ao expandirmos a exponencial em série de Taylor, o único termo que não será nulo é o termo de ordem $N$, temos
    \begin{align*}
        I(\vb{A}) &\eq \int \exp\qty(\sum_{j,k=1}^{N}\bar{\theta}_{j}A_{jk}\theta_{k})\prod_{i=1}^{N}\dd{\theta_{i}}\dd{\bar{\theta}_{i}} \\
        &\eq \dfrac{(-1)^{\frac{N(N-1)}{2}}}{N!}\int \qty(\sum_{j,k=1}^{N}\bar{\theta}_{j}A_{jk}\theta_{k})^{N}\prod_{i=1}^{N}\dd{\theta_{i}}\prod_{\ell=1}^{N}\dd{\bar{\theta}_{\ell}} \\
        &\eq \dfrac{(-1)^{\frac{N(N-1)}{2}}}{N!}\int 
            \bar{\theta}_{j_{1}}\theta_{k_{1}}
            \bar{\theta}_{j_{2}}\theta_{k_{2}} \cdots
            \bar{\theta}_{j_{N}}\theta_{k_{N}}
            A_{j_{1}k_{1}}A_{j_{2}k_{2}} \cdots A_{j_{N}k_{N}}
        \prod_{i=1}^{N}\dd{\theta_{i}}\prod_{\ell=1}^{N}\dd{\bar{\theta}_{\ell}}
    \end{align*}

Estamos considerando uma ordem de integração específica, então o argumento dentro da integral deve ser rearranjado para satisfazer esta ordem, então o que fazemos inicialmente é colocar todos os $\bar{\theta}_{i}$ à esquerda e todos os $\theta_{i}$ à direita, o que requer $N(N+1)/2$ trocas, ou seja
    \begin{equation*}
        \bar{\theta}_{j_{1}}\theta_{k_{1}}
        \bar{\theta}_{j_{2}}\theta_{k_{2}} \cdots
        \bar{\theta}_{j_{N}}\theta_{k_{N}} = (-1)^{\frac{N(N+1)}{2}}\bar{\theta}_{j_{1}}\bar{\theta}_{j_{2}} \cdots \bar{\theta}_{j_{N}}\theta_{k_{1}}\theta_{k_{2}} \cdots \theta_{k_{N}}
    \end{equation*}

Feita esta troca, precisamos nos ater às possíveis permutações entre os indices, o que sugere o uso de dois tensores de Levi-Civita, uma para considerar as permutações de $j_{i}$ e outro para as permutação de $k_{i}$, resultando então em
    \begin{align*}
        \bar{\theta}_{j_{1}} \theta_{k_{1}}
        \bar{\theta}_{j_{2}} \theta_{k_{2}} \cdots
        \bar{\theta}_{j_{N}} \theta_{k_{N}} &\eq (-1)^{\frac{N(N+1)}{2}}
        \epsilon_{j_{1}j_{2} \cdots j_{N}}
        \epsilon_{k_{1}k_{2} \cdots k_{N}}
        \bar{\theta}_{1} \bar{\theta}_{2} \cdots 
        \bar{\theta}_{N} \theta_{1} 
        \theta_{2} \cdots \theta_{N} \\
        &\eq (-1)^{\frac{N(N+1)}{2}} \epsilon_{j_{1}j_{2} \cdots j_{N}} \epsilon_{k_{1}k_{2} \cdots k_{N}} \prod_{j=1}^{N}\bar{\theta}_{j}\prod_{k=1}^{N}\theta_{k}
    \end{align*}

Então a integral que precisamos calcular é
    \begin{equation*}
        \int \prod_{j=1}^{N}\bar{\theta}_{j}\prod_{k=1}^{N}\theta_{k}\prod_{i=1}^{N}\dd{\theta_{i}}\prod_{\ell=1}^{N}\dd{\bar{\theta}_{\ell}}
    \end{equation*}

Que é facilmente determinada como sendo igual a 1. Um possível problema seria se perguntar se dependendo do valor de $N$ (sendo par ou ímpar), se ocorreria alguma troca de sinal, já que para realiar a integração na ordem proposta ainda é necessário inverter os índices do produtório em $k$ e do produtório em $k$, no entanto, se fizermos $N$ trocas com $\theta_{k}$, precisamos também fazer $N$ trocas com $\bar{\theta_{j}}$, o que resultaria em $2N$ operações, que se traduz em $(-1)^{2N} = 1,\ \forall\ n\in\mathbb{N}$. Portanto
    \begin{align*}
        I(A) &\eq \dfrac{(-1)^{\frac{N(N-1)}{2}}(-1)^{\frac{N(N+1)}{2}}}{N!}
            \epsilon_{j_{1}j_{2} \cdots j_{N}}
            \epsilon_{k_{1}k_{2} \cdots k_{N}}
            A_{j_{1}k_{1}}A_{j_{2}k_{2}} \cdots A_{j_{N}k_{N}} \\
        &\eq \dfrac{(-1)^{N^{2}}}{N!}
            \epsilon_{j_{1}j_{2} \cdots j_{N}}
            \epsilon_{k_{1}k_{2} \cdots k_{N}}
            A_{j_{1}k_{1}}A_{j_{2}k_{2}} \cdots A_{j_{N}k_{N}} \\
    \end{align*}

Como $N^{2}$ é par $\forall\ N\in\mathbb{N}$, então $(-1)^{N^{2}} = 1$, e portanto
    \begin{align*}
        I(A) &\eq \dfrac{1}{N!}
            \epsilon_{j_{1}j_{2} \cdots j_{N}}
            \epsilon_{k_{1}k_{2} \cdots k_{N}}
            A_{j_{1}k_{1}}A_{j_{2}k_{2}} \cdots A_{j_{N}k_{N}}
    \end{align*}

que é justamente a definição de determinante de uma matriz $\vb{A}$ de $\text{dim}(\vb{A}) = N$. Portanto, mostramos que
    \begin{answer}\label{eq: answer 2 pt 2}
        I(A) = \int \exp\qty(\sum_{j,k=1}^{N}\bar{\theta}_{j}A_{jk}\theta_{k})\prod_{i=1}^{N}\dd{\theta_{i}}\dd{\bar{\theta}_{i}} = \text{det}(A)
    \end{answer}

\chapter{Relações de campos de gauge não-abelianos}\label{quest: three}

\begin{exercise}{}
    No caso de um campo de gauge não-abeliano, mostre explicitamente que
    \begin{enumerate}[(a)]
        \item $[D_{\mu}, D_{\nu}]\Phi = -igF_{\mu\nu}\Phi$.
        \item $F_{\mu\nu}' \Phi' = UF_{\mu\nu}\Phi$.
    \end{enumerate}
    em que $\Phi$ é um vetor coluna de campos de bósons escalares que satisfaz a simetria de gauge do grupo, e $U = \exp[T_{a}\theta_{a}(x)]$, sendo $T_{a}$ os geradores do grupo.
\end{exercise}

\begin{itemize}
    \item Utilizar o gauge $D_{\mu}\Phi = \qty(\partial_{\mu} - ig\dfrac{\sigma_{a}{2}}A_{\mu}^{a})\Phi$ e lembrar que $F_{\mu\nu} = \partial_{\mu}A_{\nu} - \partial_{\nu}A_{\mu}$
    \item No item (b), preciso confirmar se $F_{\mu\nu}' = UF_{\mu\nu}$ e $\Phi' = U\Phi$ pra tentar abrir as contas
\end{itemize}

\chapter{Teoria \texorpdfstring{$\lambda\phi^{4}$}{}}\label{quest: four}

\begin{exercise}{}
    \begin{multicols}{2}
        Na teoria $\lambda\phi^{4}$ calcule, em regularização dimensional, com todos os detalhes necessários (incluindo a expansão em torno de $\epsilon\to0$, com $D = 4-2\epsilon$), a integral do diagrama de Feynman da figura ao lado.
        
        \begin{center}
            \begin{tikzpicture}
                \begin{feynman}
                    \vertex (i1);
                    \vertex [below right=1.5cm and 2cm of i1, dot] (i2) {};
                    \vertex [above right=1.5cm and 2cm of i2] (i3);
                    
                    \vertex [below=6cm of i1] (i4);
                    \vertex [above right=1.5cm and 2cm of i4, dot] (i5) {};
                    \vertex [below right=1.5cm and 2cm of i5] (i6);

                    \diagram* {
                        (i1) -- [plain, momentum={[arrow shorten=0.7]$p_{3}$}] (i2) -- [plain, rmomentum={[arrow shorten=0.7]$p_{4}$}] (i3),
                        (i2) -- [plain, half left, looseness=1.5] (i5),
                        (i5) -- [plain, half left, looseness=1.5] (i2),
                        (i4) -- [plain, rmomentum={[arrow shorten=0.7]$p_{1}$}] (i5) -- [plain, momentum={[arrow shorten=0.7]$p_{2}$}] (i6),
                    };
                \end{feynman}
            \end{tikzpicture}
        \end{center}
        
        
    \end{multicols}
\end{exercise}

\begin{itemize}
    \item Aula 6 + Aula 11, o diagrama é do canal-t, bem determinado então
    \item O diagrama depende dos momentos externos apenas através da variável t
\end{itemize}

\chapter{Regras de Feynman da QCD}\label{quest: five}

\begin{exercise}{}
    Utilizando as regras de Feynman da QCD, mostre que o diagrama de 1 loop (auto-energia) do propagador do quark difere do correspondente ao elétron na QED por um fator multiplicativo $C_{F} = \displaystyle\sum_{a}T_{a}^{2}$. Obs: não é necessário calcular a integração de loop.
\end{exercise}

\begin{itemize}
    \item Os diagramas de 1 loop do propagador do quark (esquerda) de do elétron (direita) são
        \begin{center}
            \begin{tikzpicture}
                \begin{feynman}
                    \vertex (i1);
                    \vertex [right=1cm of i1, dot] (i2) {};
                    \vertex [right=2cm of i2, dot] (i3) {};
                    \vertex [right=1cm of i3] (i4);

                    \diagram* {
                        (i1) -- (i2) -- [fermion] (i3) -- (i4),
                        (i2) -- [gluon, half left, looseness=1.5] (i3),
                    };
                \end{feynman}
            \end{tikzpicture}
            \hspace{2cm}
            \begin{tikzpicture}
                \begin{feynman}
                    \vertex (i1);
                    \vertex [right=1cm of i1, dot] (i2) {};
                    \vertex [right=2cm of i2, dot] (i3) {};
                    \vertex [right=1cm of i3] (i4);

                    \diagram* {
                        (i1) -- (i2) -- (i3) -- (i4),
                        (i2) -- [boson, half left, looseness=1.5] (i3),
                    };
                \end{feynman}
            \end{tikzpicture}
        \end{center}
    \item O fato de não precisar calcular a integração de loop indica que existe algum truque pra obter a resposta.
\end{itemize}

\chapter{Espalhamento elástico elétron-múon}\label{quest: six}

\begin{exercise}{}
    Calcule explicitamente a contração $L^{\alpha\beta}W_{\alpha\beta}$ no espalhamento elástico elétron-múon. Obs: despreze a massa do elétron comparada a outras escalas de energia, assumidas muito maiores.
\end{exercise}

\begin{itemize}
    \item Aparentemente é só abrir as contas com as expressões da aula 15
\end{itemize}

\chapter{Amplitude DVCS}\label{quest: seven}

\begin{exercise}{}
    Vimos que a amplitude DVCS (deeply-virtual Compton scattering) é dada por
        \begin{equation*}
            T_{\mu\nu} = i\int e^{iqz}\bra{p}T\{J_{\mu}(z)J_{\nu}(0)\}\ket{p}\dd[4]{z}
        \end{equation*}
    \begin{enumerate}[(a)]
        \item Partindo da expressão acima, mostre que $\text{Im}[T_{\alpha\beta}] = \pi W_{\alpha\beta}$, ou seja, que a parte imaginária da amplitude DVCS é proporcional ao tensor hadrônico do espalhamento inélástico profundo.
        \item Usando um conjunto completo de estados intermediários entre as correntes eletromagnéticas, argumente por quê
            \begin{equation*}
                \int e^{iqz} \bra{p}J_{\mu}(0)J_{\nu}(z)\ket{p}
            \end{equation*}
        é igual a zero. \textbf{Dica:} obtenha uma função delta de Dirac e conservação do quadri-momento total.
    \end{enumerate}
\end{exercise}



\noindent \textbf{(a)} Podemos representar as componentes de $\boldsymbol{\pi}$ por
    \begin{equation*}
        \pi_{1} = \dfrac{\pi^{+} + \pi^{-}}{\sqrt{2}} \qquad \& \qquad 
        \pi_{2} = \dfrac{i(\pi^{-} - \pi^{+})}{\sqrt{2}} \qquad \& \qquad 
        \pi_{3} = \pi_{0}
    \end{equation*}

Dessa forma, o campo $\Phi = \boldsymbol{\tau}\cdot\boldsymbol{\sigma}$ pode ser representado matricialmente
    \begin{align*}
        \Phi &\eq \tau_{1}\pi_{1} + \tau_{2}\pi_{2} + \tau_{3}\pi_{3} \\
        &\eq 
        \tau_{1} \qty(\dfrac{\pi^{+} + \pi^{-}}{\sqrt{2}}) + 
        \tau_{2}\qty[\dfrac{i(\pi^{-} - \pi^{+})}{\sqrt{2}}] + 
        \tau_{3} \pi_{0} \\
        &\eq 
        \begin{bmatrix}
            0 & 1 \\
            1 & 0
        \end{bmatrix} \qty(\dfrac{\pi^{+} + \pi^{-}}{\sqrt{2}}) +
        \begin{bmatrix}
            0 & -i \\
            i & 0
        \end{bmatrix} \qty[\dfrac{i(\pi^{-} - \pi^{+})}{\sqrt{2}}] + 
        \begin{bmatrix}
            1 & 0 \\
            0 & -1
        \end{bmatrix} \pi_{0} \\
        &\eq 
            \dfrac{1}{\sqrt{2}}
            \begin{bmatrix}
                0 & \pi^{+} + \pi^{-} \\
                \pi^{+} + \pi^{-} & 0
            \end{bmatrix} +
            \dfrac{1}{\sqrt{2}}
            \begin{bmatrix}
                0 & \pi^{-} - \pi^{+} \\
                -\pi^{-} + \pi^{+} & 0
            \end{bmatrix} +
            \begin{bmatrix}
                \pi_{0} & 0 \\
                0 & -\pi_{0}
            \end{bmatrix} \\
        &\eq 
        \begin{bmatrix}
            \pi_{0} & \sqrt{2} \pi^{-} \\
            \sqrt{2} \pi^{+} & -\pi_{0}
        \end{bmatrix} = 
        \begin{bmatrix}
        		\pi_{3} & \pi_{1} + i\pi_{2} \\
        		\pi_{1} - i\pi_{2} & \pi_{3}
        \end{bmatrix}
    \end{align*}

Portanto
    \begin{equation*}
        \Phi^{\dagger} = 
        \begin{bmatrix}
            \pi_{0}^{\ast} & \sqrt{2}(\pi^{+})^{\ast} \\
            \sqrt{2}(\pi^{-})^{\ast} & -\pi_{0}^{\ast}
        \end{bmatrix}
    \end{equation*}

Pela forma definida de $\pi^{\pm}$, é fácil ver que $(\pi^{-})^{\ast} = \pi^{+}$ e $(\pi^{+})^{\ast} = \pi^{-}$, logo
    \begin{equation*}
        \Phi^{\dagger} = 
        \begin{bmatrix}
            \pi_{0}^{\ast} & \sqrt{2}\pi^{-} \\
            \sqrt{2}\pi^{+} & -\pi_{0}^{\ast}
        \end{bmatrix}
    \end{equation*}

Segue que
    \begin{align*}
        \partial_{\mu}\Phi^{\dagger} \partial^{\mu}\Phi &\eq 
        \begin{bmatrix}
            \partial_{\mu}\pi_{0}^{\ast} & \sqrt{2}\partial_{\mu}\pi^{-} \\
            \sqrt{2}\partial_{\mu}\pi^{+} & -\partial_{\mu}\pi_{0}^{\ast}
        \end{bmatrix}
        \begin{bmatrix}
            \partial^{\mu}\pi_{0} & \sqrt{2}\partial^{\mu}\pi^{-} \\
            \sqrt{2}\partial^{\mu}\pi^{+} & -\partial^{\mu}\pi_{0}
        \end{bmatrix} \\
        &\eq
        \begin{bmatrix}
            \partial_{\mu}\pi_{0}^{\ast}\partial^{\mu}\pi_{0} + 2\partial_{\mu}\pi^{-}\partial^{\mu}\pi^{+} & \sqrt{2}\partial_{\mu}\pi_{0}^{\ast}\partial^{\mu}\pi^{-} - \sqrt{2}\partial_{\mu}\pi^{-}\partial^{\mu}\pi_{0} \\
            \sqrt{2}\partial_{\mu}\pi^{+}\partial^{\mu}\pi_{0} - \sqrt{2}\partial_{\mu}\pi_{0}^{\ast}\partial^{\mu}\pi^{2} & 2\partial_{\mu}\pi^{+}\partial^{\mu}\pi^{-} + \partial_{\mu}\pi_{0}^{\ast}\partial^{\mu}\pi_{0}
        \end{bmatrix}
    \end{align*}

Calculando o traço desta matriz:
    \begin{align*}
        \expval{\partial_{\mu}\Phi^{\dagger}\partial^{\mu}\Phi} &\eq \partial_{\mu}\pi_{0}^{\ast}\partial^{\mu}\pi_{0} + 
        2\partial_{\mu}\pi^{-}\partial^{\mu}\pi^{+} + 
        2\partial_{\mu}\pi^{+}\partial^{\mu}\pi^{-} + 
        \partial_{\mu}\pi_{0}^{\ast}\partial^{\mu}\pi_{0} \\
        &\eq 2\partial_{\mu}\pi_{0}^{\ast} \partial^{\mu}\pi_{0} + 
        2\partial_{\mu}\pi^{-}\partial^{\mu}\pi^{+} + 
        2\partial_{\mu}\pi^{+}\partial^{\mu}\pi^{-}
    \end{align*}

Fazendo o mesmo procedimento para $\Phi^{\dagger}\Phi$, temos
    \begin{align*}
        \Phi^{\dagger}\Phi &\eq 
        \begin{bmatrix}
            \pi_{0}^{\ast} & \sqrt{2}\pi^{-} \\
            \sqrt{2}\pi^{+} & -\pi_{0}^{\ast}
        \end{bmatrix}
        \begin{bmatrix}
            \pi_{0} & \sqrt{2} \pi^{-} \\
            \sqrt{2} \pi^{+} & -\pi_{0}
        \end{bmatrix} \\
        &\eq 
        \begin{bmatrix}
            \pi_{0}^{\ast} \pi_{0} + 2 \pi^{-} \pi^{+} & 
            \sqrt{2} \pi_{0}^{\ast} \pi^{-} - \sqrt{2} \pi^{-}\pi_{0} \\
            \sqrt{2} \pi^{+} \pi_{0} - \sqrt{2}\pi_{0}^{\ast} \pi^{+} & 
            2 \pi^{+} \pi^{-} + \pi_{0}^{\ast} \pi_{0}
        \end{bmatrix}
    \end{align*}

Então
    \begin{align*}
        \expval{\Phi^{\dagger}\Phi} &\eq \pi_{0}^{\ast}\pi_{0} + 2\pi^{-}\pi^{+} + 2\pi^{+}\pi^{-} + \pi_{0}^{\ast}\pi_{0} \\
        &\eq 2\pi_{0}^{\ast} \pi_{0} + 2\pi^{-}\pi^{+} + 2\pi^{+}\pi^{-}
    \end{align*}

A lagrangiana fica então
    \begin{align*}
        \mathcal{L} &\eq \dfrac{1}{4}\qty(
            2\partial_{\mu}\pi_{0}^{\ast} \partial^{\mu}\pi_{0} + 
            2\partial_{\mu}\pi^{-}\partial^{\mu}\pi^{+} + 
            2\partial_{\mu}\pi^{+}\partial^{\mu}\pi^{-}
        ) - \dfrac{m^2}{4}\qty(
            2\pi_{0}^{\ast} \pi_{0} + 2\pi^{-}\pi^{+} + 2\pi^{+}\pi^{-}
        ) \\
        &\eq \qty[
            \dfrac{1}{2}\partial_{\mu}\pi_{0}^{\ast} \partial^{\mu}\pi_{0} - 
            \dfrac{m^2}{2} \pi_{0}^{\ast} \pi_{0}
        ] + \qty[
            \qty(
                \dfrac{1}{2}\partial_{\mu}\pi^{-} \partial^{\mu}\pi^{+} +
                \dfrac{1}{2}\partial_{\mu}\pi^{+} \partial^{\mu}\pi^{-}
            ) -
            \dfrac{m^2}{2} 
            \qty( 
                \pi^{-}\pi^{+} +
                \pi^{+}\pi^{-}
            )
        ]
    \end{align*}

Sabendo que $\pi_{0}(x)$ é um campo real, temos que $\pi_{0}^{\ast} = \pi_{0}$, o que nos permite escrever o primeiro $[\cdots]$ da lagrangiana sob a forma de uma lagrangiana livre de um campo escalar neutro para $\pi_{0}$:
    \begin{answer}\label{eq: free lagrangian of a neutral scalar field}
        \mathcal{L}_{\pi_{0}} = \dfrac{1}{2}\partial_{\mu}\pi_{0}\partial^{\mu}\pi_{0} - \dfrac{1}{2}m^2\pi_{0}^2
    \end{answer}

No segundo $[\cdots]$, identificamos que $\partial_{\mu}\pi^{-}\partial^{\mu}\pi^{+} = \partial_{\mu}\pi^{+}\partial^{\mu}\pi^{-}$ e $\pi^{-}\pi^{+} = \pi^{+}\pi^{-}$, nos permitindo escrever uma lagrangiana de campo escalar carregado
    \begin{answer}\label{eq: free lagrangian of a charged scalar field}
        \mathcal{L}_{\pi^{\pm}} = \partial_{\mu}\pi^{+}\partial^{\mu}\pi^{-} - m^2 \pi^{+}\pi^{-}
    \end{answer}
	

\noindent \textbf{(b)} A lagrangiana original pode ser reescrita após calcularmos os traços. O primeiro traço da lagrangiana fica, lembrando que $\Phi^{\dagger} = \Phi = \boldsymbol{\tau}\cdot\boldsymbol{\pi}$:
    \begin{align*}
        \expval{\partial_{\mu}\Phi^{\dagger} \partial^{\mu}\Phi} &\eq
        \expval{\partial_{\mu}(\boldsymbol{\tau}\cdot\boldsymbol{\pi})\partial^{\mu}(\boldsymbol{\tau}\cdot\boldsymbol{\pi})} = 
        \expval{\sum_{a,b}\partial_{\mu}(\tau_{a}\pi_{a})\partial^{\mu}(\tau_{b}\pi_{b})} \\
        &\eq 
        \expval{\sum_{a,b}\tau_{a}\tau_{b}(\partial_{\mu}\pi_{a})(\partial^{\mu}\pi_{b})} =
        \sum_{a,b} \expval{\tau_{a}\tau_{b}} (\partial_{\mu}\pi_{a})(\partial^{\mu}\pi_{b})
    \end{align*}
As matrizes de Pauli satisfazem a propriedade $\expval{\tau_{a}\tau_{b}} = 2\delta_{ab}$, portanto
    \begin{equation*}
        \expval{\partial_{\mu}\Phi^{\dagger}\partial^{\mu}\Phi} = 
        2\sum_{a,b} \delta_{ab} (\partial_{\mu}\pi_{a})(\partial^{\mu}\pi_{b}) =
        2(\partial_{\mu}\pi_{a})(\partial^{\mu}\pi_{a}) = 2(\partial_{\mu}\boldsymbol{\pi})\cdot(\partial^{\mu}\boldsymbol{\pi})
    \end{equation*}


O segundo traço da lagrangiana fica
    \begin{align*}
        \expval{\Phi^{\dagger}\Phi} &\eq 
        \expval{(\boldsymbol{\tau}\cdot\boldsymbol{\pi})(\boldsymbol{\tau}\cdot\boldsymbol{\pi})} = 
        \expval{\sum_{a,b}\tau_{a}\pi_{a}\tau_{b}\pi_{b}} =
        \sum_{a,b}\pi_{a}\pi_{b}\expval{\tau_{a}\tau_{b}} \\
        &\eq
        2\sum_{a,b}\pi_{a}\pi_{b}\delta_{ab} = 
        2\pi_{a}\pi_{a} =
        2(\boldsymbol{\pi}\cdot\boldsymbol{\pi})
    \end{align*}

A lagrangiana na representação cartesiana pode ser escrita então por
    \begin{equation*}
        \mathcal{L} = \dfrac{1}{2}(\partial_{\mu}\boldsymbol{\pi})\cdot(\partial^{\mu}\boldsymbol{\pi}) - 
        \dfrac{m^2}{2}(\boldsymbol{\pi}\cdot\boldsymbol{\pi})
    \end{equation*}

Considerando uma transformação $U(\boldsymbol{\theta}) = \exp(-i\boldsymbol{\tau}\cdot\boldsymbol{\theta})$, com $\boldsymbol{\theta}$ um vetor de componentes infinitesimais no espaço de isospin, podemos aproximar
    \begin{equation*}
        U(\boldsymbol{\theta}) \approx \boldone_{2\times2} - i\boldsymbol{\tau}\cdot\boldsymbol{\theta}
    \end{equation*}

Sendo $\Phi$ uma matriz $2\times2$ hermitiana e de traço nulo, ele é um elemento da álgebra de Lie $\mathfrak{su}(2)$, cujos elementos se transformam como $A' = UAU^{-1}$, para uma transformação unitária $U$ em $\text{SU}(2)$, então como $U^{\dagger}(\boldsymbol{\theta}) = U^{-1}(\boldsymbol{\theta})$ e é um elemento de $\text{SU}(2)$, uma transformação equivalente à do enunciado que mantém o campo invariante é $\Phi' = U(\boldsymbol{\theta}) \Phi U^{\dagger}(\boldsymbol{\theta})$. Usar esta transformação ao invés de apenas $\Phi' = U(\boldsymbol{\theta})\Phi$ é possível, pois ambas vão gerar a mesma variação dos campos $\pi_{j}$ (basta utilizar a forma $\Phi = \boldsymbol{\tau}\cdot\boldsymbol{\pi}$ em $\delta\Phi$ e identificar a parte que representa a variação nos campos $\pi_{j}$) , a única diferença seria o fato de $U(\boldsymbol{\theta})\Phi$ não manter a hermiticidade e o traço nulo ao determinar $\delta\Phi$, logo, utilizo a transformação padrão de elementos de uma álgebra de Lie $\mathfrak{su}(2)$ apenas por buscar manter essas duas propriedades em $\delta\Phi$. Portanto
    \begin{align*}
        \Phi' = U(\boldsymbol{\theta})\Phi U^{\dagger}(\boldsymbol{\theta}) &\eq  
        \qty[\Phi - i(\boldsymbol{\tau}\cdot\boldsymbol{\theta})\Phi]
        \qty[\boldone_{2\times2} + i(\boldsymbol{\tau}\cdot\boldsymbol{\theta})] \\
        &\eq
        \Phi + i\Phi(\boldsymbol{\tau}\cdot\boldsymbol{\theta}) - 
        i(\boldsymbol{\tau}\cdot\boldsymbol{\theta})\Phi + 
        \mathcal{O}(\boldsymbol{\theta}^2) \\
        &\eq \Phi + 
        i[\Phi,(\boldsymbol{\tau}\cdot\boldsymbol{\theta})] + 
        \mathcal{O}(\boldsymbol{\theta}^2)
    \end{align*}

Ignorando os termos quadráticos em $\boldsymbol{\theta}$ por ele ter componentes infinitesimais, temos que a variação do campo nos dá
    \begin{equation*}
        \delta\Phi = i\sum_{a}[\Phi, \tau_{a}]\theta_{a} = i\sum_{a,b} [\tau_{b},\tau_{a}]\theta_{a}\pi_{b}
    \end{equation*}

As matrizes de Pauli por serem geradores de uma álgebra de Lie satisfazem a relação de comutação
    \begin{equation*}
        [\tau_{a},\tau_{b}] = \sum_{c}2i\epsilon_{abc}\tau_{c}
    \end{equation*}
ou seja
    \begin{equation*}
        \delta\Phi = -2\sum_{a,b,c} \epsilon_{bac}\tau_{c}\theta_{a}\pi_{b} = 2\sum_{a,b,c}\epsilon_{abc}\tau_{c}\theta_{a}\pi_{b}
    \end{equation*}

Podemos considerar que 
    \begin{equation*}
        \delta\Phi = \delta\qty(\sum_{c}\tau_{c}\pi_{c}) = \sum_{c}\tau_{c} \delta\pi_{c} = 2\sum_{a,b,c}\epsilon_{abc}\tau_{c}\theta_{a}\pi_{b}
    \end{equation*}

Então a variação do campo fica
    \begin{equation*}
        \delta \pi_{c} = 2\sum_{a,b} \epsilon_{abc}\theta_{a}\pi_{b}
    \end{equation*}

Para obter a corrente de Noether, consideramos a simetria global em $\theta_{k}$, de tal forma que para $k=1,2,3$, temos
    \begin{equation*}
        J_{k}^{\mu} = \sum_{c}\fdv{\mathcal{L}}{(\partial_{\mu}\pi_{c})} \fdv{\pi_{c}}{\theta_{k}}
    \end{equation*}

O segundo termo é facilmente obtido, tal que
    \begin{equation*}
        \fdv{\pi_{c}}{\theta_{k}} = 2\sum_{a,b} \epsilon_{abc}\delta_{ak}\pi_{b} = 2\sum_{b} \epsilon_{kbc} \pi_{b}
    \end{equation*}

Já o primeiro
    \begin{align*}
        \fdv{\mathcal{L}}{(\partial_{\mu}\pi_{c})} &\eq \dfrac{1}{2}\fdv{}{(\partial_{\mu}\pi_{c})}\qty[
            \sum_{d}\partial_{\nu}\pi_{d}\partial^{\nu}\pi_{d}
        ] = \dfrac{1}{2}\sum_{d}\qty[
            \fdv{(\partial_{\nu}\pi_{d})}{(\partial_{\mu}\pi_{c})} \partial^{\nu}\pi_{d} +
            \partial_{\nu}\pi_{d} g^{\nu\beta} \fdv{(\partial_{\beta}\pi_{d})}{(\partial_{\mu}\pi_{c})}
        ] \\
        &\eq \dfrac{1}{2}\sum_{d}\qty[
            \delta_{\nu}^{\mu}\delta_{cd} \partial^{\nu}\pi_{d} + 
            \partial_{\nu}\pi_{d} g^{\nu\beta} \partial_{\beta}^{\mu}\delta_{cd}
        ] = 
        \dfrac{1}{2}\sum_{d}\qty[
            \delta_{cd}\partial^{\mu}\pi_{d} + 
            \partial^{\mu}\pi_{d}\delta_{cd}
        ] \\
        &\eq \dfrac{1}{2}\qty[\partial^{\mu}\pi_{c} + \partial^{\mu}\pi_{c}] = \partial^{\mu}\pi_{c}
    \end{align*}

Logo
    \begin{equation*}
        J_{k}^{\mu} = 2\sum_{c}\partial^{\mu}\pi_{c}\sum_{b}\epsilon_{kbc}\pi_{b} = 2\sum_{b,c} \epsilon_{kbc}\pi_{b}\partial^{\mu}\pi_{c}
    \end{equation*}

Juntando todas as componentes $k$, concluímos que a corrente de Noether é
    \begin{answer}\label{eq: Noether current for a simple sigma model}
        J^{\mu} = 2\boldsymbol{\pi}\times\partial^{\mu}\boldsymbol{\pi}
    \end{answer}

\chapter{Cálculo de diagramas de Feynman}\label{quest: eight}

\begin{exercise}{}
    Utilizando as regras de Feynman da QCD (aula 13), calcule os diagramas de Feynman abaixo, que contribuem para o processo $e^{+}(k_{1})e^{-}(k_{2}) \to \bar{q}(p_{1})q(p_{2})$. No segundo diagrama, o glúon emitido possui momento muito baixo (\textit{soft glúon}). Mostre que os dois diagramas são infinitos, explique a origem desses infinitos, e como lidar com eles para obter um resultado físico consistente.

    \noindent\begin{tikzpicture}
        \begin{feynman}
            \vertex (i1) {$e^{+}$};
            \vertex [below right=1.5cm and 2cm of i1] (i2);
            \vertex [below left=1.25cm and 1.75cm of i2] (i3) {$e^{-}$};

            \vertex [right=6cm of i1] (i4) {$\bar{q}$};
            \vertex [below left=1.5cm and 2cm of i4] (i5);
            \vertex [below right=1.25cm and 1.75cm of i5] (i6) {$q$};

            \vertex [below left=0.375cm and 0.5cm of i4] (g1);
            \vertex [below=2.25cm of g1] (g2);

            \diagram*{
                (i3) -- [fermion] (i2) -- [fermion] (i1),
                (i2) -- [boson] (i5),
                (i4) -- [fermion] (i5) -- [fermion] (i6),
                (g1) -- [gluon, half left, looseness=1] (g2)
            };

        \end{feynman}
    \end{tikzpicture}
    \hfill
    \begin{tikzpicture}
        \begin{feynman}
            \vertex (i1) {$e^{+}$};
            \vertex [below right=1.5cm and 2cm of i1] (i2);
            \vertex [below left=1.25cm and 1.75cm of i2] (i3) {$e^{-}$};

            \vertex [right=6cm of i1] (i4) {$\bar{q}$};
            \vertex [below left=1.5cm and 2cm of i4] (i5);
            \vertex [below right=1.25cm and 1.75cm of i5] (i6) {$q$};

            \vertex [below left=0.375cm and 0.5cm of i4] (g1);
            \vertex [below right=0.5cm and 0.5cm of g1] (g2);

            \diagram*{
                (i3) -- [fermion] (i2) -- [fermion] (i1),
                (i2) -- [boson] (i5),
                (i4) -- [fermion] (i5) -- [fermion] (i6),
                (g1) -- [gluon] (g2)
            };

        \end{feynman}
    \end{tikzpicture}
\end{exercise}

% \makefinal

\end{document}
