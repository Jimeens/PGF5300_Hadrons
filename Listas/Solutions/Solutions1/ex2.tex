

Antes de determinar os operadores de criação e aniquilação, podemos calcular o momento canonicamente conjugado $\pi_{0}(x)$ do campo livre $\phi_{0}(x)$:
    \begin{equation*}
        \pi_{0} = \fdv{\mathcal{L}_{0}}{\dot{\phi_{0}}} = \fdv{}{\dot{\phi_{0}}} \qty(\dfrac{1}{2}\partial_{\mu}\phi_{0}\partial^{\mu}\phi_{0} - \dfrac{1}{2}m^2\phi_{0}^2) = \dfrac{1}{2}\fdv{}{\dot{\phi_{0}}} \qty[
            \dot{\phi}_{0}^2 - \nabla\phi_{0}\cdot\nabla\phi_{0} - m^2\phi_{0}^2
        ] = \dot{\phi}_{0}
    \end{equation*}

Portanto
    \begin{equation*}
        \pi_{0}(x) = \int \dfrac{1}{(2\pi)^3} \qty[
            a_{0}(\vb{k}) \partial_{0}f_{k}(x) +
            a_{0}^{\dagger}(\vb{k}) \partial_{0} f_{k}^{\ast}(x)
        ]\dd[3]{k}
    \end{equation*}
onde
    \begin{align}
        \pdv{f_{k}(x)}{t} &= \sqrt{2\omega_{\vb{k}}}\pdv{}{t}\qty(e^{-i\omega_{\vb{k}}t})e^{i\vb{k}\cdot\vb{x}} = -i\omega_{\vb{k}} \sqrt{2\omega_{\vb{k}}}e^{-i\omega_{\vb{k}}t+i\vb{k}\cdot\vb{x}} = -i\omega_{\vb{k}}f_{k}(x) \label{eq: derivative of f}\\
        \pdv{f_{k}^{\ast}(x)}{t} &= \sqrt{2\omega_{\vb{k}}}\pdv{}{t}\qty(e^{+i\omega_{\vb{k}}t})e^{-i\vb{k}\cdot\vb{x}} = +i\omega_{\vb{k}} \sqrt{2\omega_{\vb{k}}}e^{i\omega_{\vb{k}}t-i\vb{k}\cdot\vb{x}} = +i\omega_{\vb{k}}f_{k}^{\ast}(x) \label{eq: derivative of f*}
    \end{align}

Concluindo que
    \begin{equation*}
        \pi_{0}(x) = \int \dfrac{i\omega_{\vb{k}}}{(2\pi)^3} \qty[
            -a_{0}(\vb{k})f_{k}(x) +
            a_{0}^{\dagger}(\vb{k})f_{k}^{\ast}(x)
        ]\dd[3]{k}
    \end{equation*}

\noindent \textbf{(a)} Para determinar os operadores, podemos pensar em escrevê-los em função dos campos livres $\phi_{0}(x)$ e $\pi_{0}(x)$, onde para isso fazemos uma transformada de Fourier inversa no espaço nestes campos. No caso do campo $\phi_{0}(x)$, temos
    \begin{equation*}
        \int \phi_{0}(x)e^{-i\vb{k}\cdot\vb{x}}\dd[3]{x} = \int \qty{\int \dfrac{1}{(2\pi)^3}\qty[
            a_{0}(\vb{k}')f_{k'}(x) + 
            a_{0}^{\dagger}(\vb{k}')f_{k'}^{\ast}(x)
        ]\dd[3]{k'}}e^{-i\vb{k}\cdot\vb{x}}\dd[3]{x}
    \end{equation*}

Expandindo $f_{k}(x)$, a expressão fica 
    \begin{align*}
        \int \phi_{0}(x) e^{-i\vb{k}\cdot\vb{x}}\dd[3]{x} 
        &\eq \iint \dfrac{1}{(2\pi)^3 \sqrt{2\omega_{\vb{k}'}}} \qty[
            a_{0}(\vb{k}') e^{-i\omega_{\vb{k}'}t + i\vb{k}'\cdot\vb{x}} + 
            a_{0}^{\dagger}(\vb{k}') e^{i\omega_{\vb{k}'}t - i\vb{k}\cdot\vb{x}}
        ] e^{-i\vb{k}\cdot\vb{x}} \dd[3]{k'}\dd[3]{x} \\
        &\eq \iint \dfrac{e^{-i\omega_{\vb{k}'}t}}{(2\pi)^3\sqrt{2\omega_{\vb{k}'}}} \qty[
            a_{0}(\vb{k}') e^{i(\vb{k}' - \vb{k})\cdot \vb{x}} + 
            a_{0}^{\dagger}(\vb{k}') e^{2i\omega_{\vb{k}'}t} e^{-i(\vb{k}' + \vb{k})\cdot \vb{x}}
        ] \dd[3]{x} \dd[3]{k'}
    \end{align*}

Podemos separar as integrais em $\dd[3]{x}$, de modo que
    \begin{equation*}
        \int \phi_{0}(x)e^{-i\vb{k}\cdot\vb{x}}\dd[3]{x} = \int \dfrac{e^{-i\omega_{\vb{k}'}t}}{(2\pi)^3 \sqrt{2\omega_{\vb{k}'}}} \qty[
            a_{0}(\vb{k}') \int e^{i(\vb{k}' - \vb{k})\cdot\vb{x}}\dd[3]{x} + 
            a_{0}^{\dagger}(\vb{k}') e^{2i\omega_{\vb{k}'}t} \int e^{-i(\vb{k}'+\vb{k})\cdot\vb{x}}\dd[3]{x}
        ]\dd[3]{k'}
    \end{equation*}

Note então que as integrais em $\dd[3]{x}$ são, juntamente com o fator $1/(2\pi)^3$, distribuições delta de Dirac em 3 dimensões, tal que
    \begin{align*}
        \dfrac{1}{(2\pi)^3}\int e^{\pm i(\vb{k}' - \vb{k})\cdot\vb{x}}\dd[3]{x} &= \delta^3(\vb{k}' - \vb{k}) \\
        \dfrac{1}{(2\pi)^3}\int e^{\pm i(\vb{k}' + \vb{k})\cdot\vb{x}} \dd[3]{x} &= \delta^3(\vb{k}' + \vb{k})
    \end{align*}

Sendo assim, obtemos
    \begin{equation*}
        \int \phi_{0}(x)e^{-i\vb{k}\cdot\vb{x}}\dd[3]{x} = \int \dfrac{e^{-i\omega_{\vb{k}'}t}}{\sqrt{2\omega_{\vb{k}'}}} \qty[
            a_{0}(\vb{k}') \delta^3(\vb{k}' - \vb{k}) + 
            a_{0}^{\dagger}(\vb{k}') e^{2i\omega_{\vb{k}'}t} \delta^3(\vb{k}' + \vb{k})
        ]\dd[3]{k'}
    \end{equation*}

Podemos notar também que $\omega_{-\vb{k}} = \sqrt{\abs{-\vb{k}}^2 + m^2} = \omega_{\vb{k}}$, logo ao performar a integral obtemos
    \begin{align*}
        \int \phi_{0}(x) e^{-i\vb{k}\cdot\vb{x}}\dd[3]{x} &\eq \dfrac{e^{-i\omega_{\vb{k}}t}}{\sqrt{2\omega_{\vb{k}}}}\qty[
            a_{0}(\vb{k}) +
            a_{0}^{\dagger}(-\vb{k}) e^{2i\omega_{-\vb{k}}t}
        ]\\ 
        &\eq \dfrac{e^{-i\omega_{\vb{k}}t}}{\sqrt{2\omega_{\vb{k}}}} \qty[
            a_{0}(\vb{k}) + 
            a_{0}^{\dagger}(\vb{k}) e^{2i\omega_{\vb{k}}t}
        ]
    \end{align*}

No caso do momento conjugado, temos
    \begin{align*}
        \int \pi_{0}(x) e^{-i\vb{k}\cdot\vb{x}}\dd[3]{x} &\eq \int \qty{
            \int \dfrac{i\omega_{\vb{k}'}}{(2\pi)^3} \qty[
                -a_{0}(\vb{k}')f_{k}(x) +
                a_{0}^{\dagger}(\vb{k}')f_{k'}^{\ast}(x)
            ]\dd[3]{k'}
        }e^{-i\vb{k}\cdot\vb{x}}\dd[3]{x} \\
        &\eq \iint \dfrac{i\omega_{\vb{k}'} e^{-i\omega_{\vb{k}'}t}}{(2\pi)^3\sqrt{2\omega_{\vb{k}'}}}\qty[
            -a_{0}(\vb{k}') e^{i(\vb{k}'-\vb{k})\cdot\vb{x}} +
            a_{0}^{\dagger}(\vb{k}') e^{2i\omega_{\vb{k}'}t} e^{-i(\vb{k}'+\vb{k})\cdot\vb{x}}
        ]\dd[3]{x}\dd[3]{k'} \\
        &\eq -i\int \sqrt{\dfrac{\omega_{\vb{k}'}}{2}} e^{-i\omega_{\vb{k}'}t} \qty[
            a_{0}(\vb{k}') \delta^3(\vb{k}' - \vb{k}) -
            a_{0}^{\dagger}(\vb{k}') e^{2i\omega_{\vb{k}'}t} \delta^3(\vb{k}' + \vb{k})
        ] \dd[3]{k'} \\
        &\eq -i\sqrt{\dfrac{\omega_{\vb{k}}}{2}} e^{-i\omega_{\vb{k}}t} \qty[
            a_{0}(\vb{k}) -
            a_{0}^{\dagger}(-\vb{k}) e^{2i\omega_{-\vb{k}}t}
        ] \\
        &\eq -i\sqrt{\dfrac{\omega_{\vb{k}}}{2}} e^{-i\omega_{\vb{k}}t} \qty[
            a_{0}(\vb{k}) - 
            a_{0}^{\dagger}(-\vb{k})e^{2i\omega_{\vb{k}}t}
        ]
    \end{align*}

Com estes resultados, podemos obter as formas
    \begin{equation*}
        \sqrt{2\omega_{\vb{k}}} e^{i\omega_{\vb{k}}t} \int \phi_{0}(x) e^{-i\vb{k}\cdot\vb{x}}\dd[3]{x} = a_{0}(\vb{k}) + a_{0}^{\dagger}(-\vb{k}) e^{2i\omega_{\vb{k}}t}
    \end{equation*}
    \begin{equation*}
        i\sqrt{\dfrac{2}{\omega_{\vb{k}}}} e^{i\omega_{\vb{k}}t} \int \pi_{0}(x) e^{-i\vb{k}\cdot\vb{x}}\dd[3]{x} = a_{0}(\vb{k}) - a_{0}^{\dagger}(-\vb{k}) e^{2i\omega_{\vb{k}}t}
    \end{equation*}

A exponencial $e^{i\omega_{\vb{k}}t}$ pode se juntar à $e^{-i\vb{k}\cdot\vb{x}}$ sob a forma $e^{ikx}$, pois a integral está sendo feita apenas em $\dd[3]{x}$, de modo que as equações ficam
    \begin{equation*}
        \sqrt{2\omega_{\vb{k}}} \int \phi_{0}(x) e^{ikx}\dd[3]{x} = a_{0}(\vb{k}) + a_{0}^{\dagger}(-\vb{k}) e^{2i\omega_{\vb{k}}t}
    \end{equation*}
    \begin{equation*}
        i\sqrt{\dfrac{2}{\omega_{\vb{k}}}} \int \pi_{0}(x) e^{ikx}\dd[3]{x} = a_{0}(\vb{k}) - a_{0}^{\dagger}(-\vb{k}) e^{2i\omega_{\vb{k}}t}
    \end{equation*}

Somando as duas equações:
    \begin{align*}
        2a_{0}(\vb{k}) 
        &= i\sqrt{\dfrac{2}{\omega_{\vb{k}}}} 
            \int \pi_{0}(x) e^{ikx} \dd[3]{x} +
        \sqrt{2\omega_{\vb{k}}} 
            \int \phi_{0}(x) e^{ikx} \dd[3]{x}
    \end{align*}

Temos então
    \begin{align*}
        a_{0}(\vb{k}) 
        &= \dfrac{i}{\sqrt{2\omega_{\vb{k}}}} \int \pi_{0}(x) e^{ikx} \dd[3]{x} + 
        \sqrt{\dfrac{\omega_{\vb{k}}}{2}} \int \phi_{0}(x) e^{ikx} \dd[3]{x} 
        \\
        &= \dfrac{i}{\sqrt{2\omega_{\vb{k}}}} \int \partial_{0}\phi_{0}(x) e^{ikx} \dd[3]{x} +
        \dfrac{\omega_{\vb{k}}}{\sqrt{2\omega_{\vb{k}}}} \int \phi_{0}(x) e^{ikx} \dd[3]{x} 
        \\
        &= i \int \partial_{0}\phi_{0}(x) f_{k}^{\ast}(x) \dd[3]{x} +
        \omega_{\vb{k}} \int \phi_{0}(x) f_{k}^{\ast}(x) \dd[3]{x} 
    \end{align*}

Usando a \eqref{eq: derivative of f*},
    \begin{equation*}
        -i\partial_{0}f_{k}^{\ast}(x) = \omega_{\vb{k}}f_{k}^{\ast}(x)
    \end{equation*}

O que modifica a forma do operador na forma
    \begin{align*}
        a_{0}(\vb{k}) &\eq i\int \partial_{0}\phi_{0}(x) f_{k}^{\ast}(x) \dd[3]{x} - i\int \phi_{0}(x) \partial_{0}f_{k}^{\ast}(x) \dd[3]{x} \\ 
        &\eq i\int \Big[f_{k}^{\ast}(x) \partial_{0}\phi_{0}(x) - \partial_{0}f_{k}^{\ast}(x) \phi_{0}(x) \Big] \dd[3]{x}
    \end{align*}

Pela notação
    \begin{equation*}
        A \overset{\leftrightarrow}{\partial_0} B = 
        A (\partial_{0}B) - (\partial_{0}A) B
    \end{equation*}

Podemos concluir que o operador de aniquilação $a_{0}(\vb{k})$ admite ser escrito sob a forma
    \begin{answer}\label{eq: annihilation operator}
        a_{0}(\vb{k}) = i\int f_{k}^{\ast}(x) \overset{\leftrightarrow}{\partial_{0}}\phi_{0}(x) \dd[3]{x}
    \end{answer}

Como estamos considerando um campo $\phi_{0}(x)$ real, temos $\phi^{\dagger}_{0}(x) = \phi_{0}(x)$, portanto ao calcular $a_{0}^{\dagger}(\vb{k})$, trocaremos $i \mapsto -i$ e $f_{k}^{\ast}(x) \mapsto f_{k}(x)$, concluindo que
    \begin{answer}\label{eq: creation operator}
        a_{0}^{\dagger}(\vb{k}) = -i\int f_{k}(x) \overset{\leftrightarrow}{\partial_{0}}\phi(x)\dd[3]{x}
    \end{answer}

\noindent \textbf{(b)} Calculando a derivada de $a_{0}(\vb{k})$ em relação ao tempo, temos que como a integral é feita apenas nas coordenadas espaciais, a derivação pode comutar com a integral, de modo que
    \begin{align*}
        \partial_{0}a_{0}(\vb{k}) &\eq i\int \partial_{0}\qty[
            f_{k}^{\ast}(x) \overset{\leftrightarrow}{\partial_{0}} \phi_{0}(x)
        ] \dd[3]{x} \\
        &\eq i\int \partial_{0}\Big[
            f_{k}^{\ast}(x) \partial_{0}\phi_{0}(x) -
            \partial_{0}f_{k}^{\ast}(x) \phi_{0}(x)
        \Big] \dd[3]{x} \\
        &\eq i\int \Big[
            \partial_{0}f_{k}^{\ast}(x) \partial_{0}\phi_{0}(x) +
            f_{k}^{\ast}(x) \partial_{0}^{2}\phi_{0}(x) - 
            \partial_{0}^2f_{k}^{\ast}(x) \phi_{0}(x) -
            \partial_{0}f_{k}^{\ast}(x) \partial_{0}\phi_{0}(x)
        \Big] \dd[3]{x} \\
        &\eq i\int \Big[
            f_{k}^{\ast}(x) \partial_{0}^2\phi_{0}(x) - 
            \partial_{0}^2f_{k}^{\ast}(x) \phi_{0}(x)
        \Big] \dd[3]{x}
    \end{align*}

A derivada temporal de segunda ordem em $f_{k}^{\ast}$ explicitamente fica
    \begin{equation}\label{eq: second derivative of f*}
        \partial_{0}^2 f_{k}^{\ast}(x) = \dfrac{1}{\sqrt{2\omega_{\vb{k}}}} \partial_{0}^2 e^{ikx} = \dfrac{1}{\sqrt{2\omega_{\vb{k}}}} (i\omega_{\vb{k}})^2 e^{ikx} = -\omega_{\vb{k}}^2 f_{k}^{\ast}(x)
    \end{equation}
    
ou seja
    \begin{equation*}
        \partial_{0} a_{0}(\vb{k}) = i \int 
            f_{k}^{\ast}(x) \qty(\partial_{0}^2 + \omega_{\vb{k}}^2) \phi_{0}(x)
        \dd[3]{x}
    \end{equation*}

Como $\omega_{\vb{k}}^2 = \abs{\vb{k}}^2 + m^2$, reescrevemos
    \begin{equation*}
        \partial_{0} a_{0}(\vb{k}) = i \int 
            f_{k}^{\ast}(x) \qty(\partial_{0}^2 + \abs{\vb{k}}^2 + m^2) \phi_{0}(x)
        \dd[3]{x}
    \end{equation*}

Note que
    \begin{align*}
        \nabla \cdot \qty[f_{k}^{\ast}(x) \overset{\leftrightarrow}{\nabla} \phi_{0}(x)] &\eq \nabla\cdot\Big[
            f_{k}^{\ast}(x) \nabla \phi_{0}(x) - 
            \nabla f_{k}^{\ast}(x) \phi_{0}(x)
        \Big] \\
        &\eq \nabla f_{k}^{\ast}(x) \cdot \nabla\phi_{0}(x) + 
            f_{k}^{\ast}(x) \nabla^2\phi_{0}(x) - 
            \nabla^2f_{k}^{\ast}(x) \phi_{0}(x) -
            \nabla f_{k}^{\ast}(x) \cdot \nabla\phi_{0}(x) \\
        &\eq f_{k}^{\ast}(x) \nabla^2\phi_{0}(x) - \nabla^2f_{k}^{\ast}(x) \phi_{0}(x)
    \end{align*}
onde 
    \begin{equation*}
        \nabla^2 f_{k}^{\ast}(x) = 
        \dfrac{1}{\sqrt{2\omega_{\vb{k}}}} \nabla^2e^{ikx} = 
        \dfrac{1}{\sqrt{2\omega_{\vb{k}}}} (-i\vb{k})\cdot(-i\vb{k}) e^{ikx} =
        -\abs{\vb{k}}^2 f_{k}^{\ast}(x)
    \end{equation*}

Portanto
    \begin{equation*}
        \nabla \cdot \qty[f_{k}^{\ast}(x) \overset{\leftrightarrow}{\nabla} \phi_{0}(x)] = f_{k}^{\ast}(x) \nabla^2\phi_{0}(x) + \abs{\vb{k}}^2 f_{k}^{\ast}(x)\phi_{0}(x)
    \end{equation*}

Como o lado esquerdo desta equação é uma derivada total no espaço, temos que dentro de uma integral em $\dd[3]{x}$ esse termo vai pra zero pela hipótese do campo desaparecer nos limites asssintóticos, de modo que podemos fazer a substituição
    \begin{equation*}
        \abs{\vb{k}}^2 f_{k}^{\ast}(x)\phi_{0}(x) \mapsto -f_{k}^{\ast}(x) \nabla^2\phi_{0}(x)
    \end{equation*}
dentro da integral em $\partial_{0}a_{0}(\vb{k})$, logo
    \begin{align*}
        \partial_{0}a_{0}(\vb{k}) &\eq i\int 
            f_{k}^{\ast}(x) 
            \qty(
                \partial_{0}^2 -
                \nabla^2 +
                m^2
            )
            \phi_{0}(x) 
        \dd[3]{x} \\
        &\eq i\int 
            f_{k}^{\ast}(x)
            \qty(
                \partial_{\mu}\partial^{\mu} + 
                m^2
            )
            \phi_{0}(x)
        \dd[3]{x}
    \end{align*}

E como $\phi_{0}(x)$ é um campo livre, equação de Klein-Gordon é satisfeita, $(\partial_{\mu}\partial^{\mu} + m^2)\phi_{0}(x) = 0$, concluindo que
    \begin{answer}\label{eq: creation operator is time-independent}
        \partial_{0}a_{0}(\vb{k}) = 0 \Rightarrow a_{0}(\vb{k})~\text{é independente do tempo}
    \end{answer}

O raciocínio para $a_{0}^{\dagger}(\vb{k})$ é idêntico, de modo que a alteração pode ser visualizada por
    \begin{align*}
        \partial_{0} a_{0}(\vb{k}) &\mapsto \partial_{0} a_{0}^{\dagger}(\vb{k}) \\
        i\int 
            f_{k}^{\ast}(x)
            \qty(
                \partial_{\mu}\partial^{\mu} + 
                m^2
            )
            \phi_{0}(x)
        \dd[3]{x} &\mapsto 
        -i\int 
            f_{k}(x)
            \qty(
                \partial_{\mu}\partial^{\mu} + 
                m^2
            )
            \phi_{0}(x)
        \dd[3]{x}
    \end{align*}

Portanto 
    \begin{answer}\label{eq: annihilation operator is time-independent}
        \partial_{0}a_{0}^{\dagger}(\vb{k}) = 0 \Rightarrow a_{0}^{\dagger}(\vb{k})~\text{é independente do tempo}
    \end{answer}