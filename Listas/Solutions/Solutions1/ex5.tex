

\noindent \textbf{(a)} Expandindo o produto $\boldsymbol{\sigma}\cdot\vb{p}$, temos
    \begin{equation*}
        (\boldsymbol{\sigma}\cdot\vb{p}) = \sigma_{1}p_{1} + \sigma_{2}p_{2} + \sigma_{3}p_{3} = 
        \begin{bmatrix}
            p_{3}           &   p_{1} - ip_{2} \\
            p_{1} + ip_{2}  &   -p_{3}
        \end{bmatrix}
    \end{equation*}

O que nos dá também
    \begin{equation*}
        (\boldsymbol{\sigma}\cdot\vb{p})^{\dagger} = 
        \begin{bmatrix}
            p_{3} & p_{1} - ip_{2} \\
            p_{1} + ip_{2} & -p_{3}
        \end{bmatrix} = (\boldsymbol{\sigma}\cdot\vb{p})
    \end{equation*}

A quantidade $\chi_{s}$ é um vetor coluna tal que $\chi_{1} = \begin{bmatrix} 1 \\ 0 \end{bmatrix}$ e $\chi_{2} = \begin{bmatrix} 0 \\ 1 \end{bmatrix}$. Com isso, impomos que o produto $\chi_{s}^{\dagger}\chi_{s'} = \delta_{ss'}$. A forma explicita do espinor $\Bar{u}(\vb{p},s)$ é
    \begin{equation*}
        \bar{u}(\vb{p},s) = u^{\dagger}(\vb{p},s) \gamma^{0} = \dfrac{1}{\sqrt{E_{\vb{p}} + m}}
        \begin{bmatrix}
            (E_{\vb{p}} + m) \chi^{\dagger}_{s} & -\chi^{\dagger}_{s}(\boldsymbol{\sigma}\cdot\vb{p})^{\dagger}
        \end{bmatrix}
    \end{equation*}

Portanto, 
    \begin{align*}
        \bar{u}(\vb{p},s)u(\vb{p},s') &= \dfrac{1}{E_{\vb{p}}+m} \qty[
            (E_{\vb{p}} + m)^2 \chi^{\dagger}_{s}\chi_{s} - \chi^{\dagger}_{s}(\boldsymbol{\sigma}\cdot\vb{p})^{\dagger}(\boldsymbol{\sigma}\cdot\vb{p})\chi_{s}
        ]
    \end{align*}

O produto
    \begin{align*}
        (\boldsymbol{\sigma}\cdot\vb{p})^{\dagger}(\boldsymbol{\sigma}\cdot\vb{p}) &\eq 
        \begin{bmatrix}
            p_{3} & p_{1} - ip_{2} \\
            p_{1} + ip_{2} & -p_{3}
        \end{bmatrix}
        \begin{bmatrix}
            p_{3} & p_{1} - ip_{2} \\
            p_{1} + ip_{2} & -p_{3}
        \end{bmatrix} \\
        &\eq \begin{bmatrix}
            p_{1}^2 + p_{2}^2 + p_{3}^2 & 0 \\
            0 & p_{1}^2 + p_{2}^2 + p_{3}^2
        \end{bmatrix}\\
        &\eq \begin{bmatrix}
            \abs{\vb{p}}^2 & 0 \\
            0 & \abs{\vb{p}}^2
        \end{bmatrix} \\ 
        &\eq \abs{\vb{p}}^2 \boldone_{2\times2}
    \end{align*}

E como $E_{\vb{p}}^2 = \abs{\vb{p}}^2 + m^2$, temos
    \begin{equation*}
        (\boldsymbol{\sigma}\cdot\vb{p})^{\dagger}(\boldsymbol{\sigma}\cdot\vb{p}) = 
        (E_{\vb{p}}^2 - m^2)\boldone_{2\times2} = (E_{\vb{p}} + m)(E_{\vb{p}} - m)\boldone_{2\times2}
    \end{equation*}

Segue que 
    \begin{align*}
        \bar{u}(\vb{p},s)u(\vb{p},s') &\eq \dfrac{1}{E_{\vb{p}}+m}\qty[
            (E_{\vb{p}}+m)^2 \delta_{ss'} - (E_{\vb{p}}+m)(E_{\vb{p}}-m)\chi^{\dagger}_{s}\boldone_{2\times2}\chi_{s}
        ] \\ 
        &\eq (E_{\vb{p}} + m)\delta_{ss'} - (E_{\vb{p}} - m)\delta_{ss'} \\
        &\eq 2m\delta_{ss'}
    \end{align*}

Concluindo que
    \begin{answer}\label{eq: spinor relation for particles}
        \bar{u}(\vb{p},s)u(\vb{p},s) = 2m
    \end{answer}

Agora a forma explicita de $\bar{v}(\vb{p},s)$ é
    \begin{equation*}
        \bar{v}(\vb{p},s) = v^{\dagger}(\vb{p},s)\gamma^{0} = \dfrac{1}{\sqrt{E_{\vb{p}}+m}}
        \begin{bmatrix}
            \chi^{\dagger}_{s}(\boldsymbol{\sigma}\cdot\vb{p})^{\dagger} & 
            -(E_{\vb{p}}+m)\chi_{s}^{\dagger}
        \end{bmatrix}
    \end{equation*}

Portanto
    \begin{align*}
        \bar{v}(\vb{p},s)v(\vb{p},s') &\eq \dfrac{1}{E_{\vb{p}}+m}\qty[
            \chi^{\dagger}_{s}(\boldsymbol{\sigma}\cdot\vb{p})^{\dagger}(\boldsymbol{\sigma}\cdot\vb{p})\chi_{s'} - (E_{\vb{p}}+m)^2\chi^{\dagger}_{s}\chi_{s'}
        ] \\
        &\eq \dfrac{1}{E_{\vb{p}}+m}\qty[
            (E_{\vb{p}}+m)(E_{\vb{p}}-m)\delta_{ss'} - (E_{\vb{p}}+m)^2\delta_{ss'}
        ] \\
        &\eq (E_{\vb{p}}-m)\delta_{ss'} - (E_{\vb{p}}+m)\delta_{ss'} \\
        &\eq -2m\delta_{ss'}
    \end{align*}

Concluindo que
    \begin{answer}\label{eq: spinor relation for antiparticle}
        -\bar{v}(\vb{p},s) v(\vb{p},s) = 2m
    \end{answer}

\noindent \textbf{(b)} Usando a definição do projetor, temos
    \begin{align*}
        \Big[\Lambda^{+}(\vb{p})\Big]_{\alpha\beta} &\eq \dfrac{1}{2m}\sum_{s}u_{\alpha}(\vb{p},s)\bar{u}_{\beta}(\vb{p},s) \\
        &\eq \dfrac{1}{2m(E_{\vb{p}}+m)}\sum_{s}
        \begin{bmatrix}
            (E_{\vb{p}}+m)\chi_{s} \\
            (\boldsymbol{\sigma}\cdot\vb{p})\chi_{s}
        \end{bmatrix}_{\alpha} 
        \begin{bmatrix}
            (E_{\vb{p}}+m)\chi^{\dagger}_{s} & 
            -\chi^{\dagger}_{s}(\boldsymbol{\sigma}\cdot\vb{p})^{\dagger}
        \end{bmatrix}_{\beta} \\
        &\eq \dfrac{1}{2m(E_{\vb{p}}+m)} \sum_{s}
        \begin{bmatrix}
            (E_{\vb{p}} + m)^2 \chi_{s}\chi^{\dagger}_{s} & 
            -(E_{\vb{p}}+m)\chi_{s}\chi^{\dagger}_{s}(\boldsymbol{\sigma}\cdot\vb{p})^{\dagger} \\
            (E_{\vb{p}}+m)(\boldsymbol{\sigma}\cdot\vb{p})\chi_{s}\chi^{\dagger}_{s} &
            -(\boldsymbol{\sigma}\cdot\vb{p}) \chi_{s}\chi_{s}^{\dagger} (\boldsymbol{\sigma}\cdot\vb{p})^{\dagger}
        \end{bmatrix}_{\alpha\beta}
    \end{align*}

Note que 
    \begin{equation*}
        \chi_{1}\chi^{\dagger}_{1} = 
        \begin{bmatrix}
            1 \\ 0 
        \end{bmatrix}
        \begin{bmatrix}
            1 & 0
        \end{bmatrix} = 
        \begin{bmatrix}
            1 & 0 \\
            0 & 0
        \end{bmatrix} \qquad \& \qquad 
        \chi_{2}\chi^{\dagger}_{2} = 
        \begin{bmatrix}
            0 \\ 1 
        \end{bmatrix}
        \begin{bmatrix}
            0 & 1
        \end{bmatrix} = 
        \begin{bmatrix}
            0 & 0 \\
            0 & 1
        \end{bmatrix}
    \end{equation*}

Implicando que quando somarmos em $s$, estaremos somando estas duas matrizes, gerando a identidade $\boldone_{2\times2}$, portanto o projetor fica
    \begin{align*}
        \Big[\Lambda^{+}(\vb{p})\Big]_{\alpha\beta} &\eq \dfrac{1}{2m(E_{\vb{p}}+m)}
        \begin{bmatrix}
            (E_{\vb{p}}+m)^2 \boldone_{2\times2} & 
            -(E_{\vb{p}}+m)\boldone_{2\times2}(\boldsymbol{\sigma}\cdot\vb{p})^{\dagger} \\
            (E_{\vb{p}}+m)(\boldsymbol{\sigma}\cdot\vb{p})\boldone_{2\times2} & 
            -(\boldsymbol{\sigma}\cdot\vb{p})\boldone_{2\times2}(\boldsymbol{\sigma}\cdot\vb{p})^{\dagger}
        \end{bmatrix}_{\alpha\beta} \\
        &\eq \dfrac{1}{2m}
        \begin{bmatrix}
            (E_{\vb{p}}+m)\boldone_{2\times2} & 
            -\boldone_{2\times2}(\boldsymbol{\sigma}\cdot\vb{p})^{\dagger} \\
            (\boldsymbol{\sigma}\cdot\vb{p})\boldone_{2\times2} &
            -\boldone_{2\times2}(E_{\vb{p}} - m)
        \end{bmatrix}_{\alpha\beta} \\
        &\eq \dfrac{1}{2m}
        \begin{bmatrix}
            E_{\vb{p}}\boldone_{2\times2} & -(\boldsymbol{\sigma}\cdot\vb{p})\boldone_{2\times2} \\
            (\boldsymbol{\sigma}\cdot\vb{p})\boldone_{2\times2} & -E_{\vb{p}}\boldone_{2\times2}
        \end{bmatrix}_{\alpha\beta} + \dfrac{1}{2m}
        \begin{bmatrix}
            m\boldone_{2\times2} & 0 \\
            0 & m\boldone_{2\times2}
        \end{bmatrix}_{\alpha\beta}
    \end{align*}

Como $p_{0} = E_{\vb{p}}$, a forma explicita deste projetor é
    \begin{align*}
        \Big[\Lambda^{+}(\vb{p})\Big]_{\alpha\beta} &\eq \dfrac{1}{2m}
        \begin{bmatrix}
            p_{0} & 0 & -p_{3} & -p_{1}+ip_{2} \\
            0 & p_{0} & -p_{1}-ip_{2} & p_{3} \\
            p_{3} & p_{1}-ip_{2} & -p_{0} & 0 \\
            p_{1}+ip_{2} & -p_{3} & 0 & -p_{0}
        \end{bmatrix}_{\alpha\beta} +
        \dfrac{1}{2m}
        \begin{bmatrix}
            m & 0 & 0 & 0 \\
            0 & m & 0 & 0 \\
            0 & 0 & m & 0 \\
            0 & 0 & 0 & m
        \end{bmatrix}_{\alpha\beta} \\
        &\eq \dfrac{1}{2m}(\gamma^{0}p_{0} + \gamma^{1}p_{1} + \gamma^{2}p_{2} + \gamma^{3}p_{3})_{\alpha\beta} + \dfrac{1}{2m} (m)_{\alpha\beta} \\
        &\eq \dfrac{1}{2m}(\gamma^{\mu}p_{\mu} + m)_{\alpha\beta}
    \end{align*}

Concluindo que  
    \begin{answer}\label{eq: positive energy projector}
        \Big[\Lambda^{+}(\vb{p})\Big]_{\alpha\beta} = \dfrac{1}{2m}\displaystyle\sum_{s} u_{\alpha}(\vb{p},s) \bar{u}_{\beta}(\vb{p},s) = \dfrac{1}{2m}(\slashed{p} + m)_{\alpha\beta}
    \end{answer}

\noindent \textbf{(c)} Novamente, usando a forma explícita do projetor, temos
    \begin{align*}
        \Big[\Lambda^{-}(\vb{p})\Big]_{\alpha\beta} &\eq -\dfrac{1}{2m}\sum_{s} v_{\alpha}(\vb{p},s)\bar{v}_{\beta}(\vb{p},s) \\
        &\eq \dfrac{1}{2m(E_{\vb{p}}+m)}\sum_{s}
        \begin{bmatrix}
            (\boldsymbol{\sigma}\cdot\vb{p})\chi_{s} \\
            (E_{\vb{p}}+m)\chi_{s}
        \end{bmatrix}_{\alpha}
        \begin{bmatrix}
            \chi^{\dagger}_{s}(\boldsymbol{\sigma}\cdot\vb{p})^{\dagger}_{\beta} & 
            -(E_{\vb{p}}+m)\chi^{\dagger}_{s}
        \end{bmatrix}_{\beta} \\
        &\eq -\dfrac{1}{2m(E_{\vb{p}}+m)}\sum_{s}
        \begin{bmatrix}
            (\boldsymbol{\sigma}\cdot\vb{p})\chi_{s}\chi^{\dagger}_{s}(\boldsymbol{\sigma}\cdot\vb{p})^{\dagger} & -(E_{\vb{p}}+m)(\boldsymbol{\sigma}\cdot\vb{p})\chi_{s}\chi^{\dagger}_{s} \\
            (E_{\vb{p}}+m)\chi_{s}\chi^{\dagger}_{s}(\boldsymbol{\sigma}\cdot\vb{p})^{\dagger} & 
            -(E_{\vb{p}}+m)^2 \chi_{s}\chi^{\dagger}_{s}
        \end{bmatrix}_{\alpha\beta} \\
        &\eq -\dfrac{1}{2m}
        \begin{bmatrix}
            \boldone_{2\times2}(E_{\vb{p}} - m) & 
            -(\boldsymbol{\sigma}\cdot\vb{p})\boldone_{2\times2} \\
            \boldone_{2\times2}(\boldsymbol{\sigma}\cdot\vb{p}) &
            -(E_{\vb{p}} + m)\boldone_{2\times2}
        \end{bmatrix}_{\alpha\beta} \\
        &\eq -\dfrac{1}{2m}
        \begin{bmatrix}
            E_{\vb{p}}\boldone_{2\times2} & -(\boldsymbol{\sigma}\cdot\vb{p})\boldone_{2\times2} \\
            (\boldsymbol{\sigma}\cdot\vb{p})\boldone_{2\times2} & -E_{\vb{p}}\boldone_{2\times2}
        \end{bmatrix}_{\alpha\beta} - \dfrac{1}{2m}
        \begin{bmatrix}
            -m \boldone_{2\times 2} & 0 \\
            0 & -m \boldone_{2\times2}
        \end{bmatrix}_{\alpha\beta} \\
        &\eq -\dfrac{1}{2m}
        \begin{bmatrix}
            p_{0} & 0 & -p_{3} & -p_{1} + ip_{2} \\
            0 & p_{0} & -p_{1} - ip_{2} & p_{3} \\
            p_{3} & p_{1} - ip_{2} & -p_{0} & 0 \\
            p_{1} + ip_{2} & -p_{3} & 0 & -p_{0}
        \end{bmatrix}_{\alpha\beta} +
        \dfrac{1}{2m}
        \begin{bmatrix}
            m & 0 & 0 & 0 \\
            0 & m & 0 & 0 \\
            0 & 0 & m & 0 \\
            0 & 0 & 0 & m
        \end{bmatrix}_{\alpha\beta} \\
        &\eq \dfrac{1}{2m}(-\gamma^{0}p_{0} + \gamma^{1}p_{1} + \gamma^{2}p_{2} + \gamma^{3}p_{3})_{\alpha\beta} + \dfrac{1}{2m}(m)_{\alpha\beta} \\
        &\eq \dfrac{1}{2m}(-\gamma^{\mu}p_{\mu} + m)_{\alpha\beta}
    \end{align*}

Concluindo que
    \begin{answer}\label{eq: negative energy projector}
        \Big[\Lambda^{-}(\vb{p})\Big]_{\alpha\beta} = -\dfrac{1}{2m}\displaystyle\sum_{s} v_{\alpha}(\vb{p},s) \bar{v}_{\beta}(\vb{p},s) = \dfrac{1}{2m}(-\slashed{p} + m)_{\alpha\beta}
    \end{answer}