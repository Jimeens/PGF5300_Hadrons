Na expressão original do problema, podemos identificar que o termo $\dfrac{1}{2}\xi\vec{K}_{x}\xi$ é um termo quadrático em $\xi$, a soma dos termos $\dfrac{1}{2}\xi J + \dfrac{1}{2}\bar{\phi}\vec{K}_{x}\xi + J\xi$ é um termo linear em $\xi$ e os termos restantes independem dessa variável, o que sugere uma integração gaussiana em $\xi$.

\begin{itemize}
    \item Assumir que $\vec{K}_{x}$ é uma matriz diagonal
    \item Discretizar os campos pra usar a expressão da aula 7/8 (eu acho)
    \item Fazer uma mudança de variável conveniente
    \item Integrar em $[D\xi]$
\end{itemize}