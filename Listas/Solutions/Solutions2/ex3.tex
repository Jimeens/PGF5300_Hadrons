\noindent\textbf{(a)} Usando o gauge $D_{\mu}\Phi = \qty(\partial_{\mu} - ig\vb{A}_{\mu})\Phi$, onde $\vb{A}_{\mu} = A_{\mu}^{a}T^{a}$ e $T^{a}$ são os geradores do grupo, temos
    \begin{align*}
        [D_{\mu}, D_{\nu}]\Phi &\eq D_{\mu}(D_{\nu}\Phi) - D_{\nu}(D_{\mu}\Phi) \\
        &\eq D_{\mu}\qty(\partial_{\nu}\Phi - ig\vb{A}_{\nu}\Phi) - D_{\nu}\qty(\partial_{\mu}\Phi - ig\vb{A}_{\mu}\Phi) \\
        &\eq \qty(\partial_{\mu} - ig\vb{A}_{\mu})\qty(\partial_{\nu}\Phi - ig\vb{A}_{\nu}\Phi) - \qty(\partial_{\nu} - ig\vb{A}_{\nu})\qty(\partial_{\mu}\Phi - ig\vb{A}_{\mu}\Phi) \\
        &\eq {\color{orange}\partial_{\mu}\partial_{\nu}\Phi} - ig\vb{A}_{\mu}(\partial_{\nu}\Phi) - ig\partial_{\mu}(\vb{A}_{\nu}\Phi) - g^{2}\vb{A}_{\mu}\vb{A}_{\nu}\Phi - {\color{orange}\partial_{\nu}\partial_{\mu}\Phi} + ig\vb{A}_{\nu}(\partial_{\mu}\Phi) + \\
        &\noeq + ig\partial_{\nu}(\vb{A}_{\mu}\Phi) + g^{2}\vb{A}_{\nu}\vb{A}_{\mu}\Phi \\
        &\eq -{\color{orange} ig\vb{A}_{\mu}(\partial_{\nu}\Phi)} - ig(\partial_{\mu}\vb{A}_{\nu})\Phi - {\color{red!70!black} ig\vb{A}_{\nu}(\partial_{\mu}\Phi)} - g^{2}\vb{A}_{\mu}\vb{A}_{\nu}\Phi + {\color{red!70!black} ig\vb{A}_{\nu}(\partial_{\mu}\Phi)} +ig(\partial_{\nu}\vb{A}_{\mu})\Phi + \\
        &\noeq + {\color{orange} ig\vb{A}_{\mu}(\partial_{\nu}\Phi)} + g^{2}\vb{A}_{\nu}\vb{A}_{\mu}\Phi \\
        &\eq -ig\qty(\partial_{\mu}\vb{A}_{\nu} - \partial_{\nu}\vb{A}_{\mu})\Phi - g^{2}(\vb{A}_{\mu}\vb{A}_{\nu} - \vb{A}_{\nu}\vb{A}_{\mu})\Phi \\
        &\eq -ig\qty(\partial_{\mu}\vb{A}_{\nu} - \partial_{\nu}\vb{A}_{\mu} - ig[\vb{A}_{\mu},\vb{A}_{\nu}])\Phi
    \end{align*}
onde um dos cancelamentos é feito levando em conta que $\partial_{\mu}\partial_{\nu} = \partial_{\nu}\partial_{\mu}$. Como na teoria de gauge não-abeliana o tensor eletromagnético é dado por
    \begin{equation*}
        F_{\mu\nu} = \partial_{\mu}\vb{A}_{\nu} - \partial_{\nu}\vb{A}_{\mu} - ig[\vb{A}_{\mu},\vb{A}_{\nu}]
    \end{equation*}
temos
    \begin{answer}\label{eq: answer 3a}
        [D_{\mu}, D_{\nu}]\Phi = -igF_{\mu\nu}\Phi
    \end{answer}

\noindent\textbf{(b)} Por estarmos em uma teoria de gauge não-abeliana, $\vb{A}_{\mu}$ se transforma como
    \begin{equation*}
        \vb{A}_{\mu}' = U\vb{A}_{\mu}U^{-1} - \dfrac{i}{g}(\partial_{\mu}U)U^{-1}
    \end{equation*}
e o tensor eletromagnético da teoria é dado por
    \begin{equation*}
        F_{\mu\nu} = \partial_{\mu}\vb{A}_{\nu} - \partial_{\nu}\vb{A}_{\mu} - ig[\vb{A}_{\mu},\vb{A}_{\nu}]
    \end{equation*}

Portanto, transformar esse tensor vai ser
    \begin{align*}
        F_{\mu\nu}' &\eq \partial_{\mu}\vb{A}_{\nu}' - \partial_{\nu}\vb{A}_{\mu}' - ig[\vb{A}_{\mu}',\vb{A}_{\nu}'] \\
        &\eq \underbrace{\partial_{\mu}\qty[U\vb{A}_{\nu}U^{-1} - \dfrac{i}{g}(\partial_{\nu}U)U^{-1}]}_{\mathds{J}_{\mu\nu}} - 
        \underbrace{\partial_{\nu}\qty[U\vb{A}_{\mu}U^{-1} - \dfrac{i}{g}(\partial_{\mu}U)U^{-1}]}_{\mathds{K}_{\mu\nu}} - 
        \underbrace{ig(\vb{A}_{\mu}'\vb{A}_{\nu}' - \vb{A}_{\nu}\vb{A}_{\mu}')}_{\mathds{L}_{\mu\nu}} \\
        &\eq \mathds{J}_{\mu\nu} - \mathds{K}_{\mu\nu} - \mathds{L}_{\mu\nu}
    \end{align*}

Expandindo cada termo separadamente, temos
    \begin{align*}
        \mathds{J}_{\mu\nu} &\eq \partial_{\mu}(U\vb{A}_{\nu}U^{-1}) - \dfrac{i}{g}\partial_{\mu}[(\partial_{\nu}U)U^{-1}] \\
        &\eq (\partial_{\mu}U)\vb{A}_{\nu}U^{-1} + 
        U(\partial_{\mu}\vb{A}_{\nu})U^{-1} + 
        U\vb{A}_{\nu}(\partial_{\mu}U^{-1}) - 
        \dfrac{i}{g}(\partial_{\mu}\partial_{\nu}U)U^{-1} - 
        \dfrac{i}{g}(\partial_{\nu}U)(\partial_{\mu}U^{-1})
    \end{align*}

Podemos substituir a derivada $\partial_{\mu}U^{-1}$ a partir do seguinte raciocínio: sabendo que $U^{-1}U = \mathds{1}$, podemos derivar ambos os lados da equação, de modo que
    \begin{align*}
        \partial_{\mu}(U^{-1}U) &\eq \partial_{\mu}\mathds{1} \\
        (\partial_{\mu}U^{-1})U + U^{-1}(\partial_{\mu}U) &\eq 0 \\
        (\partial_{\mu}U^{-1})U &\eq -U^{-1}(\partial_{\mu}U) \\
        \partial_{\mu}U^{-1} &\eq -U^{-1}(\partial_{\mu}U)U^{-1}
    \end{align*}

Portanto, podemos reescrever $\mathds{J}_{\mu\nu}$ como
    \begin{align*}
        \mathds{J}_{\mu\nu} &\eq (\partial_{\mu}U)\vb{A}_{\nu}U^{-1} + 
        U(\partial_{\mu}\vb{A}_{\nu})U^{-1} - 
        U\vb{A}_{\nu}U^{-1}(\partial_{\mu}U)U^{-1} - 
        \dfrac{i}{g}(\partial_{\mu}\partial_{\nu}U)U^{-1} + \\
        &\noeq +\dfrac{i}{g}(\partial_{\nu}U)U^{-1}(\partial_{\mu}U)U^{-1}
    \end{align*}

No caso de $\mathds{K}_{\mu\nu}$, basta substituir $\mu\leftrightarrow\nu$ em $\mathds{J}_{\mu\nu}$, ou seja,
    \begin{align*}
        \mathds{K}_{\mu\nu} &\eq (\partial_{\nu}U)\vb{A}_{\mu}U^{-1} + 
        U(\partial_{\nu}\vb{A}_{\mu})U^{-1} - 
        U\vb{A}_{\mu}U^{-1}(\partial_{\nu}U)U^{-1} - 
        \dfrac{i}{g}(\partial_{\nu}\partial_{\mu}U)U^{-1} + \\
        &\noeq +\dfrac{i}{g}(\partial_{\mu}U)U^{-1}(\partial_{\nu}U)U^{-1}
    \end{align*}

Por fim, no caso de $\mathds{L}_{\mu\nu}$, temos
    \begin{align*}
        \mathds{L}_{\mu\nu} &\eq ig\qty[
            U\vb{A}_{\mu}U^{-1} - \dfrac{i}{g}(\partial_{\mu}U)U^{-1}
        ]\qty[
            U\vb{A}_{\nu}U^{-1} - \dfrac{i}{g}(\partial_{\nu}U)U^{-1}
        ] - ig\qty[
            U\vb{A}_{\nu}U^{-1} - \dfrac{i}{g}(\partial_{\nu}U)U^{-1}
        ]\times \\
        &\noeq \times \qty[
            U\vb{A}_{\mu}U^{-1} - \dfrac{i}{g}(\partial_{\mu}U)U^{-1}
        ] \\
        &\eq ig\qty[
            U\vb{A}_{\mu}\vb{A}_{\nu}U^{-1} - 
            \dfrac{i}{g}U\vb{A}_{\mu}U^{-1}(\partial_{\nu}U)U^{-1} - 
            \dfrac{i}{g}(\partial_{\mu}U)\vb{A}_{\nu}U^{-1} - 
            \dfrac{1}{g^2}(\partial_{\mu}U)U^{-1}(\partial_{\nu}U)U^{-1}
        ] - \\
        &\noeq -ig\qty[
            U\vb{A}_{\nu}\vb{A}_{\mu}U^{-1} - 
            \dfrac{i}{g}U\vb{A}_{\nu}U^{-1}(\partial_{\mu}U)U^{-1} - 
            \dfrac{i}{g}(\partial_{\nu}U)\vb{A}_{\mu}U^{-1} - 
            \dfrac{1}{g^2}(\partial_{\nu}U)U^{-1}(\partial_{\mu}U)U^{-1}
        ]
    \end{align*}

Colocando tudo junto, temos
    \begin{align*}
        F_{\mu\nu}' &\eq 
            {\color{orange}(\partial_{\mu}U)\vb{A}_{\nu}U^{-1}} + 
            U(\partial_{\mu}\vb{A}_{\nu})U^{-1} - 
            {\color{red!70!black}U\vb{A}_{\nu}U^{-1}(\partial_{\mu}U)U^{-1}} - 
            {\color{blue!70!black}\dfrac{i}{g}(\partial_{\mu}\partial_{\nu}U)U^{-1}} + \\
        &\noeq +
            {\color{green!70!black}\dfrac{i}{g}(\partial_{\nu}U)U^{-1}(\partial_{\mu}U)U^{-1}} - 
            {\color{magenta}(\partial_{\nu}U)\vb{A}_{\mu}U^{-1}} - 
            U(\partial_{\nu}\vb{A}_{\mu})U^{-1} + 
            {\color{yellow!70!black}U\vb{A}_{\mu}U^{-1}(\partial_{\nu}U)U^{-1}} + \\
        &\noeq +
            {\color{blue!70!black}\dfrac{i}{g}(\partial_{\nu}\partial_{\mu}U)U^{-1}} - 
            {\color{cyan}\dfrac{i}{g}(\partial_{\mu}U)U^{-1}(\partial_{\nu}U)U^{-1}} - 
            igU\vb{A}_{\mu}\vb{A}_{\nu}U^{-1} - 
            {\color{yellow!70!black}U\vb{A}_{\mu}U^{-1}(\partial_{\nu}U)U^{-1}} - \\
        &\noeq -
            {\color{orange}(\partial_{\mu}U)\vb{A}_{\nu}U^{-1}} + 
            {\color{cyan}\dfrac{i}{g}(\partial_{\mu}U)U^{-1}(\partial_{\nu}U)U^{-1}} + 
            igU\vb{A}_{\nu}\vb{A}_{\mu}U^{-1} +
            {\color{red!70!black}U\vb{A}_{\nu}U^{-1}(\partial_{\mu}U)U^{-1}} + \\
        &\noeq +
            {\color{magenta}(\partial_{\nu}U)\vb{A}_{\mu}U^{-1}} - 
            {\color{green!70!black}\dfrac{i}{g}(\partial_{\nu}U)U^{-1}(\partial_{\mu}U)U^{-1}} \\
        &\eq U(\partial_{\mu}\vb{A}_{\nu})U^{-1} - U(\partial_{\nu}\vb{A}_{\mu})U^{-1} - igU(\vb{A}_{\mu}\vb{A}_{\nu} - \vb{A}_{\nu}\vb{A}_{\mu})U^{-1} \\
        &\eq U\qty(\partial_{\mu}\vb{A}_{\nu} - \partial_{\nu}\vb{A}_{\mu} - ig[\vb{A}_{\mu},\vb{A}_{\nu}])U^{-1} \\
        &\eq UF_{\mu\nu}U^{-1}
    \end{align*}

Portanto, como $\Phi' = U\Phi$ e $U^{-1}U = \mathds{1}$, temos
    \begin{answer}\label{eq: answer 3b}
        F_{\mu\nu}'\Phi' = UF_{\mu\nu}U^{-1}U\Phi = UF_{\mu\nu}\Phi
    \end{answer}