\begin{itemize}
    \item Os diagramas de 1 loop do propagador do quark (esquerda) de do elétron (direita) são
        \begin{center}
            \begin{tikzpicture}
                \begin{feynman}
                    \vertex (i1);
                    \vertex [right=1cm of i1, dot] (i2) {};
                    \vertex [right=2cm of i2, dot] (i3) {};
                    \vertex [right=1cm of i3] (i4);

                    \diagram* {
                        (i1) -- (i2) -- [fermion] (i3) -- (i4),
                        (i2) -- [gluon, half left, looseness=1.5] (i3),
                    };
                \end{feynman}
            \end{tikzpicture}
            \hspace{2cm}
            \begin{tikzpicture}
                \begin{feynman}
                    \vertex (i1);
                    \vertex [right=1cm of i1, dot] (i2) {};
                    \vertex [right=2cm of i2, dot] (i3) {};
                    \vertex [right=1cm of i3] (i4);

                    \diagram* {
                        (i1) -- (i2) -- (i3) -- (i4),
                        (i2) -- [boson, half left, looseness=1.5] (i3),
                    };
                \end{feynman}
            \end{tikzpicture}
        \end{center}
    \item O fato de não precisar calcular a integração de loop indica que existe algum truque pra obter a resposta.
\end{itemize}